Before implementing half-binding and q-bindings, we implement the \texttt{PartialBindingCommScheme} interface 
to model the definition of partially-binding vector commitments provided in Definition \ref{def:comm_scheme}. 
This interface has methods that correspond to each function in the definition: \texttt{setup}, \texttt{gen}, 
\texttt{bind} (equivalent to \textsf{BindCom}), \texttt{equiv}, and \texttt{equivcom}. 
The interface also has a set of generic types to complete the definition of these methods (similar to how 
they function in the \texttt{SigmaProtocol} interface). These generic types are:
\begin{itemize}
  \item \texttt{PublicParams}: public parameters for the commitment scheme. 
  \item \texttt{BindingIndex}: a type that represents the binding index(es) of the commitment. 
  \item \texttt{CommitKey}: a type that represents the commitment key.
  \item \texttt{EquivKey}: a type that represents the equivocation key. 
  \item \texttt{Commitment}: a type that represents the commitment.
  \item \texttt{Randomness}: a type that represents the auxiliary values that are needed for equivocation. 
  \item \texttt{Msg}: a type that represents the vector of messages. 
\end{itemize}

With this interface, we implement \texttt{HalfBinding} and \texttt{QBinding} according to their 
construction in Figure \ref{fig:half-binding} of this report, and Figure 4 of \cite{StackingSigmas} 
respectively. The definition of q-binding is recursive in nature, where the base case is the 
half-binding scheme. In an attempt to improve the usability of our q-binding implementation, we introduce 
some overhead in how we represent the tree of half-bindings. Intuitively, this could simply be a 
recursive data structure, in which the every node has two children which continue down until the leaves. 
However, due to Rust's strict type system and our limited experience with the language, we were unable 
to implement this without heavily sacrificing the usability of the interface. 

Instead, we represent the tree as a set of vectors of data types, where each index in the vector represents a 
layer of the tree. We have multiple vectors for each concrete type for the half-binding implementation of the 
\texttt{PartialBindingCommScheme} interface. While this introduces some overhead as the vector must be 
traversed if we want to copy the data, it allows us to implement the interface in a way that has good 
usability and is still easy to understand. We hypothesize that this is the main reason our implementation of 
Stacking Sigmas is slower than Hall-Andersen's implementation \cite{MHAStackSig}. This will be discussed in 
Chapter \ref{sec:evaluation} in more detail.