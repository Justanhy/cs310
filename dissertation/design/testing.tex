As mentioned in Project Management (Chapter \ref{ch:project-management}), we use Test-Driven Development to 
ensure correctness. By writing tests before features, we encourage a focus on the 
requirements of our software and ensure that we design our components to target 
these requirements. Once these tests are written, we have an efficient workflow to 
consistently ensure our software is correct, even as we add new features. This improves
development speed and reduces the time needed to identify and fix bugs.

In addition, we perform extensive static analysis to complement testing and improve 
debugging. To support static analysis, we 
intentionally design our components and interfaces to reflect their 
mathematical definitions in the literature. A good example of this is our implementation of q-bindings and half-bindings
(Section \ref{design:qbinding}). Doing this allows us to easily verify that our implementations
adhere to their mathematical definitions as closely as possible, increasing the chances of 
identifying an issue if there is one.

\paragraph{Unit \& Integration Testing.}
We use the built-in testing framework in Rust to test our components. This framework allows us to write unit 
tests within the same source code file as our implementation, and conveniently run them within our 
IDE (VSCode) which consequently speeds up development. We also write integration tests with 
the same framework, except that they are often located in a separate file to emphasise that they are testing 
multiple components as these components have to be explicitly imported into the scope of this testing module. 

\paragraph{Code Coverage.} To complement testing, we use code coverage reports to ensure that we are testing 
as much of our code as possible. Code coverage is a way to track and determine which sections of code (e.g. 
functions, branches, lines etc.) are executed. This is useful for us as it highlights sections of code that 
are neglected by our existing tests and ensure that we cover important test cases. We use the 
\texttt{cargo-llvm-cov} \cite{cargo-llvm-cov} tool to carry out code coverage analysis. 
This tool is a wrapper around the Rust compiler's in-built code coverage tool, 
providing a convenient way to generate code coverage reports as they integrate directly 
with our testing suite. 

\subsection{Extracting Requirements \& Writing Tests}
Throughout this project, the requirements of our software components are mostly derived 
from the literature with the exception of our benchmarks. Therefore, our testing 
strategy is centred around testing these requirements against our implementation. Of 
course, additional unit tests are written to test the correctness of specific functions
as well, but we will not discuss these in detail. 

Extracting these requirements is a manual process that requires a thorough 
understanding of the literature. Our approach begins with reading the literature and
writing down the requirements explicitly. This is a tedious process,
but it is necessary to ensure that we have a complete understanding of the protocols
and what is required of them. We then translate these requirements into tests. 
\begin{enumerate}
  \item Usually, the first step is to complete a function for setting up mock data 
  that is likely to be used in the tests. We design this function to take in a set of 
  parameters and return a set of mock data. These can then be separately called in 
  each test, with parameters, to generate data specific to that test. 
  \item Next, we write the content of the tests themselves. Because we are designing our 
  components to model the mathematical definitions of the protocols, we can often easily 
  determine the exact methods to call. Often, a 
  challenge in TDD is not knowing what functions are available to test because they 
  are not yet implemented. However, this is not a problem in this project as we have 
  the definitions in the literature to refer to. There are two main kinds of tests 
  that we write:
  \begin{itemize}
    \item \textbf{Positive Tests:} These are tests that ensure the implementation
    works as expected. 
    \item \textbf{Negative Tests:} These are tests that ensure the implementation
    fails as expected.
  \end{itemize}
  \item Following this, we proceed to implement our components. This often starts with 
  the interfaces (if any), and then to any classes that are required. 
  \item Finally, we run the tests to ensure that they pass. If they do not, we 
  investigate and either fix the implementation or the tests.
\end{enumerate}

In Appendix \ref{code:cds-tests}, we provide an example of our tests with those 
for the CDS94 compiler. Often the first tests that are written target the main 
requirements that we glean from the literature. As development progresses, we 
continue to write tests or extend existing ones to ensure that we are testing 
comprehensively. 
