We model $\Sigma$-protocols with an interface.
Firstly, we design a set of generic types associated with the interface which will fit into 
the definition of the methods in the interface. These types are:
\begin{itemize}
  \item \texttt{Statement}: Public information about the protocol.
  \item \texttt{Witness}: Private information about the protocol (only known to Prover).
  \item \texttt{MessageA}: The first message of the protocol, sent from prover to verifier.
  \item \texttt{Challenge}: The challenge sent by the verifier to the prover.
  \item \texttt{MessageZ}: The third message of the protocol, sent from prover to verifier. 
  \item \texttt{State}: For most $\Sigma$-protocols, there is a particular state associated 
  to the execution of the first message. This state is often used by the prover again in 
  the third message but must not be sent to the verifier.
\end{itemize}

In particular, the design of the generic \texttt{State} type is worth discussing in more detail.
It is not explicitly mentioned in the formal definition of $\Sigma$-protocols, as it is 
assumed that the prover is able to track and store the private values that they obtain and 
require in the protocol. In the implementation, this has to be explicitly modelled and 
we do this using a functional programming approach, instead of an object-oriented
programming (OOP) approach. 
Firstly, this is because the functional approach is simpler, as we do not need to 
implement separate interfaces for the prover and verifier and subsequently include them as fields
within the \texttt{SigmaProtocol} interface. Secondly, the functional approach is more flexible,
as it does not restrict the user to a particular way of implementing the prover or verifier.
The functional approach simply provides two values in the first message (the state and the 
actual message), and the user of our interface can decide how to use these values.

Now, we present the methods that are associated with our $\Sigma$-protocol interface. 
These methods are: 
\begin{itemize}
  \item \texttt{first}: the first message of the protocol. This models algorithm 
  $a \leftarrow A(x, w; r^P)$ 
  in Definition \ref{def:sigma}.
  \item \texttt{second}: the second message of the protocol. $c \leftarrow \{0,1\}^\kappa$.
  \item \texttt{third}: the third message of the protocol. $z \leftarrow Z(x, w, c; r^P)$. 
  \item \texttt{verify}: verifies the transcript. $b \leftarrow \phi(x, a, c, z)$.
\end{itemize}

In Appendix \ref{code:SigmaProtocol} we provide a code snippet of these methods within 
the interface, which outlines which generic types are given as input and which are 
returned as output. Referring to the code snippet, readers will observe that our 
methods almost model the algorithms in Definition \ref{def:sigma} exactly in terms of 
input and output. The only difference is with the \texttt{State} type that we have already 
discussed.

\subsection{Schnorr's Protocol}
As a $\Sigma$-protocol, our first step in our implementation for Schnorr's protocol is 
to create concrete definitions for 
the methods and generic types of the \texttt{SigmaProtocol} interface. We use the following 
concrete types in the interface: 
\begin{itemize}
  \item \texttt{Statement}: \texttt{Schnorr} -- a type that simply contains the "public key"
  (which is the group element $H = x \cdot G$) as a field. We use the \texttt{RistrettoPoint} 
  type from \texttt{curve25519-dalek} to represent the public key, and it is essentially a 
  point on an Edward's curve. 
  \item \texttt{Witness}: \texttt{Scalar} -- also from the \texttt{curve22519-dalek} library. It 
  represents elements of the prime field in the Ristretto group. 
  \item \texttt{MessageA}: \texttt{CompressedRistretto} -- essentially the same as the 
  \texttt{RistrettoPoint} type, but differs in its representation. \texttt{CompressedRistretto}
  is a compressed representation of the point, which is more efficient to store and
  transmit.
  \item \texttt{Challenge}: \texttt{Scalar}
  \item \texttt{MessageZ}: \texttt{Scalar}
  \item \texttt{State}: \texttt{Scalar} 
\end{itemize}

The key takeaway is that we assign concrete types to the associated generic types provided in 
the interface for our implementation of Schnorr. The methods in the interface are then 
defined according to our definition of Schnorr in Definition \ref{def:schnorr}. 