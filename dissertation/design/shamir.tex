For our implementation of Shamir's Secret Sharing, we use Lagrange's interpolating 
polynomial to interpolate the $x$ and $y$ coordinates that correspond to shares. We 
chose to use Lagrange's interpolation formula because it is simple to implement, 
albeit not the most efficient. This trade off is acceptable as the main goal for the project 
is to measure the performance change with respect to the number of clauses, and 
not the pure performance of the compiler. A more efficient implementation is therefore 
not a priority, and can always be implemented in the future if needed. 

Initially, we used an existing library \texttt{vsss-rs} \cite{vsss-rs} as our implementation of 
Shamir's secret sharing scheme. However, we encountered a few issues with this library. Firstly, 
the library does not provide the functionality to complete a qualified shares given the secret 
and an unqualified set of shares. This is not surprising as this is not the main use case for 
a secret sharing scheme like Shamir's. However, this is a necessary feature for our compiler
as we use it to ensure that the prover is unable to arbitrarily select challenges to cheat the 
verifier. We initially tackled this by simply adding the missing functionality to the library, 
in the form of new functions. However, we later found out about the second issue with the library:
its implementation of Lagrange's algorithm is inefficient.

In their implementation, they call the \texttt{invert} 
method in the inner-loop of the algorithm resulting in $n^2$ calls to the function, 
where $n$ is the number of shares. The \texttt{invert} method is required to perform 
division on our group elements, but it is a costly operation. 
This can be avoided, however, by computing the numerator and denominator 
separately and performing the \texttt{invert} function only $n$ times -- outside of the inner-loop. 
The following snippet of code shows the comparison of the two implementations. 

\begin{lstlisting}[language=rust]
  // Inefficient implementation
  for i in 0..n { // outer-loop
    for j in 0..n { // inner-loop
      if i != j
        basis *= (x - x[j]) * (x[i] - x[j]).invert()
      result += basis * y[i]
    }
  }

  // More efficient implementation
  for i in 0..n {
    for j in 0..n {
      if i != j {
        numerator *= x - x[j]
        denominator *= x[i] - x[j]
      }
    }
    result += numerator * denominator.invert() * y[i]
  }
\end{lstlisting}

Due to this issue, we created our own implementation of Shamir's secret sharing scheme. 
We represent a Lagrange polynomial with two fields: 
$x$-coordinates and $y$-coordinates which are both vectors of the same length. 
For a polynomial with degree $d$, the vectors will have length $d$ as well. 

An alternative we considered is to represent the polynomial with a vector of coefficients, 
where the $i$-th element of the vector is the coefficient of the $i$-th degree term. This representation 
is useful in the method for distributing shares given the secret, as we can use Horner's method 
to evaluate the polynomial at a given point efficiently \cite{horner}. However, CDS94 does not 
rely on the secret sharing scheme's functionality for distributing shares, and instead uses
the \textsf{Complete} algorithm (Definition \ref{def:sss-completion}) to complete a qualified set.
For this algorithm, we need to use polynomial interpolation as we are given the secret and an unqualified
set of shares (and equivalently points on the polynomial). 
Therefore, we choose to represent the polynomial in this way because it is more natural for use in 
Lagrange's interpolation formula (Definition \ref{def:lagrange}). 