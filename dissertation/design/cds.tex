We implement the CDS94 compiler for disjunctions over the same $\Sigma$-protocol. This means we are concerned 
with many instances of the same $\Sigma$-protocol, even though it is possible to use CDS94 to compile 
disjunctions over different $\Sigma$-protocols. We do this to simplify our implementation and to focus on the 
most basic implementation of the CDS94 compiler. The disjunctions that this compiler supports are $d$-out-of-$n$
disjunctions, where $d$ is the minimum number of active clauses, and $n$ is the total number of clauses.
At a high level, the design of the CDS94 compiler is not complex, and simply follows the construction provided 
in Protocol \ref{prot:cds-compiler}. That said, we make certain design decisions that are worth highlighting.

Firstly, we cannot only implement the \texttt{SigmaProtocol} interface for the underlying $\Sigma$-protocol
that we want to use with the CDS94 compiler.
This is because the CDS94 compiler requires further restrictions on the generic 
types that is used during a concrete implementation. Notably, Shamir's secret sharing scheme requires that 
the shares passed to it have $x$ and $y$ coordinates that are field elements. This means we need to 
provide a way to map the generic \texttt{Challenge} type (from \texttt{SigmaProtocol} interface) to a 
type that implements the \texttt{PrimeField} interface from the \texttt{group} library (Section \ref{sec:libraries}).
For Schnorr's protocol, this is not a theoretical issue because the challenges that are used are field elements
and can be used directly with Shamir's secret sharing. However, in practice, the \texttt{Scalar} type from the
\texttt{curve25519-dalek} library does not implement the required interface. 

An alternative is to directly require that the \texttt{Challenge}
type parameter in the \texttt{SigmaProtocol} interface to implement the \texttt{PrimeField} interface. However,
this is arguably more restrictive than necessary. This is because, we may have existing implementations of a 
$\Sigma$-protocol that does not use a challenge that is a field element, but we still want to use it with the
CDS94 compiler. In this case, we can simply define a mapping function that maps the challenge to a field element
and use it with Shamir's secret sharing. 
Hence, to capture this relationship, we introduce the \texttt{Shareable} interface.

\paragraph{\texttt{Shareable}.} A simple interface to ensure type-level compatability between 
the generic types used in the CDS94 compiler and the generic types used in Shamir's secret sharing scheme.
This interface is defined as follows: 
\begin{lstlisting}[language=rust]
  pub trait Shareable: Clone + Default {
      type F: PrimeField;
      // Map the type to a prime field element
      fn share(&self) -> Self::F;
      // Derive the instance of the type from a field element
      fn derive(elem: Self::F) -> Self;
      // Convert field element into usize
      fn to_usize(elem: Self::F) -> usize;
  } 
\end{lstlisting}
\begin{itemize}
  \item The concrete type that implements this interface can be any type that also implements the \texttt{Clone} and
  \texttt{Default} traits. 
  \item The \texttt{F} type parameter is a type that implements the \texttt{PrimeField}
  interface from the \texttt{group} library. 
  \item The \texttt{share} method is a function that maps the generic
  type to the \texttt{F} type parameter.
\end{itemize}

This allows developers to implement the \texttt{Shareable} interface for any type that they want to use in the
CDS94 compiler, as long as it is possible to define a mapping function from the chosen type to one that 
implements the \texttt{PrimeField} interface. 
For example, in the case of Schnorr's protocol, the \texttt{Challenge} type is a \texttt{Scalar}
from the \texttt{curve25519-dalek} library. This type does not implement the \texttt{PrimeField} interface, but
we can implement the \texttt{Shareable} interface for it as follows:
\begin{lstlisting}[language=rust]
  impl Shareable for Scalar {
      type F = WrappedScalar;
      // ...
  }
\end{lstlisting}

where \texttt{WrappedScalar} is a wrapper type that implements the \texttt{PrimeField} interface for the 
\texttt{Scalar} type\footnote{We omit the implementation of the methods as they are specific to the 
\texttt{WrappedScalar} type and are not relevant to the discussion here.}.


\paragraph{\texttt{Composable}.} With the \texttt{Shareable} interface, we can now implement the \texttt{Composable}
interface for the base $\Sigma$-protocol to be compiled by CDS94. This interface is defined as follows:

\begin{lstlisting}[language=rust]
  pub trait Composable: SigmaProtocol<Challenge: Shareable> + HVzk {}
\end{lstlisting}

This is a simple interface that enforces the requirement for the base protocol to implement 
the \texttt{SigmaProtocol} interface, with a \texttt{Challenge} type implementing the \texttt{Shareable} 
interface, and to also implement the \texttt{HVzk} interface.

\paragraph{\texttt{CDS94}.} With the \texttt{Shareable} and \texttt{Composable} interfaces, we can now implement
the \texttt{CDS94} compiler. We require that the base $\Sigma$-protocol to be compiled implements the
\texttt{Composable} interface. Let this generic type corresponding to the base $\Sigma$-protocol be 
denoted by \texttt{S}. With \texttt{S}, we can define the generic types associated with the \texttt{SigmaProtocol} interface 
for CDS94:

\begin{itemize}
  \item \texttt{Statement}: \texttt{Statement94<S>} -- a type that contains public information 
  for the disjunctive proof: the number of clauses, the threshold, and the statement for each 
  clause of type \texttt{S}.
  \item \texttt{Witness}: \texttt{Witness94<S>} -- a type containing the private information for the 
  disjunctive proof: the witnesses for each clause, and the indexes of the active clauses 
  \item \texttt{MessageA}: \texttt{Vec<S::MessageA>} -- a vector of messages corresponding to the 
  first message in the $\Sigma$-protocol \texttt{S} for each clause.
  \item \texttt{Challenge}: \texttt{S::Challenge} -- the challenge type for the base $\Sigma$-protocol \texttt{S}.
  \item \texttt{MessageZ}: \texttt{Vec<(usize, S::Challenge, S::MessageZ)>} -- A vector of tuples containing 
  the index of the clause, the challenge corresponding to the clause, and the third message \texttt{z} 
  of that clause.
  \item \texttt{State}: \texttt{State94<S>} -- the relevant information generated from the first round
  of the disjunctive protocol that should be private to the Prover. This 
  contains the state of each instance of the base $\Sigma$-protocol (in each active clause), 
  the simulated challenge and 
  third messages $c_i$ and $z_i$ respectively for each inactive clause. 
\end{itemize}

With these generic types, we implement the methods according to the construction in Protocol \ref{prot:cds-compiler}.
Interested readers can compare how our implementation models the construction by referring to the source code 
in Appendix \ref{code:cds}.
