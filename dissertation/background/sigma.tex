\section{$\Sigma$-protocols}
\begin{definition}[$\Sigma$-Protocol \cite{StackingSigmas}]\label{def:sigma}
$\Pi = (A, Z, \phi)$ is a $\Sigma$-protocol for $\mathcal R$ if it consists of the three probabilistic polynomial time algorithms $A$, $Z$, and $\phi$, and is a 3-round protocol where a prover algorithm $P$ and a verifier algorithm $V$ exchange messages. We provide the description of each algorithm below; $x$ is a statement, $w \in w(x)$ is the witness for $x$, and $r^P$ is some form of randomness provided by the prover (usually a random number generator):
\begin{itemize}
    \item $a \leftarrow A(x,w; r^p)$: With $x$, $w$, and $r^P$ as input, returns the message from the first round of the protocol, $a$. This algorithm is executed by the prover.
    \item $c \samplefrom \{0,1\}^\kappa$: With the statistical parameter $\kappa$, generate a random challenge $c$ as the message for the second round. This is executed by the verifier.
    \item $z \leftarrow Z(x,w,c; r^p)$: On $x$, $w$, $c$, and $r^P$ as input, returns the message for the third round, $z$. This algorithm is executed by the prover.
    \item $b \leftarrow \phi(x,a,c,z)$: On $x$, $a$, $c$, and $z$ as input, returns a bit $b \in \{0,1\}$. The verifier executes $\phi$ and it returns $b = 1$ if the transcript is consistent with the statement, and $b = 0$ otherwise.
\end{itemize}
A $\Sigma$-protocol has the following properties:
\begin{enumerate}
    \item \textbf{Completeness.} For any $x$, $w$, and $r^p$, $\phi(x,a,c,z) = 1$ if $(a,c,z)$ is a valid transcript for $x$.
    $$
    \begin{gathered}
        Pr\left[\phi(x,a,c,z) = 1 \st a \leftarrow A(x,w;r^p); c\samplefrom \{0,1\}^\kappa; z \rightarrow Z(x,w,c;r^p)\right] = 1
    \end{gathered}
    $$
    \item \textbf{Special Soundness.} There exists a PPT algorithm called the extractor $\mathcal E$. $\mathcal E$ can compute an element of $w(x)$ when given any two transcripts $(a,c,z)$ and $(a,c',z')$ for statement $x$, where $c \ne c'$ and $\phi(x,a,c,z) = \phi(x,a,c',z') = 1$. Soundness is concerned with ensuring that a prover cannot cheat -- the verifier will always reject if the transcript if invalid.
    
    \item \textbf{Special Honest-Verifier Zero-Knowledge (SHVZK).} $\Pi$ is SHVZK if there is a special PPT simulator $\shvzk(1^\lambda, x, c^*)$ that takes the challenge $c^*$ in addition to the statement $x$. For any $x$ and $w$, the distribution over the output of $\shvzk$ is indistinguishable from the distribution over transcripts produced by an interaction of the prover and the verifier when the challenge is $c^*$.
    \begin{multline*}
        \{(a, z) \mid c^* \samplefrom \{0,1\}^\kappa; (a,z) \leftarrow \shvzk(1^\lambda,x,c^*)\} 
        \approx_{c^*} \\
        \{(a,z) \mid r^p \samplefrom \{0,1\}^\lambda; a \leftarrow A(x,w;r^p); c^* \samplefrom \{0,1\}^\kappa; z \leftarrow Z(x,w,c^*;r^p)\}
    \end{multline*}
\end{enumerate}
\end{definition}

\begin{definition}[Witness Indistinguishable -- WI \cite{10.1145/100216.100272}]\label{def:wi}
To say that a $\Sigma$-protocol is witness indistinguishable over $\mathcal R$ means that for any verifier $V'$, any sufficiently large input $x$, any two witnesses $w_1$ and $w_2$ for $x$, and any fixed challenge $c^*$, the resulting transcripts $(a_1, c, z_1)$ and $(a_2,c,z_2)$ are indistinguishable. Here, $a_i$ is the first round message, and $z_i$ is the third round message. This property guarantees that the prover does not reveal any information about which of the two witnesses corresponds to the true statement (the active clause).
\end{definition}

\begin{definition}[Informal definition of Witness Hiding -- WH \cite{10.1145/100216.100272}]\label{def:wh}
A $\Sigma$-protocol is witness hiding over a generator $\mathcal G$ if even a cheating verifier cannot calculate the witness for $x$ with a significant probability. This generator $\mathcal G$ produces pairs $(x,w) \in \mathcal R$ when $x$ is given as input. This property is weaker than general zero-knowledge because it only guarantees that the verifier cannot learn about the witness, not anything else. However, witness hiding can replace zero-knowledge in many protocol constructions, especially in $\Sigma$-protocols where the witness is the only information to be hidden. See \cite{10.1145/100216.100272} for more information.
\end{definition}

