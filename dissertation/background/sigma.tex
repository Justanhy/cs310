\section{$\Sigma$-protocols}
\begin{definition}[$\Sigma$-Protocol \cite{StackingSigmas}]\label{def:sigma}
A $\Sigma$-protocol $\Pi = (A, Z, \phi)$ for $\mathcal R$ is a 3-round protocol between a prover algorithm $P$ and a verifier algorithm $V$. The protocol consists of a tuple of probabilistic polynomial time algorithms $(A, Z, \phi)$ with the following interfaces:
\begin{itemize}
    \item $a \leftarrow A(x,w; r^p)$ : Given statement $x$, witness $w \in w(x)$, and prover randomness $r^p$ as input; output the first message $a$ that $P$ sends to $V$ in the first round. 
    \item $c \samplefrom \{0,1\}^\kappa$: $V$ samples a random challenge $c$ and sends it to $P$ in the second round. 
    \item $z \leftarrow Z(x,w,c; r^p)$: Given $x$, $w$, $c$, and $r^p$ as input; output the message $z$ that $P$ sends to $V$ in the third round.
    \item $b \leftarrow \phi(x,a,c,z)$: Given $x$, and the messages in the transcript, output a bit $b \in \{0,1\}$. This algorithm is executed by $V$, and $V$ accepts if $b = 1$.
\end{itemize}
A $\Sigma$-protocol has the following properties:
\begin{enumerate}
    \item \textbf{Completeness.} $\Pi$ is complete if for any $x$, $w \in w(x)$, and any prover randomness 
    $r^p \samplefrom \{0,1\}^\lambda$, the verifier accepts with probability 1. A proof is complete if for all valid 
    transcripts $(a,c,z)$, $\phi(x,a,c,z) = 1$.
    $$
    \begin{gathered}
        Pr\left[\phi(x,a,c,z) = 1 \st a \leftarrow A(x,w;r^p); c\samplefrom \{0,1\}^\kappa; z \rightarrow Z(x,w,c;r^p)\right] = 1
    \end{gathered}
    $$
    \item \textbf{Special Soundness.} $\Pi$ is said to have special soundness if there exists a PPT extractor 
    $\mathcal E$, such that given any two transcripts $(a,c,z)$ and $(a,c',z')$ for statement $x$, where $c \ne c'$ 
    and $\phi(x,a,c,z) = \phi(x,a,c',z') = 1$, an element of $w(x)$ can be computed by $\mathcal E$. Soundness is concerned 
    with ensuring that a prover cannot cheat -- the verifier will always reject if the transcript is invalid.
    \item \textbf{Special Honest-Verifier Zero-Knowledge (SHVZK).} $\Pi$ is SHVZK if there exists a PPT simulator $\mathcal S$, such that for any $x$, $w$, $(x,w) \in \mathcal R$, the distribution over the output $\mathcal S(1^\lambda, x, c^*)$ is indistinguishable from the distribution over transcripts produced by the interaction between $V$ and $P$ when the challenge is $c^*$.
    \begin{multline*}
        \{(a, z) \mid c^* \samplefrom \{0,1\}^\kappa; (a,z) \leftarrow \mathcal S(1^\lambda,x,c^*)\} 
        \approx_{c^*} \\
        \{(a,z) \mid r^p \samplefrom \{0,1\}^\lambda; a \leftarrow A(x,w;r^p); c^* \samplefrom \{0,1\}^\kappa; z \leftarrow Z(x,w,c^*;r^p)\}
    \end{multline*}
\end{enumerate}
\end{definition}

\begin{definition}[Witness Indistinguishable -- WI]\label{def:wi}
A $\Sigma$-protocol is witness indistinguishable over $\mathcal R$ if for any $V'$, any large enough input $x$, any $w_1,w_2 \in w(x)$, and for any fixed challenge $c^*$, the distribution over transcripts in the form $(a_1, c, z_1)$ and $(a_2,c,z_2)$ are indistinguishable, where $a_i \leftarrow A(x,w_i;r^p)$ and $z_i \leftarrow Z(x,w_i, c^*; r^p)$ for $i \in \{1,2\}$. This means that the prover reveals no information about which are the active clauses. 
\end{definition}

\begin{definition}[Informal definition of Witness Hiding -- WH]\label{def:wh}
For any $x$ that is generated with a certain probability distribution by a generator $\mathcal G$ which outputs pairs 
$(x,w) \in \mathcal R$, a $\Sigma$-protocol is witness hiding over $\mathcal G$, if it does not help even a cheating 
verifier to compute a witness for $x$ with non-negligible probability. Refer to \cite{10.1145/100216.100272} for details. 
WH is a weaker property than general zero-knowledge, as it only asserts that the verifier cannot learn about the witness 
(not asserting anything about other information). That said, it can replace zero-knowledge in many protocol constructions, 
as it is in most $\Sigma$-protocols where the witness is the only private information to hide.

\end{definition}

