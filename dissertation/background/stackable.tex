\subsection{Stacking Sigmas Compiler}
In Section 6 of \cite{StackingSigmas}, Goel \emph{et al} introduce the two main properties of stackable $\Sigma$-protocols: \textit{extended honest verifier 
zero-knowledge}, and \textit{recyclable third round messages}. Moreover, they go on to show that many $\Sigma$-protocols satisfy these properties, 
proving that all $\Sigma$-protocols can be made EHVZK (Observation 1), and many natural $\Sigma$-protocols have recyclable third-round messages. 

In this section, we state the definition of these two properties as shown in \cite{StackingSigmas}, but do not dive into the details of how a large class of 
$\Sigma$-protocols have these two properties. We encourage readers to refer to \cite{StackingSigmas} for the proofs of these properties and the observations 
related to them.

\begin{definition}[Extended Honest Verifier Zero-Knowledge -- EHVZK]
  The $\Sigma$-protocol $\Pi = (A, Z, \phi)$ is EHVZK if there exists a \textit{deterministic} polynomial time "extended simulator" algorithm 
  $\mathcal S^{\text{\tiny EHVZK}}(1^\lambda, x, c, z)$ such that an efficiently samplable distribution 
  $\mathcal D_{x,c}^{(z)}$ exists, for any $(x,w) \in \mathcal R$ and $c \in \{0,1\}^\kappa$. 
  $\mathcal D_{x,c}^(z)$ is a distribution of third round message $z$ conditioned on $x$ and $c$:
  \begin{multline*}
    \left\{
      (a,c,z) \st r^P \samplefrom \{0,1\}^\lambda; a \leftarrow A(x, w; r^P); z \leftarrow Z(x,w,c; r^P)
    \right\} \\
    \approx
    \left\{
      (a,c,z) \st z \samplefrom \mathcal D_{x,c}^{(z)}; a \leftarrow \mathcal S^{\text{\tiny EHVZK}}(1^\lambda, x, c, z)
    \right\}
  \end{multline*}
  Intuitively this means, that the such that the transcript $(a,c,z)$ is indistinguishable from the output of the simulator
  $\mathcal S^{\text{\tiny EHVZK}}(1^\lambda, x, c, z)$.
\end{definition}

\begin{definition}[Recyclable 3rd Round Messages]
  A $\Sigma$-protocol $\Pi = (A, Z, \phi)$ for $\mathcal R$ has recyclable third round messages if for every challenge $c \in \{0,1\}^\kappa$, there is an 
  efficiently sampleable distribution $\mathcal D_c^{(z)}$:
  $$
    \mathcal D_c^{(z)} \approx \left\{z \st r^P \samplefrom \{0,1\}^\lambda; a \leftarrow A(x,w;r^P); z \leftarrow Z(x,w,c;r^P) \right\}\quad 
    \forall (x,w) \in \mathcal R 
  $$
\end{definition}
  Essentially, this means that the distribution of $z$ is not dependent on the statement $x$ and hence does not "leak information" about the statement. 
  Consequently, this means that $z$ can be reused for any statement $x$, even non-active clauses, and still hide the active clauses. 

\begin{definition}[Stackable $\Sigma$-protocol -- Definition 9 of \cite{StackingSigmas}]
  \label{def:stackable}
  A $\Sigma$-protocol $\Pi = (A, Z, \phi)$ is \textit{stackable} if it is EHVZK and has recyclable third-round messages.
\end{definition}

  Now, we present the Stacking Sigmas Compiler for stacking disjunctions of the same protocol (Section 7 of \cite{StackingSigmas}). The compiler 
  presented in Protocol \ref{prot:stacksig-compiler} is identical to that in Figure 5 of \cite{StackingSigmas}, and is produced by Theorem 5 of \cite{StackingSigmas}.

  \paragraph{Theorem 5 of \cite{StackingSigmas}.} Given a \textit{stackable} $\Sigma$-protocol $\Pi = (A, Z, \phi)$ and a 
  1-out-of-$l$ binding vector commitment scheme, we can produce a compiled $\Sigma$-protocol $\Pi' = (A', Z', \phi')$ that is also \textit{stackable}. If 
  $\Pi$ is for the relation $\mathcal R: \mathcal X \times \mathcal W \rightarrow \{0,1\}$, then $\Pi'$ is for the relation $\mathcal R': \mathcal X^l \times 
  ([l] \times \mathcal W) \rightarrow \{0,1\}$, where $\mathcal R'((x_1,\ldots, x_n), (a,w)) := \mathcal R(x_a, w)$.
  The 
  description of $\Pi'$ is shown in Protocol $\ref{prot:stacksig-compiler}$, and the proof of this theorem can be found in \cite{StackingSigmas}.

\begin{protocol}[label={prot:stacksig-compiler}]{Stacking Sigmas Compiler. A compiler for composing $n$ instances of a 
  $\Sigma$-protocol $\Pi$ into a single $\Sigma$-protocol $\Pi'$ that proves the \textbf{disjunction} 
  of these $n$ instances.}
  \textbf{Statement:} $x = x_1,\ldots, x_n$ \\
  \textbf{Witness:} $w = (\alpha, w_\alpha)$
  \begin{itemize}
      \item \textbf{First Round:} the Prover, $P$, computes $A'(x,w; r^P) \rightarrow a$ accordingly:
      \begin{itemize}
          \item Parse $r^P = (r^P_\alpha \| r)$
          \item Compute $a_\alpha \leftarrow A(x_\alpha, w_\alpha; r^P_\alpha)$
          \item Set $\mathbf{v} = (v_1,\ldots, v_l)$, where $v_\alpha = a_\alpha$ and $\forall i \in [l] \setminus \alpha,\ v_i = 0$.
          \item Compute $(\textsf{ck}, \textsf{ek}) \leftarrow \textsf{Gen}(\textsf{pp}, B = \{\alpha\})$.
          \item Compute $(\textsf{com}, \textsf{aux}) \leftarrow \textsf{EquivCom}(\textsf{pp}, \textsf{ek}, \mathbf{v}; r)$.
          \item Send $a = (\textsf{ck, com})$ to the verifier. 
      \end{itemize}
      \item \textbf{Second Round:} $V$ sends $c \leftarrow \{0,1\}^\lambda$ to $P$. 
      \item \textbf{Third Round:} Prover computes $Z'(x,w,c;r^P) \rightarrow z$:
        \begin{itemize}
          \item Parse $r^P = (r^P_\alpha \| r)$
          \item Compute $z^* \leftarrow Z(x_\alpha, w_\alpha, c; r^P_\alpha)$
          \item For $i \in [l] \setminus \alpha$, compute $a_i \leftarrow \mathcal S^{\text{\tiny EHVZK}}(x_i, c, z^*)$. 
          \item Set $\mathbf{v}' = (a_1, \ldots, a_l)$
          \item Compute $r' \leftarrow \textsf{Equiv}(\textsf{pp, ek,} \mathbf{v}, \mathbf{v}'; \textsf{aux})$ (where $\textsf{aux}$ can be 
          regenerated with $\textsf{r}$).
          \item Send $z = (\textsf{ck}, z^*, r')$ to the verifier.
        \end{itemize}
      \item \textbf{Verification:} Verifier computes $\phi'(x,a,c_0,z) \rightarrow b$ as follows:
      \begin{itemize}
        \item Parse $a = (\textsf{ck, com})$ and $z = (\textsf{ck}, z^*, r')$.
        \item Set $a_i \leftarrow \mathcal S^{\text{\tiny EHVZK}}(x_i, c, z^*)$
        \item Set $\mathbf{v}' = (a_1, \ldots, a_l)$
        \item Compute and return: 
      \end{itemize}
      \[
          b = (\textsf{ck} \verify \textsf{ck}') \land \left(\textsf{com} \verify \textsf{BindCom}(\textsf{pp, ck}, \mathbf{v}'; r')\right) \land 
          \left(\bigwedge_{i \in [l]} \phi(x_i, a_i, c,z^*)\right)
        \]
  \end{itemize}
\end{protocol}

\paragraph{Communication Complexity.} By using the Q-Binding construction (Figure 4 of \cite{StackingSigmas}) together with Half-Bindings 
(Figure \ref{fig:half-binding}) we obtain a $1$-out-of-$2^q$ 
binding vector commitment scheme with communication complexity $q \cdot CC(\textsf{HalfBinding})$ 
(\textbf{Theorem 2 \& Corollary 1} in \cite{StackingSigmas}). According to the proofs in Theorem 5 \cite{StackingSigmas}, the communication complexity of 
$\Pi'$ is 
$CC(\Pi_l) = CC(\Pi) + |\textsf{ck}| + |\textsf{com}| + |\textsf{r'}|$, where the last three components are directly related to the choice of the 
partially-binding vector commitment. 

Since we are using a Q-Binding construction, we get the following communication complexity: 
$CC(\Pi_l) = CC(\Pi) + q(|\textsf{ck}_{1/2}| + |\textsf{com}_{1/2}| + |\textsf{r'}_{1/2}|)$. From Figure \ref{fig:half-binding}, we can see that 
$|\textsf{ck}_{1/2}| = |g|$, $|\textsf{com}_{1/2} = 2|g|$, and $|r'_{1/2}| = 2|z|$, where $g \in \G$ and $z \in \Z_{|\G|}$. In this case, $|g|$ 
depends on the number of bytes required to represent the group elements in $\G$, and $|z|$ depends on the size of the group $|\G|$. Therefore, 
$CC(\Pi_l) = CC(\Pi) + q(3|g| + 2|z|) = CC(\Pi) + \log_2(l)(3|g| + 2|z|)$. Hence, the communication complexity grows logarithmically with the 
number of clauses $l$.