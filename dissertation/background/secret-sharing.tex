\subsection{Secret Sharing Scheme}
A \textit{secret sharing scheme}, is a method of distributing a secret $s$ to $n$ participants in a way that no one participant has intelligible information about the secret. This is achieved by splitting up $s$ into \textit{shares}, distributing one share to each participant in a way that \textit{only} a subset of participants can reconstruct $s$. Subsets that can reconstruct the secret are called \textit{qualified sets}. The set of all qualified sets is the secret sharing scheme's \textit{access structure}.

In this work, we are concerned with \textit{perfect} secret sharing schemes. Perfect secret sharing schemes are ones where the participants in \textit{non-qualified} sets cannot obtain any information whatsoever about the secret. Additionally, for CDS94, we require our secret sharing scheme to have a few additional properties. 

\begin{definition}[Secret Sharing Schemes for CDS94]\label{def:secret-sharing}
Let $\Pi$ be a $\Sigma$-protocol for the relationship $\mathcal R = \{(x,w)\}$. We define a \textit{secret sharing scheme} for CDS94 as $S(k)$, where $k$ is the length of $x$ in bits. $S(k)$ splits a secret $s$ into $n$ shares such that $n$ is polynomial in $k$ ($n = poly(k)$). Let $D(s)$ refer to the probability distribution of all the shares that are produced when the secret $s$ is distributed. If we consider a subset of participants $A$, then $D_A(s)$ refers to the distribution of shares that only includes participants in $A$. Since the scheme is perfect, the probability distribution $D_A(s)$ is not affected by any other subset $B$ for any non-qualified set $A$. Therefore, we can simply write $D_A$ instead of $D_A(s)$ when $A$ is non-qualified. We now define the properties we require:
    \begin{enumerate}
        \item The length of shares produced in $S(k)$ is related to $k$ through a polynomial function.
        \item The secret can be distributed and reconstructed in a time that is polynomial in $k$.
        \item With a complete set of $n$ shares and the secret $s$, it is possible to check in a time that is polynomial in $k$ whether all qualified sets of shares determine $s$ as the secret.
        \item It is always possible to complete a set of shares for participants in a non-qualified set $A$, distributed according to $D_A$, to a full set of shares that are distributed according to $D(s)$ and consistent with the secret $s$. This completion process can be done in a time that is polynomial in $k$.
        \item The probability distribution $D_A$ for any non-qualified set $A$ is such that shares for the participants in $A$ are independently and uniformly chosen.
    \end{enumerate}
\end{definition}

An $S(k)$ which fulfils properties 1-4, is known as \textit{semi-smooth}. With a semi-smooth scheme, we can compile a SHVZK $\Sigma$-protocol using Theorem 9 of CDS94. A \textit{smooth} scheme, is where $S(k)$ fulfils all 5 properties. With this we can compile a HVZK $\Sigma$-protocol using Theorem 8 of CDS94.

\subsubsection{Shamir's Secret Sharing}\label{sec:sss}
In our implementation of the CDS94 compiler, we choose to make use of Shamir's secret sharing scheme \cite{DBLP:journals/cacm/Shamir79}. 

\begin{definition}[Shamir's Secret Sharing (SSS) Scheme] \label{def:sss}
 A smooth secret sharing scheme and a \textit{threshold sharing scheme}, SSS produces qualified sets of size $d$. Any $d$ out of $n$ participants can reconstruct the secret; with $d-1$ shares and less, no information about the secret can be obtained. 

 To distribute a secret:
\begin{enumerate}
    \item We first choose a random polynomial of degree $d - 1$. The constant term of the polynomial is the secret $s$ itself.
    \item We then associate each participant with an $x \in \Z_q$ where $q$ is a large prime. For each participant, we calculate $y_i$ by evaluating the polynomial at $x_i$ for the $i$-th participant. The $i$-th share (which corresponds to the $i$-th participant) is then computed by concatenating $x_i$ with $y_i$ ($share(i) = x_i\circ y_i$).
\end{enumerate}

To reconstruct a secret:
\begin{enumerate}
    \item Any $d$ or more participants can combine their shares. Using Lagrange interpolation, a polynomial of degree $d - 1$ can be computed which passes through the $d$ points that correspond to the shares. This means that the constant term can be derived and the secret is reconstructed. 
    \item If less than $d$ participants combine their shares, they will not have enough information to reconstruct the secret. This is because the polynomial that is derived from these shares is completely random. 
\end{enumerate}
\end{definition}

\begin{definition}[Qualified Set Completion for SSS]\label{def:sss-completion}
Given the secret $s$, an unqualified set of shares $U$, and an array of indexes for the active clauses $A$, we define the algorithm $Complete(s, U, A) \rightarrow Q$, where $Q$ is a set of shares with $x$ coordinates/values corresponding to the indexes in $A$. 

\begin{enumerate}
    \item Firstly, ensure that threshold is set to $N - d + 1$, where $d$ is the number of active clauses; in other words the threshold should be $k + 1$ where $k$ is the number of inactive clauses and $|U| = k$. 
    \item Construct a lagrange polynomial of degree $k$, with the secret $s$ and shares in the set $U$. 
    \item With this polynomial, interpolate at $x = i$ for $i \in A$ to determine the $y_i$ for each $i \in A$.
    \item Return $Q \leftarrow \{share(i) =  x_i\circ y_i\ |\ i \in A\}$.
    % \item Take these as $share(c_i)$, taking the first relevant number of bits if somehow the challenge is smaller than the shares, and extrapolating with random bits if the challenge is meant to be larger.
\end{enumerate}
    
\end{definition}

Notably, we do not use SSS within CDS94 to reconstruct a secret; we use it together with a known secret to complete a qualified set using the procedure outlined in Definition \ref{def:sss-completion}.
