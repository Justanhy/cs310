\subsection{Shamir's Secret Sharing Scheme}
A \textit{secret sharing scheme} is a method of distributing a secret $s$ to $n$ participants in a way that no one participant has intelligible information about the secret. This is achieved by splitting up $s$ into \textit{shares}, distributing one share to each participant in a way that a subset of participants can reconstruct $s$. Subsets that can reconstruct the secret are called \textit{qualified sets}. For \textit{perfect} secret sharing schemes, like Shamir's, participants in the complement \textit{non-qualified} sets cannot obtain any information about the secret.

Shamir's secret sharing scheme \cite{DBLP:journals/cacm/Shamir79} is also what is known as a \textit{threshold sharing scheme}. These are schemes that produce  qualified sets of size $d$. Any $d$ out of $n$ participants can reconstruct the secret; with $d-1$ shares and less, no  information about the secret can be obtained. 

\textbf{Share Reconstruction for Shamir's}
Explain in more detail in the future, but essentially:
\begin{enumerate}
    \item Given secret $s$, unqualified set of shares $U$, and array of indexes for active clauses $A$
    \item Firstly, ensure that threshold is set to $N - d + 1$, where $d$ is the number of active clauses; in other words the threshold should be $k + 1$ where $k$ is the number of inactive clauses and $|U| = k$. 
    \item Construct a lagrange polynomial of degree $k$, with the secret $s$ and set $U$. 
    \item With this polynomial, interpolate at $x = i$ for $i \in A$ to determine the share for each $i \in A$. 
    \item Take these as $share(c_i)$, taking the first relevant number of bits if somehow the challenge is smaller than the shares, and extrapolating with random bits if the challenge is meant to be larger.
\end{enumerate}