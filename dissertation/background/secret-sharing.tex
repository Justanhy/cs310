\subsection{Secret Sharing Scheme}
A \textit{secret sharing scheme} \cite{DBLP:journals/cacm/Shamir79, DBLP:conf/mark2/Blakley79}, is a technique for dividing a secret $s$ into $n$ \textit{shares}, and distributing these shares to $n$ participants. These shares are made such that none of the $n$ participants can obtain any useful information about the secret $s$ unless a certain subset of participants combine their parts -- a subset like this is known as a \textit{qualified set}. The collection of all qualified sets is known as the \textit{access structure} of the secret sharing scheme. 

In this work, we are concerned with \textit{perfect} secret sharing schemes. 
Perfect secret sharing schemes are ones where the participants in \textit{non-qualified} sets cannot obtain any information 
whatsoever about the secret \cite{CDS94}. Additionally, for CDS94, we require our secret sharing scheme to have a few 
additional properties. 

\begin{definition}[Secret Sharing Schemes for \cite{CDS94}]\label{def:secret-sharing}
Let $\Pi$ be a $\Sigma$-protocol for the relationship $\mathcal R = \{(x,w)\}$. We define a \textit{secret sharing scheme} for CDS94 as $S(k)$, where $k$ is the length of $x$ in bits. $S(k)$ splits a secret $s$ into $n$ shares such that $n$ is polynomial in $k$ ($n = poly(k)$). We define $D(s)$ as the probability distribution of the shares that are produced when the secret $s$ is distributed. If we consider a subset of participants $A$, then $D_A(s)$ refers to the distribution of shares that only includes participants in $A$. Since the scheme is perfect, the probability distribution $D_A(s)$ is not affected by any other subset $B$ for any \textit{non-qualified} set $A$. Therefore, we can simply write $D_A$ instead of $D_A(s)$ when $A$ is non-qualified. We now define the properties we require:
    \begin{enumerate}
        \item The length of shares produced in $S(k)$ is related to $k$ through a polynomial function.
        \item The secret can be distributed and reconstructed in a time that is polynomial in $k$.
        \item With a complete set of $n$ shares and the secret $s$, it is possible to check in a time that is polynomial in $k$ whether all qualified sets of shares determine $s$ as the secret.
        \item For any secret $s$, it is always possible to complete a set of shares for participants in a non-qualified set $A$, distributed according to $D_A$, to a full set of shares that are distributed according to $D(s)$ and consistent with the secret $s$. This completion process can be done in a time that is polynomial in $k$.
        \item The probability distribution $D_A$ for any non-qualified set $A$ is such that shares for the participants in $A$ are independently and uniformly chosen.
    \end{enumerate}
\end{definition}

An $S(k)$ which fulfils properties 1-4, is known as \textit{semi-smooth}. With a semi-smooth scheme, 
we can compile a SHVZK $\Sigma$-protocol using Theorem 9 of \cite{CDS94}. A \textit{smooth} scheme, is where $S(k)$ 
fulfils all 5 properties. With this we can compile a HVZK $\Sigma$-protocol using Theorem 8 of \cite{CDS94}.

\subsection{Shamir's Secret Sharing}\label{sec:sss}
In our implementation of the CDS94 compiler, we choose Shamir's secret sharing scheme 
\cite{DBLP:journals/cacm/Shamir79}. First, we need to define Lagrange's interpolating polynomial 
as it is relevant in the definition of Shamir's secret sharing scheme.

\begin{definition}[Lagrange's Interpolating Polynomial \cite{lagrange}]\label{def:lagrange}
    Lagrange's interpolating polynomial is a polynomial that intersects $n$ data points $(x_0, y_0), (x_1, y_1), \ldots, (x_{n-1}, y_{n-1})$. It can be written in the 
    form:
    
    $$
    L(x) = \sum_{i=0}^{n-1} y_i \ell_i(x),
    $$
    
    where $\ell_i(x)$ is the $i$th Lagrange basis polynomial, defined by:
    
    $$
    \ell_i(x) = \prod_{\substack{j=0 \\ j\neq i}}^{n-1} \frac{x - x_j}{x_i - x_j}.
    $$
    
    The Lagrange interpolating polynomial $L(x)$ is a polynomial of degree at most 
    $n-1$ that satisfies $L(x_i) = y_i$ for $i = 0, 1, \ldots, n-1$. In Shamir's secret sharing 
    scheme, we rely on polynomial interpolation to distribute and 
    reconstruct the secret from a set of shares. 
\end{definition}

\begin{definition}[Shamir's Secret Sharing (SSS) Scheme] \label{def:sss}
 A smooth secret sharing scheme and a \textit{threshold sharing scheme}, Shamir's secret sharing scheme distributes $n$ shares such that any $d < n$ shares are able to reconstruct the secret. No intelligible information about the secret is attained with $d-1$ shares and less.

 To distribute a secret $s$:
\begin{enumerate}
    \item We first choose a random polynomial of degree $d - 1$ in which the constant term is $s$. 
    $$
    q(x) = s + a_1x + \ldots + a_{d-1}x^{d-1}
    $$
    \item Next, for participants $i \in \{1, \ldots, n\}$, we distribute the share $share(i) = (i, q(i))$ 
    to the $i$-th participant. We can easily represent $share(i)$ in bits by simply concatenating 
    the bits of $i$ and $q(i)$.
    % We do this by choosing $d$ points 
    % in the 2-dimensional plane $(x_0, y_0), \ldots, (x_{d-1}, y_{d-1})$: 
    % Let $x_0 = 0$ and $y_0 = s$. For $i \in \{1, \ldots, d-1\}$, we set $x_i = i$ and 
    % $y_i \samplefrom \Z_q$ where $q$ is a large prime. These $d$ points now represent our 
    % degree $d - 1$ polynomial. 
\end{enumerate}

To reconstruct a secret:
\begin{enumerate}
    \item Any $d$ or more participants can combine their shares. Using interpolation, a
    polynomial $h$ of degree $d - 1$ can be computed which passes through the $d$ points that 
    correspond to the shares. This polynomial $h$ is equivalent to that which was used to 
    distribute the secret, polynomial $q$. This means that the constant term can be derived by 
    evaluating $h(0) = q(0) = s$.
    \item If less than $d$ participants combine their shares, they will not have enough information 
    to reconstruct the secret. This is because if a random polynomial $h$ is interpolated from 
    $d$ points with non-zero $x$-values, then the value of $h(0) = s$ is uniformly random. This means
    that with only $d-1$ points (or shares), the values that $s$ can take are uniformly distributed
    in the domain of the $y$-axis. 
\end{enumerate}
\end{definition}

\begin{definition}[Qualified Set Completion for SSS]\label{def:sss-completion}
Suppose we have a set $\mathcal S$ of shares, which have been previously distributed by 
Shamir's secret sharing scheme with the secret $s$ using polynomial $q$ with degree $d$. 
We define the algorithm $\textsf{Complete}(s, U, A) \rightarrow Q$, where
\begin{itemize}
    \item $A = \{i \st (i, q(i)) \in \mathcal S\} \land |A| = |\mathcal S| - d$ is the set of indexes for 
    a subset of shares in $\mathcal S$.
    \item $U = \{ share(i) \st i \in \mathcal S \setminus A\} \land |U| = d$ is a 
    an unqualified set of shares.  
\end{itemize}

$\textsf{Complete}$ returns a qualified set of shares $Q = \{share(i) \st i \in A\}$, which is 
the corresponding shares of the indexes in $A$. We briefly outline how this is done:

\begin{enumerate}
    \item Using polynomial interpolation, construct a polynomial of degree $d$, 
    with the shares in the set $U$ and $share(0) = (0, s)$ (this gives us $d + 1$ shares). 
    \item With this polynomial, interpolate at $x = i$ for $i \in A$ to determine the 
    corresponding $y_i$ for each $i \in A$.
    \item Return $Q \leftarrow \{share(i) =  (x_i, y_i)\ |\ i \in A\}$.
    % \item Take these as $share(c_i)$, taking the first relevant number of bits if somehow the challenge is smaller than the shares, and extrapolating with random bits if the challenge is meant to be larger.
\end{enumerate}
    
\end{definition}

Notably, we do not use SSS within CDS94 to reconstruct a secret; we use it together with a known secret to complete a qualified set using the procedure outlined in Definition \ref{def:sss-completion}.

