\section{Disjunctive Zero-Knowledge}

\begin{definition}[NP Relations]
Let $R \subseteq \{0,1\}^* \times \{0,1\}^*$ be a binary relation. Then $w(x) = \{w \mid (x,w) \in R\}$ and $L_R = \{x \mid \exists w, (x,w) \in R\}$. If $(x,w) \in R$, we say that $w$ is a witness for $x$. $R$ is an NP-relation if it fulfils the following two properties:
\begin{enumerate}
    \item \textbf{Polynomially bounded.} We say that $R$ is \textit{polynomially bounded} if there exists a polynomial $p$ such that $|w| \le p(|x|), \forall (x,w) \in R$. 
    \item \textbf{Polynomial-time verification.} There exists a polynomial-time algorithm for deciding membership in $R$. Consequently, $L_R \in NP$. 
\end{enumerate}

\textit{Throughout this document, we will use $\mathcal R$ to refer to a binary NP-relation.}
\end{definition}

\begin{definition}[Zero-Knowledge \cite{PAZK}]\label{def:zeroknowledge}
A proof or argument system $(P,V)$ is zero-knowledge over $\mathcal R$ if there exists a \textit{probabilistic polynomial time} (PPT) simulator $\mathcal S$, such that for all $(x,w) \in R$, the distribution of the output $\mathcal S(1^\lambda, x)$ of the simulator is indistinguishable from the distribution over the conversations generated by the interaction of $P$ and $V$, from the perspective of $V$; we denote this with $\ViewPV$. Conversations between $P$ and $V$ are ordered triples of the form $(a,c,z)$, and are known as \textit{transcripts}.
\end{definition}

Intuitively, this means that $V$ should not learn anything from the transcripts  with $P$ that they cannot already learn on their own by running the simulator $\mathcal S$; they learn nothing new.

\begin{definition}[Disjunctive Zero-Knowledge]
Given a sequence of statements $(x_1,x_2,\ldots, x_l)$, a \textit{disjunctive zero-knowledge proof} is a protocol to 
prove in zero-knowledge that $x_1 \in \mathcal L_1 \lor x_2 \in \mathcal L_2 \lor \ldots \lor x_l \in \mathcal L_l$, for 
NP languages $\mathcal L_i$. We term clauses for which the prover has a witness for as \textit{active} clauses. 
\end{definition}

\begin{definition}[Honest-Verifier Zero-Knowledge -- HVZK]\label{def:hvzk}
A proof system is HVZK if it only requires that $\mathcal S$ is an efficient simulator 
for honest (non-malicious) probabilistic polynomial time verifier strategies $V$. If $V$ is malicious then the distribution 
of the output $\mathcal S(x)$ will no longer be indistinguishable from $\ViewPV$ for such proof systems. 
\end{definition}
