\section{Disjunctive Zero-Knowledge}

\begin{definition}[NP Relations]
Suppose we have a binary relation $R$, where $R \subseteq \{0,1\}^* \times \{0,1\}^*$. 
In this context, we define the witness $w(x)$ as $\{w \mid (x,w) \in R\}$, where $w$ is the witness for statement $x$. We also define the language $L_R = \{x \mid \exists w, (x,w) \in R\}$. We say that $R$ is an NP-relation if it satisfies the following conditions:
\begin{enumerate}
    \item \textbf{Polynomially bounded.} The size of the witness $w$ is bounded by a polynomial $p$ that is a function of the length of $x$. In other words, $R$ is considered to be polynomially bounded if $|w| \le p(|x|)$ for all $(x,w) \in R$.
    \item \textbf{Polynomial-time verification.} There is a polynomial-time algorithm available for determining whether the pair ($x$ and $w$) is a member of $R$. This means that $L_R \in NP$.
\end{enumerate}
\textit{In this document, we will use the symbol $\mathcal R$ to represent a binary NP-relation.}

\end{definition}

\begin{definition}[Zero-Knowledge \cite{PAZK}]\label{def:zeroknowledge}
A proof system is considered zero-knowledge over the relationship $\mathcal R$ if there is a \textit{probabilistic polynomial time} (PPT) algorithm known as the simulator $\mathcal S$.  
We define this proof system with $(P, V)$. On input $x$, the simulator produces a simulated transcript of the conversation between the prover $P$ and verifier $V$. This transcript should be indistinguishable from the actual conversations between the prover and verifier, $\ViewPV$, for all valid inputs $(x,w) \in R$. Transcripts are ordered triple of the form $(a,c,z)$.

Essentially, this means that the verifier should not gain any new information from the transcripts that they could not already obtain by running the simulator on their own, thus learning nothing new.
\end{definition}


\begin{definition}[Disjunctive Zero-Knowledge]
Previously defined in our progress report for this project \cite{prog-report}. Given a sequence of statements $(x_1,x_2,\ldots, x_l)$, a \textit{disjunctive zero-knowledge proof} is a protocol to 
prove in zero-knowledge that $x_1 \in \mathcal L_1 \lor x_2 \in \mathcal L_2 \lor \ldots \lor x_l \in \mathcal L_l$, for 
NP languages $\mathcal L_i$.
\end{definition}

\begin{definition}[Honest-Verifier Zero-Knowledge -- HVZK]\label{def:hvzk}
A proof system is considered HVZK if it only needs an efficient simulator $\mathcal S$ for honest probabilistic polynomial time verifier strategies $V$ to achieve indistinguishability between the distribution of the output $\mathcal S(x)$ and the actual conversations between the prover and the verifier $\ViewPV$. However, if the verifier is malicious, then the output distribution of $\mathcal S(x)$ will no longer be indistinguishable from $\ViewPV$ for such proof systems.
% A proof system is HVZK if it only requires that $\mathcal S$ is an efficient simulator 
% for honest (non-malicious) probabilistic polynomial time verifier strategies $V$. If $V$ is malicious then the distribution 
% of the output $\mathcal S(x)$ will no longer be indistinguishable from $\ViewPV$ for such proof systems. 
\end{definition}
