% !TeX root = ../dissertation.tex
\subsection{Partially-Binding Vector Commitments}
In Section 5 of \cite{StackingSigmas}, Goel {\em{et al}} introduce the concept of partially-binding vector commitment schemes. 
These commitment schemes allow a Prover to make a commitment to a vector of length $l$ with $t$ binding positions; 
the remaining positions (indexes) in the vector are equivocable. 
Here, we recall the definition of these commitment schemes and refer interested readers to Figure 2 of \cite{StackingSigmas} for the construction of the general $t$-out-of-$l$ 
scheme. 

\begin{definition}[t-out-of-l Binding Vector Commitment \cite{StackingSigmas}]\label{def:comm_scheme}
  Given a message space $\mathcal M$, the security paramater $\lambda$, the length of the vector $l$, and the number of binding positions $t$: 
  the tuple of PPT algorithms $(\mathsf{Setup}, \mathsf{Gen}, \mathsf{EquivCom}, \mathsf{Equiv}, \mathsf{BindCom})$ defines a $t$-out-of-$l$ 
  partially-binding vector commitment scheme. These algorithms are defined as follows:
  \begin{itemize}
  \item $\mathsf{pp} \leftarrow \mathsf{Setup}(1^\lambda)$: Given the security parameter $\lambda$,
  the $\mathsf{Setup}$ algorithm outputs public parameters $\mathsf{pp}$ for the commitment scheme.
  \item $(\mathsf{ck},\mathsf{ek}) \leftarrow \mathsf{Gen}(\mathsf{pp},B)$: Given public parameters $\mathsf{pp}$ 
  and a set $B$, where $B \subseteq [l] \land |B| = t$. The $\mathsf{Gen}$ algorithm returns a commitment key $\mathsf{ck}$ and equivocation key $\mathsf{ek}$.

  \item $(\mathsf{com},\mathsf{aux}) \leftarrow \mathsf{EquivCom}(\mathsf{pp},\mathsf{ek},v;r)$: Given public parameter 
  $\mathsf{pp}$, equivocation key $\mathsf{ek}$, vector $v$ with length $l$, and randomness $r$ as input, the $\mathsf{EquivCom}$ algorithm returns a 
  partially-binding commitment $\mathsf{com}$ and auxiliary equivocation information $\mathsf{aux}$. This means that the vector $v$ can be equivocated in 
  equivocable locations and $\mathsf{com}$ will be the same.  

  \item $r \leftarrow \mathsf{Equiv}(\mathsf{pp},\mathsf{ek},v,v',\mathsf{aux})$: Given public parameters $\mathsf{pp}$, 
  equivocation key $\mathsf{ek}$, original message vector $v$ and updated message vector $v'$ 
  where $\forall i \in B: v_i = v'_i$, and auxiliary equivocation information $\mathsf{aux}$. $\mathsf{Equiv}$ returns equivocation 
  randomness $r$.

  \item $\mathsf{com} \leftarrow \mathsf{BindCom}(\mathsf{pp}, \mathsf{ck}, v; r)$: Given public parameters 
  $\mathsf{pp}$, commitment key $\mathsf{ck}$, vector $v$, and randomness $r$ as input, $\mathsf{BindCom}$ returns a commitment 
  $\mathsf{com}$. This algorithm plays a similar role to that of $\mathsf{Open}$ in a typical commitment scheme.
  \end{itemize}

  Partially-binding vector commitment schemes have the following properties
  
  \begin{enumerate}
    \item \textbf{(Perfect) Hiding.} Suppose there are two vectors $\mathbf{v}^{(1)}, \mathbf{v}^{(2)} \in \mathcal M^l$ and two corresponding sets of 
    binding positions $B^{(1)}, B^{(2)} \in {(l) \choose t}$. A perfectly hiding commitment scheme is one where the commitment key $ck$ and commitment $com$ 
    produced by any two sets of binding position and original vectors $(\mathbf{B}^{(1)}, \mathbf{v}^{(1)})$ and $(\mathbf{B}^{(2)}, \mathbf{v}^{(2)})$ are 
    indistinguishable from each other when they are equivocated to the same vector $\mathbf{v}'$ (provided that $\mathbf{v}'$ is a valid equivocation for 
    both vectors). 

    \item \textbf{(Computational) Partial Binding.} A malicious user (that generates $\mathsf{ck}$ itself) is not able to cheat the system by equivocating 
    on more than $l - t$ positions.

    \item \textbf{Partial Equivocation.} It is always possible to equivocate to any vector $\mathbf{v}'$ as long as $\forall i \in B: v_i = v'_i$. 
  \end{enumerate}

  There is an efficiency requirement which requires the size of the commitment to be independent from the size of the messages 
  in the vector. This can be achieved by using a collision resistant hash function $H$. 
  For more details on each property, we refer the reader to Definition 4 of \cite{StackingSigmas}. 
\end{definition}


\subsection{Half-Bindings \& Q-Bindings}\label{sec:qbinding}
In this work, we make use of the generic 1-out-of-$2^q$ construction provided in Section 5.3 of \cite{StackingSigmas} 
for our implementation of the Stacking Sigmas compiler. We use this 
particular construction as it allows us to yield commitments that are logarthmic in size (in bytes) to the original 
vector of messages (\textbf{Theorem 2 \& Corollary 1} in \cite{StackingSigmas}).
A core ingredient in this construction is a 1-out-of-2 partially-binding vector commitment scheme, which is used recursively 
to form a \textit{binary} "tree of commitments". 
For brevity, we refer to the generic construction as "q-bindings" and the 1-out-of-2 construction as "half-bindings".
The leaves of this tree are the original messages in our vector; intermediate nodes are the resulting commitments to the 
nodes' children using half-bindings. 
Essentially, intermediate commitments are regarded as messages as well, and are commited to recursively until there is only 
1 root node -- the final commitment. 

While the source material provides the general $t$-out-of-$l$ construction for partially-binding vector commitments, 
it does not explicitly provide that for half-bindings. 
In Figure \ref{fig:half-binding}, we provide the construction for half-bindings derived from the general $t$-out-of-$l$ 
construction. 
The proof of correctness for this construction is trivial as it can be easily seen to be a specific case of the general 
construction when $t = 1$ and $l = 2$. 

\begin{figure}[h]
  \caption{Construction of a 1-out-of-2 partially-binding vector commitment scheme.}
  \label{fig:half-binding}
\end{figure}
\begin{breakablefig}
    \begin{minipage}{0.45\linewidth}
      \vspace{-4em}
      \underline{$pp \leftarrow$ Setup$(1^\lambda)$:}
      \begin{enumerate}
        \item $\G \leftarrow $ GenGroup$(1^\lambda)$; $g_0, h \samplefrom \G$
        \item $\textbf{return } pp \leftarrow (\G, g_0, h)$
      \end{enumerate}  
    \end{minipage}
    \begin{minipage}{0.5\linewidth}
      \underline{$(com, aux) \leftarrow$ EquivCom$(pp, ek, m_1, m_2)$:}
      \begin{enumerate}
        \item Extract $ck$ from $ek$
        \item $r \samplefrom \Z_{|\G|}^2$
        \item com $\leftarrow$ BindCom$(pp,ck,m_1,m_2,r)$
        \item \textbf{return} $(\text{com}, r)$
      \end{enumerate}  
    \end{minipage}

    \underline{$(ck, ek) \leftarrow$ Gen$(pp, B)$:}
    \begin{enumerate}
      \item Let $E = \{1,2\} \setminus B$ be the set of equivocal indexes. $|B| = 1$.
      \item Generate trapdoor $td$ for $i \in E: td \samplefrom \Z_{|\G|}, g_i \leftarrow h \cdot td$
      \item Interpolate the first element: $g_1 = \begin{cases}
       g_2 - g_0 & \text{if $2 \in E$} \\
       g_1 & \text{if $1 \in E$}
      \end{cases}$
      \item $ck \leftarrow g_1$
      \item $ek \leftarrow (B, td, ck)$
      \item \textbf{return} $(ck, ek)$
    \end{enumerate}

    \underline{$com \leftarrow$ BindCom$(pp,ck, m_1,m_2, r)$:}
    \begin{enumerate}
      \item Interpolate $g_2$: $g_2 = g_1 + g_0$
      \item $(r_1, r_2) \leftarrow r$
      \item Commit individually $\textbf{for } j \in \{1,2\}: com_j \leftarrow h\cdot r_j + g_j \cdot m_j \in \G$
      \item \textbf{return} $(com_1,com_2)$
    \end{enumerate}

    \underline{$r \leftarrow$ Equiv$(pp, ek, m_1, m_2, m_1', m_2', aux)$:}
    \begin{enumerate}
      \item Extract $B$ and $td$ from $ek$; Let $E = \{1,2\} \setminus B$.
      \item Parse $aux = (r_1, r_2) \in \Z_{|\G|}^2$
      \item Interpolate $g_2 = g_1 + g_0$
      \item for $i \in B: r'_i \leftarrow r_i$
      \item for $i \in E: r'_i \leftarrow r_i - td \cdot (m_i' - m_i)$
      \item \textbf{return} $r' \leftarrow (r_1', r_2')$
    \end{enumerate}
\end{breakablefig}

  \paragraph*{Communication Complexity.} Observing the construction of half-bindings, we can see that the size (in bytes) of the outputs of each method 
  is directly related to the choice of the group $\G$ and the number of bytes required to represent a group element $g \in \G$ and scalars $r \in \Z_{|\G|}$. 
  
  % For ease of reference, we also include the q-binding scheme (1-out-of-$2^q$ parital-binding vector commitment
  % scheme) in Figure \ref{fig:q-binding}.
  
  % \begin{figure}[h]
  %   \caption{Construction of a 1-out-of-$2^q$ partially-binding vector commitment scheme.}
  %   \label{fig:q-binding}
  % \end{figure}
  % \begin{breakablefig}
  %     \begin{minipage}{0.45\linewidth}
  %       \vspace{-2em}
  %       \underline{$pp \leftarrow$ Setup$(1^\lambda)$:}
  %       \begin{enumerate}
  %         \item $pp_A \leftarrow $ PBComm$_A$.Setup$(1^\lambda)$
  %         \item $pp_B \leftarrow $ PBComm$_B$.Setup$(1^\lambda)$
  %         \item \textbf{return} $(pp_A, pp_B)$
  %       \end{enumerate}  
  %     \end{minipage}
  %     \begin{minipage}{0.5\linewidth}
  %       \underline{$(ck, ek) \leftarrow $ Gen$(pp = (pp_A, pp_B), B = \{i\})$:}
  %       \begin{enumerate}
  %         \item Compute $i_A = i \bmod \ell_A, i_B = \lfloor i / \ell_A \rfloor$
  %       \end{enumerate}  
  %     \end{minipage}
  
  %     \underline{$(ck, ek) \leftarrow$ Gen$(pp, B)$:}
  %     \begin{enumerate}
  %       \item $i_A = i \bmod \ell_A$, $i_B = \lfloor i/\ell_A \rfloor$ {\footnotesize // Compute indices}
  %       \item $\mathbf{v}^{(A)} \leftarrow \mathbf{0}$, $v(A) \leftarrow v_i \cdot \delta_{i_A}$ {\footnotesize // Form a length 
  %       $l_A$ vector $v_A$, which is zero everywhere except at $i_A$ where it is $v_i$;}
  %       \item $(com_A, aux_A) \leftarrow \text{PBCommA.EquivCom}(pp_A, ek_A, v(A))$;
  %       \item Form a length $l_B$ vector $v_B$, which is zero everywhere except at $i_B$ where it is $com_A$;
  %       \item $v(B) \leftarrow 0$, $v(B) \leftarrow com_A \cdot \delta_{i_B}$;
  %       \item $(com_B, aux_B) \leftarrow \text{PBCommB.EquivCom}(pp_B, ek_B, v(B))$ \texttt{Return the root/outer commitment}
  %       \item \textbf{return} $(com_B, (aux_A, aux_B))$;
  %     \end{enumerate}
  
  %     \underline{$com \leftarrow$ BindCom$(pp,ck, m_1,m_2, r)$:}
  %     \begin{enumerate}
  %       \item Interpolate $g_2$: $g_2 = g_1 + g_0$
  %       \item $(r_1, r_2) \leftarrow r$
  %       \item Commit individually $\textbf{for } j \in \{1,2\}: com_j \leftarrow h\cdot r_j + g_j \cdot m_j \in \G$
  %       \item \textbf{return} $(com_1,com_2)$
  %     \end{enumerate}
  
  %     \underline{$r \leftarrow$ Equiv$(pp, ek, m_1, m_2, m_1', m_2', aux)$:}
  %     \begin{enumerate}
  %       \item Extract $B$ and $td$ from $ek$; Let $E = \{1,2\} \setminus B$.
  %       \item Parse $aux = (r_1, r_2) \in \Z_{|\G|}^2$
  %       \item Interpolate $g_2 = g_1 + g_0$
  %       \item for $i \in B: r'_i \leftarrow r_i$
  %       \item for $i \in E: r'_i \leftarrow r_i - td \cdot (m_i' - m_i)$
  %       \item \textbf{return} $r' \leftarrow (r_1', r_2')$
  %     \end{enumerate}
  % \end{breakablefig}

