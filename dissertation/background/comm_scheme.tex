% !TeX root = ../dissertation.tex
\subsubsection{Partially-Binding Vector Commitments}
In Section 5 of \cite{StackingSigmas}, Goel {\em{et al}} introduce the concept of partially-binding vector commitment schemes. 
These commitment schemes allow a Prover to make a commitment to a vector of length $l$ with $t$ binding positions; 
the remaining positions (indexes) in the vector are equivocable. 
Here, we recall the definition of these commitment schemes and refer interested readers to Figure 2 of \cite{StackingSigmas} for the construction of the general $t$-out-of-$l$ 
scheme. 

\subsubsection{Half-Bindings \& Q-Bindings}
In this work, we make use of the generic 1-out-of-$2^q$ construction provided in Section 5.3 of \cite{StackingSigmas} for our implementation of the Stacking Sigmas compiler. We use this 
particular construction as it allows us to yield commitments that are logarthmic in size (in bytes) to the original vector of messages. 
A core ingredient in this construction is a 1-out-of-2 partially-binding vector commitment scheme, which is used recursively to form a \textit{binary} "tree of commitments". 
For brevity, we refer to the generic construction as "q-bindings" and the 1-out-of-2 construction as "half-bindings".
The leaves of this tree are the original messages in our vector; intermediate nodes are the resulting commitments to the nodes' children using half-bindings. 
Essentially, intermediate commitments are regarded as messages as well, and are commited to recursively until there is only 1 root node -- the final commitment. 

While the source material provides the general $t$-out-of-$l$ construction for partially-binding vector commitments, it does not explicitly provide that for half-bindings. 
In Figure \ref{fig:half-binding}, we provide the construction for half-bindings derived from the general $t$-out-of-$l$ construction. 
The proof of correctness for this construction is trivial as it can be easily seen to be a specific case of the general construction when $t = 1$ and $l = 2$. 

\begin{figure}
  \centering
  \begin{construction}[]
    \begin{minipage}{0.45\linewidth}
      \underline{$pp \leftarrow$ Setup$(1^\lambda)$:}
      \begin{enumerate}
        \item $\G \leftarrow $ GenGroup$(1^\lambda)$; $g_0, h \samplefrom \G$
        \item $\textbf{return } pp \leftarrow (\G, g_0, h)$
      \end{enumerate}  
    \end{minipage}
    \begin{minipage}{0.5\linewidth}
      \underline{$(com, aux) \leftarrow$ EquivCom$(pp, ek, m_1, m_2)$:}
      \begin{enumerate}
        \item Extract $ck$ from $ek$
        \item $r \samplefrom \Z_{|\G|}^2$
        \item com $\leftarrow$ BindCom$(pp,ck,m_1,m_2,r)$
        \item \textbf{return} $(\text{com}, r)$
      \end{enumerate}  
    \end{minipage}

    \underline{$(ck, ek) \leftarrow$ Gen$(pp, B)$:}
    \begin{enumerate}
      \item Let $E = \{1,2\} \setminus B$ be the set of equivocal indexes. $|B| = 1$.
      \item Generate trapdoor $td$ for $i \in E: td \samplefrom \Z_{|\G|}, g_i \leftarrow h \cdot r_i$
      \item Interpolate the first element: $g_1 \begin{cases}
       g_2 - g_0 & \text{if $2 \in E$} \\
       g_1 & \text{if $1 \in E$}
      \end{cases}$
      \item $ck \leftarrow g_1$
      \item $ek \leftarrow (B, td, ck)$
      \item \textbf{return} $(ck, ek)$
    \end{enumerate}

    \underline{$com \leftarrow$ BindCom$(pp,ck, m_1,m_2, r)$:}
    \begin{enumerate}
      \item Interpolate $g_2$: $g_2 = g_1 + g_0$
      \item $(r_1, r_2) \leftarrow r$
      \item Commit individually $\textbf{for} j \in \{1,2\}: com_j \leftarrow h\cdot r_j + g_j \cdot m_j \in \G$
      \item \textbf{return} $(com_1,com_2)$
    \end{enumerate}

    \underline{$r \leftarrow$ Equiv$(pp, ek, m_1, m_2, m_1', m_2', aux)$:}
    \begin{enumerate}
      \item Extract $B$ and $td$ from $ek$; Let $E = \{1,2\} \setminus B$.
      \item Parse $aux = (r_1, r_2) \in \Z_{|\G|}^2$
      \item Interpolate $g_2 = g_1 + g_0$
      \item for $i \in B: r'_i \leftarrow r_i$
      \item for $i \in E: r'_i \leftarrow r_i - td \cdot (m_i' - m_i)$
      \item \textbf{return} $r' \leftarrow (r_1', r_2')$
    \end{enumerate}
  \end{construction}
  \caption{Construction of a 1-out-of-2 partially-binding vector commitment scheme.}
  \label{fig:half-binding}
\end{figure}

