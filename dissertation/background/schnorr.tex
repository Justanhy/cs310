\section{Schnorr's Identification Protocol}\label{sec:schnorr}
An \textit{identification scheme} involves two participants, a prover $P$ and a verifier $V$, whose goal is for $P$ to prove their identity to $V$. Specifically, $V$ should be persuaded that $P$ has access to the private key that corresponds to $P$'s public key. An example of an identification scheme is the commonly used password authentication protocol.

Schnorr's protocol, as described in \cite{Schnorr}, is a method of proving identity in which the prover, denoted by $P$, demonstrates knowledge of the discrete log $w$ of a group element $H$ in a finite abelian group $(\G, +)$ with $+$ as the binary operator\footnote{We choose to define $\G$ with the $+$ as our implementation uses elliptic curves which are finite abelian groups over addition. The equivalent definition with a group defined over the multiplicative operator is $h = g^w$}. Specifically, $H = w \cdot G$ for some generator $G \in \G$. The protocol relies on the discrete log assumption \cite{Diffie1976NewDI}, which states that computing $w$ given $H$ and $G$ is currently computationally infeasible, assuming that $\G$ is large enough. However, proving that $H = w \cdot G$ given $w$ and $G$ can be done efficiently. 

Our implementation of the protocol will use the Ristretto group \cite{ristretto_web}, which is constructed from a family of elliptic curves known as Edwards curves \cite{Edwards2007} and is of prime order. 

\begin{definition}[Schnorr's Protocol \cite{Schnorr}]\label{def:schnorr}
    Let $\G$ be an elliptic curve over the finite field $\F_q$ and let $E(\F_q)$ denote the group of points on $\G$. Suppose that the prover $P$ and verifier $V$ agree on $\G$ and $\F_q$, and let $H \in E(\F_q)$ be the public key that corresponds to the private key $w$, where $H = w \cdot G$. The prover convinces the verifier that they possess knowledge of the private key $w$ by following Protocol \ref{prot:schnorr}.
\end{definition}

\begin{protocol}[label={prot:schnorr}]{Schnorr's protocol. Public information: $\G = E(\F_q),\ H \in \G$. Statement to prove: $H = w \cdot G$} 
    \vspace{-0.5cm}
    \begin{align*}
        &\text{Prover}(w_p)& 
        &&
        &\text{Verifier}& 
        \\
        &r \samplefrom \F_q,\ \mathcal A = r \cdot G&
        &\overset  {\mathcal A} {\xrightarrow{\hspace{3cm}}}&
        && 
        \\
        &&
        &&
        &c \samplefrom \F_q&
        \\
        &&
        &\overset c {\xleftarrow{\hspace{3cm}}}&
        &&
        \\
        &z \leftarrow cw_p + r&
        &&
        && 
        \\
        &&
        &\overset {z} {\xrightarrow{\hspace{3cm}}}&
        &z \cdot G \verify \mathcal A + c \cdot H&
    \end{align*}
    \tcblower
    Communication between Prover and Verifier in Schnorr's protocol. Solving for the final equation shows  that $V$ accepts if and only if $w = w_P$.
    $$
    \begin{gathered}
    z \cdot G = \mathcal A + c \cdot H \iff
    r \cdot G + c \cdot w_P \cdot G  = r \cdot G + c \cdot w \cdot G \iff
    w  =  w_P
    \end{gathered}
    $$
\end{protocol}



\paragraph{Simulator $\mathcal S$ for Schnorr's protocol.} From our definition of a Zero Knowledge protocol in Definition 
\ref{def:zeroknowledge}, we shared that every zero-knowledge proof system has a PPT algorithm called a simulator. We can construct this simulator by running the protocol "in reverse" shown in Figure \ref{fig:schnorr-sim}.

\begin{figure}[h]
    \centering
    \begin{problem}[width=\linewidth/2]{Let $\mathcal S(H)$ be our simulator}
        $z \samplefrom \F_q$ 
        
        $c \samplefrom \F_q$
        
        $\mathcal A = z \cdot G - c \cdot H$
        \tcblower
        \textbf{output} $(\mathcal A,c,z)$
    \end{problem}
    \caption{A Simulator for Schnorr}
    \label{fig:schnorr-sim}
\end{figure}

This is a valid simulator as the resulting $\mathcal A$ is random because $z$ and $c$ are chosen randomly. This means that the output of our simulator is effectively random and will have the same distribution over the transcript in a real interaction.
Note that the simulator presented here achieves only honest-verifier zero-knowledge (Definition
\ref{def:hvzk}) as the only input to the simulator is the statement to prove $H$. We can 
easily modify this simulator to achieve special honest-verifier zero-knowledge (Definition 
\ref{def:sigma}) by adding a challenge $c$ as an input to the simulator.

