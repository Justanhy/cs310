\subsection{CDS94 Compiler}
In this paper, Cramer {\em{et al}} \cite{CDS94} presents 2 primary ways to compile $\Sigma$-protocols depending 
on the underlying choice of $\Sigma$-protocol and the secret sharing scheme. Our implementation makes use of Theorem 
8 of the paper because we choose to use Schnorr's protocol (Section \ref{sec:schnorr}) and Shamir's secret sharing 
(Section \ref{sec:sss}). More details regarding our implementation will be discussed in a further section. Now, we recall 
theorem 8 of \cite{CDS94} -- note that we alter the notation slightly from the original paper to be more consistent 
with the rest of this report.

% In our implementation, we will use Schnorr's discrete log protocol over Ristretto25519 and Shamir's secret sharing scheme to demonstrate the compilation of $\Sigma$-protocol into a $\Sigma$-protocol for the disjunction of $n$ statements. 
% We will attempt to make the implementation as general as possible to open up for the future possibility to take any $\Sigma$-protocol that suits our requirements and transform it disjunctive zero-knowledge $\Sigma$-protocol.

% \subsection{The Witness Indistinguishable (WI) compilation}

% In their paper, Cramer {\em{et al}} \cite{CDS94} presents 2 primary ways to construct a WI protocol from a $\Sigma$-protocol $\mathcal P$ (Theorem 8 and 9). 

% \begin{itemize}
%     \item Theorem 8 requires a smooth secret sharing scheme, and a HVZK $\Sigma$-protocol, while
%     \item Theorem 9 requires special honest-verifier ZK (SHVZK) with at least a semi-smooth secret sharing scheme.
% \end{itemize}

% Since, SSS is a smooth threshold secret sharing scheme (required for 8), and Schnorr's protocol is SHVZK (required for 9), we can choose either construction. 
% \textit{We will use \textbf{Theorem 8} in this project.}

\textbf{Theorem 8 \cite{CDS94}}. Suppose we have a $\Sigma$-protocol $\Pi = (A, Z, \phi)$ for a relation $\mathcal R$ that fulfils the HVZK property, a smooth secret sharing scheme $S(k)$, and a relation $\mathcal R_\Gamma$ where $((x_1,\ldots,x_m),(w_1,\ldots,w_m)) \in \mathcal R_\Gamma$ if and only if $|x_i| = k$ for all $i$, and the set $Q = \{i \st (x_i,w_i) \in \mathcal R\}$  is a qualified set for $S(k)$.

With this, we can construct a $\Sigma$-protocol, $\Pi' = (A', Z', \phi')$, that is witness indistinguishable (Definition \ref{def:wi}) for the relation $\mathcal R_\Gamma$. We provide the description of this $\Sigma$-protocol in Protocol \ref{prot:cds-compiler}. Interested readers can refer to the original paper for the proof of this theorem \cite{CDS94}. 

\begin{protocol}[label={prot:cds-compiler}]{CDS94 Compiler: takes $n$ instances of 
    $\Sigma$-protocol $\Pi$ and provides a single $\Sigma$-protocol $\Pi'$ that proves the \textbf{disjunction} of these $n$ instances.}
    Let $A$ be the set of indices $i$ of the \textit{active clauses}. \\
    \textbf{Statement:} $x = x_1,\ldots, x_n$ \\
    \textbf{Witness:} $w = \{w_i\st i \in A\ \land (x_i, w_i) \in R\}$
    \begin{itemize}
        \item \textbf{First Round:} the Prover, $P$, computes $A'(x,w; r^P) \rightarrow a$ accordingly:
        \begin{itemize}
            \item For each $i \in \bar A$, run the simulator for $\mathcal P$ for the statement $x_i$ to produce the transcripts $(m_1^i, c_i, m_2^i)$.
            \item For each $i \in A$, compute $m_1^i \leftarrow A(x_i, w_i; r^P)$.
            \item Send $a \leftarrow (m_1^1, \ldots, m_1^n)$ to $V$.
        \end{itemize}
        \item \textbf{Second Round:} $V$ sends $c_0 \leftarrow \{0,1\}^\lambda$ to $P$. 
        \item \textbf{Third Round:} Prover computes $Z'(x,w,c;r^P) \rightarrow z$:
        \begin{itemize}
            \item Compute $Q \leftarrow Complete(c_0, U, A)$, where $U = \{c_i\st i \in \bar A\}$.
            \item Set $c_i = share(i)$, for each $share(i) \in Q$
            \item For $i \in A$, compute $m_2^i \leftarrow Z(x_i, w_i, c_i; r^P)$. 
            \item For $i \in \bar A$, $m_2^i$ has already been computed in the first round using the simulator.
            \item Send $z \leftarrow ((c_1, m_2^1), \ldots, (c_n,m_2^n))$ to $V$.
        \end{itemize}
        \item \textbf{Verification:} Verifier computes $\phi'(x,a,c_0,z) \rightarrow b$ as follows:
        \begin{itemize}
            \item Extract $c_i$ and $m_2^i$ from $z \leftarrow ((c_1, m_2^1), \ldots, (c_n,m_2^n))$
            \item Check that all transcripts $(m_1^i, c_i, m_2^i)$ is accepted by each verifier in $\Pi$.
            \item Ensure each $share(c_i)$ is consistent with $c_0$ (the secret) using the reconstruction algorithm in Definition \ref{def:sss}.
            \item If any of the checks fail return $b \leftarrow 0$; else return $b \leftarrow 1$.
        \end{itemize}
    \end{itemize}
\end{protocol}

Using this compiler, we can construct a $\Sigma$-protocol that has a communication complexity linear in the number of clauses.