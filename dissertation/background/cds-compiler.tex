\subsection{CDS94 Compiler}


In our implementation, we will use Schnorr's discrete log protocol over Ristretto25519 and Shamir's secret sharing scheme to demonstrate the compilation of $\Sigma$-protocol into a $\Sigma$-protocol for the disjunction of $n$ statements. 

We will attempt to make the implementation as general as possible to open up for the future possibility to take any $\Sigma$-protocol that suits our requirements and transform it disjunctive zero-knowledge $\Sigma$-protocol.

\subsubsection{The Witness Indistinguishable (WI) compilation}

In their paper, Cramer {\em{et al}} \cite{CDS94} presents 2 primary ways to construct a WI protocol from a $\Sigma$-protocol $\mathcal P$ (Theorem 8 and 9). 

\begin{itemize}
    \item Theorem 8 requires a smooth secret sharing scheme, and a HVZK $\Sigma$-protocol, while
    \item Theorem 9 requires special honest-verifier ZK (SHVZK) with at least a semi-smooth secret sharing scheme.
\end{itemize}

Since, SSS is a smooth threshold secret sharing scheme (required for 8), and Schnorr's protocol is SHVZK (required for 9), we can choose either construction. 
\textit{We will use \textbf{Theorem 8} in this project.}

\textbf{Theorem 8 \cite{CDS94}}. Given $\mathcal P$, $R_\Gamma$, $\Gamma$, and $\mathcal S(k)$ where

\begin{itemize}
    \item $\mathcal P$ is a 3-round public coin ($\Sigma$-protocol) HVZK proof of knowledge for relation $R$, which satisfies the special soundness property.
    \item $\Gamma = \{ \Gamma(k) \}$ is a family of monotone access structures
    \item $\{\mathcal S(k)\}$ is a family of smooth secret sharing schemes such that the access structure of $\mathcal S(k)$ is $\Gamma(k)^*$
    \item $R_\Gamma$ is a relation where $((x_1,\ldots,x_m),(w_1,\ldots,w_m)) \in R_\Gamma$ if and only if ($\iff$)
    \begin{itemize}
        \item all $x_i$'s are of the same length $k$, and 
        \item the set of indices $i$ for which $(x_i,w_i) \in R$ corresponds to a qualified set in $\Gamma(k)$
    \end{itemize}
\end{itemize}

Then, there exists a $\Sigma$-protocol that is witness indistinguishable for the relation $R_\Gamma$. (The proof of Theorem 8 and 9 is given in their paper.)

\subsubsection{Protocol Description}
Let $A \in \Gamma$ denote the set of indices $i$ for which our prover $P$ knows a witness for $x_i$

\begin{enumerate}
    \item For each $i \in \bar A$, $P$ runs a simulator on input $x_i$ to produce transcripts of conversations in the form $(m_1^i, c_i, m_2^i)$.
    \begin{itemize}
        \item For each $i \in A$, $P$ inputs the witness $w_i$ for $x_i$ to $\mathcal P$ and takes what the prover $P^*$ in $\mathcal P$ sends as $m_1$ as $m_1^i$.
        \item Essentially we take the return value of the first round of $\mathcal P$ as our message $m_1^i$.
        \item Finally, $P$ sends all $m_1^i$ to $V$, where $i = 1, \ldots, n$
    \end{itemize}
    \item $V$ chooses a $t$-bit string $s$ at random and sends it to $P$
    \item $P$ forms challenges $c_i$ for $i \in A$, such that $share(c_i) \cup \{share(c_j)|j \in \bar A\}$ is a qualified set in $\Gamma$ consistent with $s$.
    \begin{itemize}
        \item For $i \in A$, $P$ uses it's knowledge of $w(x_i)$ to compute a valid $m_2^i$ for $(m_1^i, c_i)$ by running the prover's algorithm in $\mathcal P$.
        \item $P$ then sends $c_i, m_2^i$ for $i = 1 ,\ldots, n$ to $V$. 
    \end{itemize}
    \item $V$ checks that all conversations $(m_1^i, c_i, m_2^i)$ would lead to acceptance by the verifier in $\mathcal P$
    \begin{itemize}
        \item During this process, $V$ checks that $share(c_i)$ is consistent with secret $s$.
        \item Accept if all true, otherwise reject. 
    \end{itemize}
\end{enumerate}