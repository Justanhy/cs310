% !TeX root = dissertation.tex
A zero-knowledge proof \cite{GMR85} is a protocol with two parties, the prover and  verifier, and an NP statement that is publicly known to both parties. The goal of the prover is to prove to the verifier that this statement is true without revealing any extra information. In other words, the verifier does not learn anything apart from the validity of the statement. Early research in this area showed that all languages in NP have 
zero-knowledge proof systems \cite{DBLP:conf/focs/GoldreichMW86}. This served as a catalyst for years of research into developing 
more efficient zero-knowledge proofs for various use cases. Today, efficient zero-knowledge proofs are being used 
in practice, with several use cases such as e-voting and secure decentralized systems \cite{evoting, zcash}.

It is often desirable to have a zero-knowledge proof for a \textit{disjunctive statement}.
A disjunctive statement is an NP 
statement with a set of clauses that are connected with logical ORs:

$$
clause_1 \lor clause_2 \lor \cdots \lor clause_n
$$

Such zero-knowledge proofs fall under the category of "disjunctive zero-knowledge": the goal is for the Prover 
to show that they have knowledge of at least 1 of the clauses while revealing no additional information. 
In other words, the Prover shows that at least 1 of the clauses is true and does not reveal 
information such as the location of these clauses or the "answer" to these clauses. 
We can think of each clause as a statement that has an "answer" to it -- formally this "answer" 
is known as a \textit{witness} for the statement. Clauses that the Prover has 
knowledge of are called the \textit{active clauses}. 
Disjunctive statements have very useful properties that occur commonly in practice, and adding zero-knowledge can be very beneficial to  
systems that desire both privacy and verifiability. For example, it can be used to prove an 
individual's membership to a particular group while revealing nothing about the individual's identity. 
They can also be used to show that a bug exists in a verifier's code base \cite{StackedGF}. Consequently, there has been broad research interest in determining how to construct disjunctive zero-knowledge proofs, and how to do so efficiently: saving on communication complexity (the size, in bytes, of the messages between prover and verifier) and computational complexity.  

One approach is to alter the underlying zero-knowledge protocols manually so that they are compatible with disjunctive statements. This approach has been studied by recent work, and it has shown that disjunctive zero-knowledge proofs can be attained with a communication complexity that is sub-linear in the number of clauses \cite{StackedGF,attema}. Unfortunately, this approach does not generalise well and only works for the individual protocols
they target. An alternative approach to this is to develop generic compilers for disjunctive zero-knowledge \cite{CDS94,StackingSigmas}. These 
compilers can target a large class of zero-knowledge proofs known as $\Sigma$-protocols (also known as 
3-round public coin proofs of knowledge).

Research targeting \textit{disjunctive zero-knowledge compilers} began with Cramer, Damg{\aa}rd, and Schoenmakers \cite{CDS94}. 
They proposed a generic compiler to compose multiple instances of $\Sigma$-protocols into a disjunctive zero-knowledge proof. According to their findings, the proof has a communication complexity that grows linearly with the number of clauses.
In a more recent development, Goel {\em et al.} \cite{StackingSigmas} go a step further by introducing a generic compiler that focuses on a large subset of $\Sigma$-protocols and achieves a communication complexity that is sub-linear in the number of clauses.

\section{Our Contributions}

Despite a considerable amount of research, there are few noteworthy instances of these compilers being applied in practical scenarios\footnote{Hall-Andersen \cite{MHAStackSig} provides a benchmark of applying the compiler from
\cite{StackingSigmas} to Schnorr's discrete log protocol \cite{Schnorr}.}. 
In this work, we seek to fill in this gap and have implemented
the CDS94 compiler \cite{CDS94} and the Stacking Sigmas compiler from \cite{StackingSigmas}. 
We also benchmark these protocols to measure and compare the difference in their performances. 
From our analysis of the collected data, we provide insights into the trade-offs between these 
protocols and suggest when they might be the most useful. 

As a project extension, we are also working on an implementation of the compiler 
introduced in "Speed-Stacking: Fast Sublinear Zero-Knowledge Proofs for Disjunctions" 
\cite{SpeedStacking}. This compiler builds on the work in \cite{StackingSigmas} and 
explores how sublinear-sized zero-knowledge proofs can be compiled into a 
sublinear-sized disjunctive zero-knowledge proof, with a \textit{sublinear} 
running time. Specifically, we are working on applying this compiler to Compressed 
$\Sigma$-protocols \cite{attema}. This is still a work in progress.

Our aim is for our work to serve as a basis for future research to compare newly developed compilers with the existing ones, which could result in further advancements with a wider impact on cryptographic systems that rely on this particular use case. 

\section{Related work}
\label{sec:related_work}
In their contribution to \cite{StackingSigmas}, Hall-Andersen introduces an implementation of the Stacking Sigmas compiler \cite{MHAStackSig}. Their approach involves using the Stacking Sigmas compiler in combination with Schnorr over Ristretto25519 \cite{ristretto_web} to create efficient ring signatures. In our work, we include a comparison of our implementation of the compilers (both \cite{CDS94} and \cite{StackingSigmas}) to Hall-Andersen's implementation. 
The key difference between our implementations is that we use a newer commitment scheme (present in the updated and latest version of 
\cite{StackingSigmas}). Additionally, we also seek to improve the usability of the compiler by providing a more user-friendly API. 



% Stacked Garbling implementation