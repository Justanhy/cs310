% !TeX root = dissertation.tex
Zero-Knowledge proofs and arguments \cite{GMR85} are protocols with two parties: a Prover, and a Verifier. 
The prover's goal is to convince the verifier of the validity of an NP statement, while revealing zero additional 
information except that the statement is valid. In other words, the verifier learns nothing new, except whether 
the statement being proven is true or false. Early research in this area showed that all languages in NP have 
zero-knowledge proof systems \cite{DBLP:conf/focs/GoldreichMW86}, sparking a long line of research into developing 
more efficient zero-knowledge proofs for various use cases. Today, efficient zero-knowledge proofs are being used 
in practice, with several use cases such as e-voting and to secure decentralized systems \cite{evoting, zcash}.

In many cases, it is desirable to have a zero-knowledge proof for a \textit{disjunctive statement}, which is an NP 
statement with a set of clauses that are connected with logical ORs:

$$
clause_1 \lor clause_2 \lor \cdots \lor clause_n
$$

Such zero-knowledge proofs fall under the category of "disjunctive zero-knowledge": the goal is for the Prover 
to show that they have knowledge of at least 1 of the clauses while revealing no additional information. 
In other words, the Prover shows that at least 1 of the clauses are true and does not reveal 
information such as the location of these clauses or the "answer" to these clauses. 
We can think of each clause as a statement that has an "answer" to it -- formally this "answer" 
is known as a \textit{witness} for the statement. Clauses that the Prover has 
knowledge of are called the \textit{active clauses}. 
Disjunctive statements have very useful properties that occur commonly in practice, and adding zero-knowledge can be very beneficial to  
systems that desire both privacy and verifiability. For example, a disjunctive zero-knowledge proof can be used to prove an 
individual's membership to a particular group while revealing nothing about the identity of the individual. They can also be 
used to reveal the existence of a bug in a verifier's code base as shown in \cite{StackedGF}. Consequently, there has been 
wide research interest in determining how to construct disjunctive zero-knowledge proofs, and how to do so efficiently: saving 
on communication complexity (the size, in bytes, of the messages between prover and verifier) and computational complexity.  

One approach is to modify the underlying protocols manually so that they support disjunctive statements. Exciting results from recent 
work has shown that it is possible to achieve protocols with a communication complexity sub-linear in the number of clauses using 
this approach \cite{StackedGF,attema}. Unfortunately, this approach does not generalise well and only work for the individual protocols 
that they target. A solution to this is to develop generic compilers for disjunctive zero-knowledge \cite{CDS94,StackingSigmas}. These 
compilers are able to target a large class of zero-knowledge proofs known as 3-round public coin proofs of knowledge (or more popularly 
known as $\Sigma$-protocols).

Research targeting \textit{disjunctive zero-knowledge compilers} began with Cramer, Damg{\aa}rd, and Schoenmakers \cite{CDS94}. 
They proposed a generic compiler to compose multiple instances of $\Sigma$-protocols into a disjunctive zero-knowledge proof. 
Their results showed that the communication complexity of the resulting proof is linear in the number of clauses.
More recently, Goel {\em et al.} \cite{StackingSigmas} improved on this further, providing a generic compiler that still targets 
a large class of $\Sigma$-protocols while achieving a communication complexity that is sub-linear in the number of clauses.

\section{Our Contributions}

While extensive research has been conducted, there is a lack of notable real-world implementations 
of these results
\footnote{It should be noted that Hall-Andersen \cite{MHAStackSig} has provided a benchmark of applying the compiler in 
\cite{StackingSigmas} to Schnorr's discrete log protocol \cite{Schnorr}.}. 
In this work, we seek to fill in this gap and build upon past research and have implemented
the CDS94 compiler \cite{CDS94} and the Stacking Sigmas compiler from \cite{StackingSigmas}. 
We also benchmark these protocols to measure and compare the difference in their performances. 
From our analysis of the collected data, we provide insights into the trade-offs between these 
protocols and suggest when they might be the most useful. 

As a project extension, we are also working on an implementation of the compiler 
introduced in "Speed-Stacking: Fast Sublinear Zero-Knowledge Proofs for Disjunctions" 
\cite{SpeedStacking}. This compiler builds on the work in \cite{StackingSigmas} and 
explores how sublinear-sized zero-knowledge proofs can be compiled into a 
sublinear-sized disjunctive zero-knowledge proof, which has a \textit{sublinear} 
running time. Specifically, we are working on applying this compiler to Compressed 
$\Sigma$-protocols \cite{attema}. This is still a work-in-progress.

We hope that our work helps lay the foundation for future work to compare 
new compilers to existing ones and in turn lead to further improvements 
that may have a broad impact on existing and upcoming cryptographic systems that rely on such a use case. 

\section{Related work}
\label{sec:related_work}
As part of their work in \cite{StackingSigmas}, Hall-Andersen provides an implementation of the Stacking Sigmas (SS) compiler \cite{MHAStackSig}. 
In this implementation, they apply the Stacking Sigmas compiler to Schnorr over Ristretto25519 to obtain efficient ring signatures. 
In our work, we include a comparison of our implementation of the compilers (both \cite{CDS94} and \cite{StackingSigmas}) to Hall-Andersen's implementation. 
The key difference between our implementations is that we use a newer commitment scheme (present in the updated and latest version of 
\cite{StackingSigmas}). Additionally, we also seek to improve the usability of the compiler by providing a more user-friendly API. 
More is discussed in Section \ref{sec:implementation}.



% Stacked Garbling implementation