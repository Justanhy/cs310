\begin{abstract}
    % Your abstract goes here. This should be about 2-3 paragraphs summarising the motivation for 
    % your project and the main outcomes (software, results, etc.) of your project. 
    Zero-knowledge proofs are protocols that allow a prover to prove the validity of a 
    statement to a verifier without revealing any information about the statement. There has 
    been a long line of research into zero-knowledge proofs -- 
    in particular, zero-knowledge proofs for disjunctive statements have been a core target. 
    Disjunctive statements are made up of a set of clauses joined by logical OR operators, 
    and the goal of the prover is to prove that at least one of the clauses is true. 
    
    In this work we look at one approach for constructing disjunctive zero-knowledge proofs: 
    general \emph{compilers} for zero-knowledge proofs. This approach has been explored by in early work by 
    Cramer and Damg{\aa}rd
    in \textquote{Proofs of Partial Knowledge and Simplified Design of Witness Hiding Protocols}, 
    and more recently by Goel {\em et al.} in \textquote{Stacking Sigmas: A Framework to Compose 
    $\Sigma$-Protocols for Disjunctions}. This project looks at implementing the compilers 
    proposed in these two papers, benchmarking their performance, and comparing the results. 
    There has been a lack of implementations of these compilers, and this work serves fill 
    in this gap, as well as to provide a better understanding of their performances, and 
    lay the groundwork for future work to benchmark against past work. 

    Our results show that Goel {\em et al.}'s compiler, Stacking Sigmas, outperforms Cramer and 
    Damg{\aa}rd's compiler, CDS94, in terms of communication complexity, providing significant 
    savings in the size of the proof as the number of clauses increase. We also show that 
    CDS94 is faster than Stacking Sigmas when the number of clauses are small, but ultimately
    slows down as the number of clauses increase due to how we have implemented the compiler.
    

\end{abstract}