Overall, we accomplished our goals of firstly implementing the CDS94 \cite{CDS94} and Stacking 
Sigmas \cite{StackingSigmas} compiler, and secondly benchmarking their performance. Through our 
work, we have shown that Stacking Sigmas outperforms CDS94 by providing significant savings in 
the communication size of the proof, especially as the number of clauses increase. 
Furthermore, while CDS94 is faster than Stacking Sigmas when the number of clauses are small,
we show that the difference in speed is negligible because the number of clauses are small. 
When the number of clauses increase, the difference in speed is more significant and CDS94 
performs worse than Stacking Sigmas in this case. Moreover, we also reveal that the bottleneck of the CDS94 compiler is the secret sharing scheme,
and highlight how the choice of this scheme and its implementation has a significant impact on the performance
of the compiler. This is all while we uphold our goal of keeping the implementation of the compiler
easily usable and extensible, in order to lay a good foundation for future work in this area. 

Additionally, there were many useful lessons that we learnt throughout the project. Firstly, we 
learnt the importance of proper project management and planning for a successful project. The 
risk management plan that we created at the start of the project helped us to identify potential
risks and to mitigate them. This allowed us to focus on the project and to avoid any distractions
that could have hindered our progress significantly. Secondly, we realised the importance 
and convenience of good testing practices. Because of our testing practices, we were able to
identify and fix bugs early, accelerating our progress. Lastly, we were able to learn more about 
the inner workings of zero-knowledge proofs and other cryptographic concepts. Work in this field 
is often theoretical, and it is interesting to see how these concepts can be implemented in practice.

\section{Future work}
\label{sec:futurework}
There are many directions that we can take this project in the future. 

\paragraph{Speed Stacking.} Firstly, our work on the 
Speed Stacking compiler \cite{SpeedStacking} is still ongoing. We have implemented the base 
$\Sigma$-protocol of our choice (Compressed $\Sigma$-protocols or Folding Arguments) and 
are currently working on the compiler. Possible extensions to this, would be to use the same 
compiler for interactive oracle proofs \cite{iops}. 

\paragraph{Stacking Sigmas.} Next, our implementation of Stacking Sigmas is the Self-Stacking 
implementation, where we use the same $\Sigma$-protocol for each clause. We can also implement
the Cross-Stacking compiler, which is compatible with different $\Sigma$-protocols for each clause.
This would allow us to compare the performance of the compiler when using different $\Sigma$-protocols, 
and possibly test how the choice of $\Sigma$-protocols affects the performance of the compiler. Another 
extension is to explore Goel {\em et al.}'s idea for a $k$-out-of-$l$ compiler (Section 9 of \cite{StackingSigmas}).
This idea relies on the parallel execution of $k$ 1-out-of-$l$ proofs together with a family of hash functions to 
ensure that the verifier can check that each execution is for a unique clause. 

\paragraph{CDS94.} Similar to Stacking Sigmas, we can also explore the idea of compiling different $\Sigma$-protocols
with CDS94. However, a more closely related idea is to explore the idea of using a different secret sharing scheme or 
a more efficient implementation of Shamir's secret sharing scheme. This would allow us to compare the performance of
the compiler when using different secret sharing schemes, and test how the choice of secret sharing scheme
affects the performance of the compiler. \\

All these suggestions, rely on the benchmarking model that we have implemented. A useful extension to this would be to 
implement a benchmarking framework that provides a convenient testing interface for different compilers. This framework 
may have features to produce mock data automatically and simply take instances of compilers and base protocols as input, 
and subsequently run the benchmarking process. It would also be useful to automatically generate CSV files from 
the raw data collected from these benchmarks, as they are currently manually extracted. This will help to enhance 
the benchmarking process and make it far more convenient to use. 

\section*{Acknowledgements}
\label{sec:acknowledgements}
We would like to thank our supervisor, Nicholas Spooner, for his invaluable guidance and support throughout the project.
