% !TeX root = dissertation.tex
Before discussing our implementation and results, we introduce the necessary background concepts for this project. 
We first formalise the notation we use in this report and highlight broad concepts that are relevant to the entire project. 
These concepts include zero-knowledge, disjunctive zero-knowledge, and $\Sigma$-protocols. 
After which, we split the remaining section into subsections: 
each for one of the compilers we implement \cite{CDS94, StackingSigmas}%, SpeedStacking}.

\section{Notation \& Terminology}\label{sec:notation}
\begin{table}[h]
  \centering
  \label{tab:notation}
  \caption{Notation used in this report}
  \begin{tabular}{|A{0.1\linewidth}|M{0.8\linewidth}|}
    \hline
    Symbol & Details \\\hline
    $\lambda$ & Computational security parameter. This refers to the ability of a cryptographic system to remain secure against an adversary who is computationally bounded. \\\hline
    $\kappa$ & Statistical security parameter. This pertains to the security provided by negligible statistical probability. \\\hline
    $\verify$ & Boolean assertion. \\\hline
    $\|$ & Bit concatenation: $0000 \| 1111 = 00001111$ \\\hline
    $\samplefrom$ & Sampling from a distribution. $x \samplefrom \mathcal D$ means that we sample $x$ from 
    the distribution "$\mathcal D$". \\
    \hline
    $[l]$ & The range of integers from 1 to $l$ \\
    \hline
  \end{tabular}
\end{table}

Throughout this paper, when "randomness" is mentioned in the definition of certain protocols, it usually refers to random bits generated by random number generators (RNG).

\subsection{Disjunctive Zero-Knowledge}

\begin{definition}[NP Relations]
Let $R \subseteq \{0,1\}^* \times \{0,1\}^*$ be a binary relation. Then $w(x) = \{w \mid (x,w) \in R\}$ and $L_R = \{x \mid \exists w, (x,w) \in R\}$. If $(x,w) \in R$, we say that $w$ is a witness for $x$. $R$ is an NP-relation if it fulfils the following two properties:
\begin{enumerate}
    \item \textbf{Polynomially bounded.} We say that $R$ is \textit{polynomially bounded} if there exists a polynomial $p$ such that $|w| \le p(|x|), \forall (x,w) \in R$. 
    \item \textbf{Polynomial-time verification.} There exists a polynomial-time algorithm for deciding membership in $R$. Consequently, $L_R \in NP$. 
\end{enumerate}

Throughout this document, we will use $\mathcal R$ to refer to a binary NP-relation.
\end{definition}

\begin{definition}[Zero-Knowledge]\label{def:zeroknowledge}
A proof or argument system $(P,V)$ is zero-knowledge over $\mathcal R$ if there exists a \textit{probabilistic polynomial time} (PPT) simulator $\mathcal S$, such that for all $(x,w) \in R$, the distribution of the output $\mathcal S(1^\lambda, x)$ of the simulator is indistinguishable from the distribution over the conversations generated by the interaction of $P$ and $V$, from the perspective of $V$; we denote this with $\ViewPV$. Conversations between $P$ and $V$ are ordered triples of the form $(a,c,z)$, and are known as \textit{transcripts}.
\end{definition}

Intuitively, this means that $V$ should not learn anything from the transcripts  with $P$ that they cannot already learn on their own by running the simulator $\mathcal S$; they learn nothing new.

\begin{definition}[Disjunctive Zero-Knowledge]
Given a sequence of statements $(x_1,x_2,\ldots, x_l)$, a \textit{disjunctive zero-knowledge proof} is a protocol to prove in zero-knowledge that $x_1 \in \mathcal L_1 \lor x_2 \in \mathcal L_2 \lor \ldots \lor x_l \in \mathcal L_l$, for NP languages $\mathcal L_i$. We term clauses for which the prover has a witness for as \textit{active} clauses. 
\end{definition}

\begin{definition}[Honest-Verifier Zero-Knowledge (HVZK)]
A proof system is HVZK if it only requires that $\mathcal S$ is an efficient simulator 
for honest (non-malicious) probabilistic polynomial time verifier strategies $V$. If $V$ is malicious then the distribution 
of the output $\mathcal S(x)$ will no longer be indistinguishable from $\ViewPV$ for such proof systems. 
\end{definition}


We model $\Sigma$-protocols with an interface.
Firstly, we design a set of generic types associated with the interface which will fit into 
the definition of the methods in the interface. These types are:
\begin{itemize}
  \item \texttt{Statement}: Public information about the protocol.
  \item \texttt{Witness}: Private information about the protocol (only known to Prover).
  \item \texttt{MessageA}: The first message of the protocol, sent from prover to verifier.
  \item \texttt{Challenge}: The challenge sent by the verifier to the prover.
  \item \texttt{MessageZ}: The third message of the protocol, sent from prover to verifier. 
  \item \texttt{State}: For most $\Sigma$-protocols, there is a particular state associated 
  to the execution of the first message. This state is often used by the prover again in 
  the third message but must not be sent to the verifier.
\end{itemize}

In particular, the design of the generic \texttt{State} type is worth discussing in more detail.
It is not explicitly mentioned in the formal definition of $\Sigma$-protocols, as it is 
assumed that the prover is able to track and store the private values that they obtain and 
require in the protocol. In the implementation, this has to be explicitly modelled and 
we do this using a functional programming approach, instead of an object-oriented
programming (OOP) approach. 
Firstly, this is because the functional approach is simpler, as we do not need to 
implement separate interfaces for the prover and verifier and subsequently include them as fields
within the \texttt{SigmaProtocol} interface. Secondly, the functional approach is more flexible,
as it does not restrict the user to a particular way of implementing the prover or verifier.
The functional approach simply provides two values in the first message (the state and the 
actual message), and the user of our interface can decide how to use these values.

Now, we present the methods that are associated with our $\Sigma$-protocol interface. 
These methods are: 
\begin{itemize}
  \item \texttt{first}: the first message of the protocol. This models algorithm 
  $a \leftarrow A(x, w; r^P)$ 
  in Definition \ref{def:sigma}.
  \item \texttt{second}: the second message of the protocol. $c \leftarrow \{0,1\}^\kappa$.
  \item \texttt{third}: the third message of the protocol. $z \leftarrow Z(x, w, c; r^P)$. 
  \item \texttt{verify}: verifies the transcript. $b \leftarrow \phi(x, a, c, z)$.
\end{itemize}

In Appendix \ref{code:SigmaProtocol} we provide a code snippet of these methods within 
the interface, which outlines which generic types are given as input and which are 
returned as output. Referring to the code snippet, readers will observe that our 
methods almost model the algorithms in Definition \ref{def:sigma} exactly in terms of 
input and output. The only difference is with the \texttt{State} type that we have already 
discussed.

\subsection{Schnorr's Protocol}
As a $\Sigma$-protocol, our first step in our implementation for Schnorr's protocol is 
to create concrete definitions for 
the methods and generic types of the \texttt{SigmaProtocol} interface. We use the following 
concrete types in the interface: 
\begin{itemize}
  \item \texttt{Statement}: \texttt{Schnorr} -- a type that simply contains the "public key"
  (which is the group element $H = x \cdot G$) as a field. We use the \texttt{RistrettoPoint} 
  type from \texttt{curve25519-dalek} to represent the public key, and it is essentially a 
  point on an Edward's curve. 
  \item \texttt{Witness}: \texttt{Scalar} -- also from the \texttt{curve22519-dalek} library. It 
  represents elements of the prime field in the Ristretto group. 
  \item \texttt{MessageA}: \texttt{CompressedRistretto} -- essentially the same as the 
  \texttt{RistrettoPoint} type, but differs in its representation. \texttt{CompressedRistretto}
  is a compressed representation of the point, which is more efficient to store and
  transmit.
  \item \texttt{Challenge}: \texttt{Scalar}
  \item \texttt{MessageZ}: \texttt{Scalar}
  \item \texttt{State}: \texttt{Scalar} 
\end{itemize}

The key takeaway is that we assign concrete types to the associated generic types provided in 
the interface for our implementation of Schnorr. The methods in the interface are then 
defined according to our definition of Schnorr in Definition \ref{def:schnorr}. 
\section{Schnorr's Identification Protocol}\label{sec:schnorr}
An \textit{identification scheme} involves two participants, a prover $P$ and a verifier $V$, whose goal is for $P$ to prove their identity to $V$. Specifically, $V$ should be persuaded that $P$ has access to the private key that corresponds to $P$'s public key. An example of an identification scheme is the commonly used password authentication protocol.

Schnorr's protocol, as described in \cite{Schnorr}, is a method of proving identity in which the prover, denoted by $P$, demonstrates knowledge of the discrete log $w$ of a group element $H$ in a finite abelian group $(\G, +)$ with $+$ as the binary operator\footnote{We choose to define $\G$ with the $+$ as our implementation uses elliptic curves which are finite abelian groups over addition. The equivalent definition with a group defined over the multiplicative operator is $h = g^w$}. Specifically, $H = w \cdot G$ for some generator $G \in \G$. The protocol relies on the discrete log assumption \cite{Diffie1976NewDI}, which states that computing $w$ given $H$ and $G$ is currently computationally infeasible, assuming that $\G$ is large enough. However, proving that $H = w \cdot G$ given $w$ and $G$ can be done efficiently. 

Our implementation of the protocol will use the Ristretto group \cite{ristretto_web}, which is constructed from a family of elliptic curves known as Edwards curves \cite{Edwards2007} and is of prime order. 

\begin{definition}[Schnorr's Protocol \cite{Schnorr}]\label{def:schnorr}
    Let $\G$ be an elliptic curve over the finite field $\F_q$ and let $E(\F_q)$ denote the group of points on $\G$. Suppose that the prover $P$ and verifier $V$ agree on $\G$ and $\F_q$, and let $H \in E(\F_q)$ be the public key that corresponds to the private key $w$, where $H = w \cdot G$. The prover convinces the verifier that they possess knowledge of the private key $w$ by following Protocol \ref{prot:schnorr}.
\end{definition}

\begin{protocol}[label={prot:schnorr}]{Schnorr's protocol. Public information: $\G = E(\F_q),\ H \in \G$. Statement to prove: $H = w \cdot G$} 
    \vspace{-0.5cm}
    \begin{align*}
        &\text{Prover}(w_p)& 
        &&
        &\text{Verifier}& 
        \\
        &r \samplefrom \F_q,\ \mathcal A = r \cdot G&
        &\overset  {\mathcal A} {\xrightarrow{\hspace{3cm}}}&
        && 
        \\
        &&
        &&
        &c \samplefrom \F_q&
        \\
        &&
        &\overset c {\xleftarrow{\hspace{3cm}}}&
        &&
        \\
        &z \leftarrow cw_p + r&
        &&
        && 
        \\
        &&
        &\overset {z} {\xrightarrow{\hspace{3cm}}}&
        &z \cdot G \verify \mathcal A + c \cdot H&
    \end{align*}
    \tcblower
    Communication between Prover and Verifier in Schnorr's protocol. Solving for the final equation shows  that $V$ accepts if and only if $w = w_P$.
    $$
    \begin{gathered}
    z \cdot G = \mathcal A + c \cdot H \iff
    r \cdot G + c \cdot w_P \cdot G  = r \cdot G + c \cdot w \cdot G \iff
    w  =  w_P
    \end{gathered}
    $$
\end{protocol}



\paragraph{Simulator $\mathcal S$ for Schnorr's protocol.} From our definition of a Zero Knowledge protocol in Definition 
\ref{def:zeroknowledge}, we shared that every zero-knowledge proof system has a PPT algorithm called a simulator. We can construct this simulator by running the protocol "in reverse" shown in Figure \ref{fig:schnorr-sim}.

\begin{figure}[h]
    \centering
    \begin{problem}[width=\linewidth/2]{Let $\mathcal S(H)$ be our simulator}
        $z \samplefrom \F_q$ 
        
        $c \samplefrom \F_q$
        
        $\mathcal A = z \cdot G - c \cdot H$
        \tcblower
        \textbf{output} $(\mathcal A,c,z)$
    \end{problem}
    \caption{A Simulator for Schnorr}
    \label{fig:schnorr-sim}
\end{figure}

This is a valid simulator as the resulting $\mathcal A$ is random because $z$ and $c$ are chosen randomly. This means that the output of our simulator is effectively random and will have the same distribution over the transcript in a real interaction.
Note that the simulator presented here achieves only honest-verifier zero-knowledge (Definition
\ref{def:hvzk}) as the only input to the simulator is the statement to prove $H$. We can 
easily modify this simulator to achieve special honest-verifier zero-knowledge (Definition 
\ref{def:sigma}) by adding a challenge $c$ as an input to the simulator.



% CDS %
\section{CDS94}
In this section, we introduce the components necessary for the \cite{CDS94} compiler. In addition to a $\Sigma$-protocol, which 
is relevant to every compiler in this project, the CDS94 compiler requires a \emph{secret sharing scheme} and the compiler 
itself. In the following 2 subsections, we introduce these components.
\subsection{Shamir's Secret Sharing Scheme}
A \textit{secret sharing scheme} is a method of distributing a secret $s$ to $n$ participants in a way that no one participant has intelligible information about the secret. This is achieved by splitting up $s$ into \textit{shares}, distributing one share to each participant in a way that a subset of participants can reconstruct $s$. Subsets that can reconstruct the secret are called \textit{qualified sets}. For \textit{perfect} secret sharing schemes, like Shamir's, participants in the complement \textit{non-qualified} sets cannot obtain any information about the secret.

Shamir's secret sharing scheme \cite{DBLP:journals/cacm/Shamir79} is also what is known as a \textit{threshold sharing scheme}. These are schemes that produce  qualified sets of size $d$. Any $d$ out of $n$ participants can reconstruct the secret; with $d-1$ shares and less, no  information about the secret can be obtained. 

\textbf{Share Reconstruction for Shamir's}
Explain in more detail in the future, but essentially:
\begin{enumerate}
    \item Given secret $s$, unqualified set of shares $U$, and array of indexes for active clauses $A$
    \item Firstly, ensure that threshold is set to $N - d + 1$, where $d$ is the number of active clauses; in other words the threshold should be $k + 1$ where $k$ is the number of inactive clauses and $|U| = k$. 
    \item Construct a lagrange polynomial of degree $k$, with the secret $s$ and set $U$. 
    \item With this polynomial, interpolate at $x = i$ for $i \in A$ to determine the share for each $i \in A$. 
    \item Take these as $share(c_i)$, taking the first relevant number of bits if somehow the challenge is smaller than the shares, and extrapolating with random bits if the challenge is meant to be larger.
\end{enumerate}
\subsubsection{CDS94 Compiler}
In this paper, Cramer {\em{et al}} \cite{CDS94} presents 2 primary ways to compile $\Sigma$-protocols depending 
on the underlying choice of $\Sigma$-protocol and the secret sharing scheme. Our implementation makes use of Theorem 
8 of the paper because we choose to use Schnorr's protocol (Section \ref{sec:schnorr}) and Shamir's secret sharing 
(Section \ref{sec:sss}). More details regarding our implementation will be discussed in a further section. Now, we recall 
theorem 8 of \cite{CDS94} -- note that we alter the notation slightly from the original paper to be more consistent 
with the rest of this report.

% In our implementation, we will use Schnorr's discrete log protocol over Ristretto25519 and Shamir's secret sharing scheme to demonstrate the compilation of $\Sigma$-protocol into a $\Sigma$-protocol for the disjunction of $n$ statements. 
% We will attempt to make the implementation as general as possible to open up for the future possibility to take any $\Sigma$-protocol that suits our requirements and transform it disjunctive zero-knowledge $\Sigma$-protocol.

% \subsubsection{The Witness Indistinguishable (WI) compilation}

% In their paper, Cramer {\em{et al}} \cite{CDS94} presents 2 primary ways to construct a WI protocol from a $\Sigma$-protocol $\mathcal P$ (Theorem 8 and 9). 

% \begin{itemize}
%     \item Theorem 8 requires a smooth secret sharing scheme, and a HVZK $\Sigma$-protocol, while
%     \item Theorem 9 requires special honest-verifier ZK (SHVZK) with at least a semi-smooth secret sharing scheme.
% \end{itemize}

% Since, SSS is a smooth threshold secret sharing scheme (required for 8), and Schnorr's protocol is SHVZK (required for 9), we can choose either construction. 
% \textit{We will use \textbf{Theorem 8} in this project.}

\textbf{Theorem 8 \cite{CDS94}}. Given $\Pi = (A, Z, \phi)$, $\mathcal R_\Gamma$ and $\mathcal S(k)$ where

\begin{itemize}
    \item $\Pi$ is a 3-round public coin ($\Sigma$-protocol) HVZK proof of knowledge for relation $\mathcal R$.
    \item $\{\mathcal S(k)\}$ is a family of smooth secret sharing schemes.
    \item $\mathcal R_\Gamma$ is a relation where $((x_1,\ldots,x_m),(w_1,\ldots,w_m)) \in \mathcal R_\Gamma$ if and only if ($\iff$)
    \begin{itemize}
        \item all $x_i$'s are of the same length $k$, and 
        \item the set of indices $i$ for which $(x_i,w_i) \in \mathcal R$ corresponds to a qualified set for $S(k)$
    \end{itemize}
\end{itemize}

Then, there exists a $\Sigma$-protocol, $\Pi' = (A', Z', \phi')$, that is witness indistinguishable 
(Definition \ref{def:wi}) for the relation $\mathcal R_\Gamma$. The description of this protocol is outlined in 
Protocol \ref{prot:cds-compiler}. Interested readers can refer to the original paper for the proof of this theorem \cite{CDS94}. 

\begin{protocol}[label={prot:cds-compiler}]{CDS94 Compiler. A compiler for composing $n$ instances of a 
    $\Sigma$-protocol $\Pi$ into a single $\Sigma$-protocol $\Pi'$ that proves the \textbf{disjunction} 
    of these $n$ instances.}
    Let $A$ be the set of indices $i$ of the \textit{active clauses}. \\
    \textbf{Statement:} $x = x_1,\ldots, x_n$ \\
    \textbf{Witness:} $w = \{w_i\st i \in A\ \land (x_i, w_i) \in R\}$
    \begin{itemize}
        \item \textbf{First Round:} the Prover, $P$, computes $A'(x,w; r^P) \rightarrow a$ accordingly:
        \begin{itemize}
            \item For each $i \in \bar A$, run the simulator for $\mathcal P$ for the statement $x_i$ to produce the transcripts $(m_1^i, c_i, m_2^i)$.
            \item For each $i \in A$, compute $m_1^i \leftarrow A(x_i, w_i; r^P)$.
            \item Send $a \leftarrow (m_1^1, \ldots, m_1^n)$ to $V$.
        \end{itemize}
        \item \textbf{Second Round:} $V$ sends $c_0 \leftarrow \{0,1\}^\lambda$ to $P$. 
        \item \textbf{Third Round:} Prover computes $Z'(x,w,c;r^P) \rightarrow z$:
        \begin{itemize}
            \item Compute $Q \leftarrow Complete(c_0, U, A)$, where $U = \{c_i\st i \in \bar A\}$.
            \item Set $c_i = share(i)$, for each $share(i) \in Q$
            \item For $i \in A$, compute $m_2^i \leftarrow Z(x_i, w_i, c_i; r^P)$. 
            \item For $i \in \bar A$, $m_2^i$ has already been computed in the first round using the simulator.
            \item Send $z \leftarrow ((c_1, m_2^1), \ldots, (c_n,m_2^n))$ to $V$.
        \end{itemize}
        \item \textbf{Verification:} Verifier computes $\phi'(x,a,c_0,z) \rightarrow b$ as follows:
        \begin{itemize}
            \item Extract $c_i$ and $m_2^i$ from $z \leftarrow ((c_1, m_2^1), \ldots, (c_n,m_2^n))$
            \item Check that all conversations $(m_1^i, c_i, m_2^i)$ would lead to acceptance by the verifier in $\mathcal P$.
            \item Check that each $share(c_i)$ is consistent with the secret $c_0$ using the reconstruction algorithm in Definition \ref{def:sss}.
            \item If any of the checks fail return $b \leftarrow 0$; else return $b \leftarrow 1$.
        \end{itemize}
    \end{itemize}
\end{protocol}

Using this compiler, we can construct a $\Sigma$-protocol that has a communication complexity linear in the number of clauses.

% Stacking Sigmas %
\section{Stacking Sigmas}
Moving on, we introduce the concept of a \emph{partially-binding vector commitment scheme} 
and present the Stacking Sigmas compiler proposed by \cite{StackingSigmas}. This compiler aims to improve on the 
communication complexity of the \cite{CDS94} compiler, by reducing the communication size to $O(\log n)$, where $n$ is the 
number of clauses in the disjunction. 
% Partially Binding Vector Commitment from Discrete Log
% !TeX root = ../dissertation.tex
\subsubsection{Partially-Binding Vector Commitments}
In Section 5 of \cite{StackingSigmas}, Goel {\em{et al}} introduce the concept of partially-binding vector commitment schemes. 
These commitment schemes allow a Prover to make a commitment to a vector of length $l$ with $t$ binding positions; 
the remaining positions (indexes) in the vector are equivocable. 
Here, we recall the definition of these commitment schemes and refer interested readers to Figure 2 of \cite{StackingSigmas} for the construction of the general $t$-out-of-$l$ 
scheme. 

\subsubsection{Half-Bindings \& Q-Bindings}
In this work, we make use of the generic 1-out-of-$2^q$ construction provided in Section 5.3 of \cite{StackingSigmas} for our implementation of the Stacking Sigmas compiler. We use this 
particular construction as it allows us to yield commitments that are logarthmic in size (in bytes) to the original vector of messages. 
A core ingredient in this construction is a 1-out-of-2 partially-binding vector commitment scheme, which is used recursively to form a \textit{binary} "tree of commitments". 
For brevity, we refer to the generic construction as "q-bindings" and the 1-out-of-2 construction as "half-bindings".
The leaves of this tree are the original messages in our vector; intermediate nodes are the resulting commitments to the nodes' children using half-bindings. 
Essentially, intermediate commitments are regarded as messages as well, and are commited to recursively until there is only 1 root node -- the final commitment. 

While the source material provides the general $t$-out-of-$l$ construction for partially-binding vector commitments, it does not explicitly provide that for half-bindings. 
In Figure \ref{fig:half-binding}, we provide the construction for half-bindings derived from the general $t$-out-of-$l$ construction. 
The proof of correctness for this construction is trivial as it can be easily seen to be a specific case of the general construction when $t = 1$ and $l = 2$. 

\begin{figure}
  \centering
  \begin{construction}[]
    \begin{minipage}{0.45\linewidth}
      \underline{$pp \leftarrow$ Setup$(1^\lambda)$:}
      \begin{enumerate}
        \item $\G \leftarrow $ GenGroup$(1^\lambda)$; $g_0, h \samplefrom \G$
        \item $\textbf{return } pp \leftarrow (\G, g_0, h)$
      \end{enumerate}  
    \end{minipage}
    \begin{minipage}{0.5\linewidth}
      \underline{$(com, aux) \leftarrow$ EquivCom$(pp, ek, m_1, m_2)$:}
      \begin{enumerate}
        \item Extract $ck$ from $ek$
        \item $r \samplefrom \Z_{|\G|}^2$
        \item com $\leftarrow$ BindCom$(pp,ck,m_1,m_2,r)$
        \item \textbf{return} $(\text{com}, r)$
      \end{enumerate}  
    \end{minipage}

    \underline{$(ck, ek) \leftarrow$ Gen$(pp, B)$:}
    \begin{enumerate}
      \item Let $E = \{1,2\} \setminus B$ be the set of equivocal indexes. $|B| = 1$.
      \item Generate trapdoor $td$ for $i \in E: td \samplefrom \Z_{|\G|}, g_i \leftarrow h \cdot r_i$
      \item Interpolate the first element: $g_1 \begin{cases}
       g_2 - g_0 & \text{if $2 \in E$} \\
       g_1 & \text{if $1 \in E$}
      \end{cases}$
      \item $ck \leftarrow g_1$
      \item $ek \leftarrow (B, td, ck)$
      \item \textbf{return} $(ck, ek)$
    \end{enumerate}

    \underline{$com \leftarrow$ BindCom$(pp,ck, m_1,m_2, r)$:}
    \begin{enumerate}
      \item Interpolate $g_2$: $g_2 = g_1 + g_0$
      \item $(r_1, r_2) \leftarrow r$
      \item Commit individually $\textbf{for} j \in \{1,2\}: com_j \leftarrow h\cdot r_j + g_j \cdot m_j \in \G$
      \item \textbf{return} $(com_1,com_2)$
    \end{enumerate}

    \underline{$r \leftarrow$ Equiv$(pp, ek, m_1, m_2, m_1', m_2', aux)$:}
    \begin{enumerate}
      \item Extract $B$ and $td$ from $ek$; Let $E = \{1,2\} \setminus B$.
      \item Parse $aux = (r_1, r_2) \in \Z_{|\G|}^2$
      \item Interpolate $g_2 = g_1 + g_0$
      \item for $i \in B: r'_i \leftarrow r_i$
      \item for $i \in E: r'_i \leftarrow r_i - td \cdot (m_i' - m_i)$
      \item \textbf{return} $r' \leftarrow (r_1', r_2')$
    \end{enumerate}
  \end{construction}
  \caption{Construction of a 1-out-of-2 partially-binding vector commitment scheme.}
  \label{fig:half-binding}
\end{figure}


% Section on stacking sigmas - talk about how stacking works
\subsubsection{Stacking Sigmas Compiler}
In Section 6 of \cite{StackingSigmas}, Goel \emph{et al} introduce the two main properties of stackable $\Sigma$-protocols: \textit{extended honest verifier 
zero-knowledge}, and \textit{recyclable third round messages}. Moreover, they go on to show that many $\Sigma$-protocols satisfy these properties, 
proving that all $\Sigma$-protocols can be made EHVZK (Observation 1), and many natural $\Sigma$-protocols have recyclable third round messages. 

In this section, we state the definition of these two properties as shown in \cite{StackingSigmas}, but do not dive into the details of how a large class of 
$\Sigma$-protocols have these two properties. We encourage readers to refer to \cite{StackingSigmas} for the proofs of these properties and the observations 
related to them.

\begin{definition}[Extended Honest Verifier Zero-Knowledge -- EHVZK]
  The $\Sigma$-protocol $\Pi = (A, Z, \phi)$ is EHVZK if there exists a \textit{deterministic} polynomial time "extended simulator" algorithm 
  $\mathcal S^{\text{\tiny EHVZK}}(1^\lambda, x, c, z)$ such that the following two distributions are indistinguishable for any $(x,w) \in \mathcal R$ 
  and $c \in \{0,1\}^\kappa$:
  \begin{multline*}
    \left\{
      (a,c,z) \st r^P \samplefrom \{0,1\}^\lambda; a \leftarrow A(x, w; r^P); z \leftarrow Z(x,w,c; r^P)
    \right\} \\
    \approx
    \left\{
      (a,c,z) \st z \samplefrom \mathcal D_{x,c}^{(z)}; a \leftarrow \mathcal S^{\text{\tiny EHVZK}}(1^\lambda, x, c, z)
    \right\}
  \end{multline*}
  Intuitively, this means that the simulation starts with a specific third round message $z$, and determining a unique first round message $a$ with respect to 
  $z$ and a fixed challenge $c$.
\end{definition}

\begin{definition}[Recyclable 3rd Round Messages]
  A $\Sigma$-protocol $\Pi = (A, Z, \phi)$ for $\mathcal R$ has recyclable third round messages if for every challenge $c \in \{0,1\}^\kappa$, there is an 
  efficiently sampleable distribution $\mathcal D_c^{(z)}$:
  $$
    \mathcal D_c^{(z)} \approx \left\{z \st r^P \samplefrom \{0,1\}^\lambda; a \leftarrow A(x,w;r^P); z \leftarrow Z(x,w,c;r^P) \right\}\quad 
    \forall (x,w) \in \mathcal R 
  $$
\end{definition}
  Essentially, this means that the distribution of $z$ is not dependent on the statement $x$ and hence does not "leak information" about the statement. 
  Consequently, this means that $z$ can be reused for any statement $x$, even non-active clauses, and still hide the active clauses. 

  Now, we present the Stacking Sigmas Compiler for stacking disjunctions of the same protocol (Section 7 of \cite{StackingSigmas}). The compiler 
  presented in Protocol \ref{prot:stacksig-compiler} is identical to that in Figure 5 of \cite{StackingSigmas}, and is produced by Theorem 5 of \cite{StackingSigmas}.

  \paragraph{Theorem 5 of \cite{StackingSigmas}.} Given a \textit{stackable} $\Sigma$-protocol $\Pi = (A, Z, \phi)$ and a 
  1-out-of-$l$ binding vector commitment scheme, we can produce a compiled $\Sigma$-protocol $\Pi' = (A', Z', \phi')$ that is also \textit{stackable}. If 
  $\Pi$ is for the relation $\mathcal R: \mathcal X \times \mathcal W \rightarrow \{0,1\}$, then $\Pi'$ is for the relation $\mathcal R': \mathcal X^l \times 
  ([l] \times \mathcal W) \rightarrow \{0,1\}$, where $\mathcal R'((x_1,\ldots, x_n), (a,w)) := \mathcal R(x_a, w)$.
  The 
  description of $\Pi'$ is shown in Protocol $\ref{prot:stacksig-compiler}$, and the proof of this theorem can be found in \cite{StackingSigmas}.

\begin{protocol}[label={prot:stacksig-compiler}]{Stacking Sigmas Compiler. A compiler for composing $n$ instances of a 
  $\Sigma$-protocol $\Pi$ into a single $\Sigma$-protocol $\Pi'$ that proves the \textbf{disjunction} 
  of these $n$ instances.}
  \textbf{Statement:} $x = x_1,\ldots, x_n$ \\
  \textbf{Witness:} $w = (\alpha, w_\alpha)$
  \begin{itemize}
      \item \textbf{First Round:} the Prover, $P$, computes $A'(x,w; r^P) \rightarrow a$ accordingly:
      \begin{itemize}
          \item Parse $r^P = (r^P_\alpha \| r)$
          \item Compute $a_\alpha \leftarrow A(x_\alpha, w_\alpha; r^P_\alpha)$
          \item Set $\mathbf{v} = (v_1,\ldots, v_l)$, where $v_\alpha = a_\alpha$ and $\forall i \in [l] \setminus \alpha,\ v_i = 0$.
          \item Compute $(\textsf{ck}, \textsf{ek}) \leftarrow \textsf{Gen}(\textsf{pp}, B = \{\alpha\})$.
          \item Compute $(\textsf{com}, \textsf{aux}) \leftarrow \textsf{EquivCom}(\textsf{pp}, \textsf{ek}, \mathbf{v}; r)$.
          \item Send $a = (\textsf{ck, com})$ to the verifier. 
      \end{itemize}
      \item \textbf{Second Round:} $V$ sends $c \leftarrow \{0,1\}^\lambda$ to $P$. 
      \item \textbf{Third Round:} Prover computes $Z'(x,w,c;r^P) \rightarrow z$:
        \begin{itemize}
          \item Parse $r^P = (r^P_\alpha \| r)$
          \item Compute $z^* \leftarrow Z(x_\alpha, w_\alpha, c; r^P_\alpha)$
          \item For $i \in [l] \setminus \alpha$, compute $a_i \leftarrow \mathcal S^{\text{\tiny EHVZK}}(x_i, c, z^*)$. 
          \item Set $\mathbf{v}' = (a_1, \ldots, a_l)$
          \item Compute $r' \leftarrow \textsf{Equiv}(\textsf{pp, ek,} \mathbf{v}, \mathbf{v}'; \textsf{aux})$ (where $\textsf{aux}$ can be 
          regenerated with $\textsf{r}$).
          \item Send $z = (\textsf{ck}, z^*, r')$ to the verifier.
        \end{itemize}
      \item \textbf{Verification:} Verifier computes $\phi'(x,a,c_0,z) \rightarrow b$ as follows:
      \begin{itemize}
        \item Parse $a = (\textsf{ck, com})$ and $z = (\textsf{ck}, z^*, r')$.
        \item Set $a_i \leftarrow \mathcal S^{\text{\tiny EHVZK}}(x_i, c, z^*)$
        \item Set $\mathbf{v}' = (a_1, \ldots, a_l)$
        \item Compute and return: 
      \end{itemize}
      \[
          b = (\textsf{ck} \verify \textsf{ck}') \land \left(\textsf{com} \verify \textsf{BindCom}(\textsf{pp, ck}, \mathbf{v}'; r')\right) \land 
          \left(\bigwedge_{i \in [l]} \phi(x_i, a_i, c,z^*)\right)
        \]
  \end{itemize}
\end{protocol}

\paragraph{Communication Complexity.} By using the Q-Binding construction (Figure 4 of \cite{StackingSigmas}) together with Half-Bindings 
(Figure \ref{fig:half-binding}) we obtain a $1$-out-of-$2^q$ 
binding vector commitment scheme with communication complexity $q \cdot CC(\textsf{HalfBinding})$ 
(\textbf{Theorem 2 \& Corollary 1} in \cite{StackingSigmas}). According to the proofs in Theorem 5 \cite{StackingSigmas}, the communication complexity of 
$\Pi'$ is 
$CC(\Pi_l) = CC(\Pi) + |\textsf{ck}| + |\textsf{com}| + |\textsf{r'}|$, where the last three components are directly related to the choice of the 
partially-binding vector commitment. 

Since we are using a Q-Binding construction, we get the following communication complexity: 
$CC(\Pi_l) = CC(\Pi) + q(|\textsf{ck}_{1/2}| + |\textsf{com}_{1/2}| + |\textsf{r'}_{1/2}|)$. From Figure \ref{fig:half-binding}, we can see that 
$|\textsf{ck}_{1/2}| = |g|$, $|\textsf{com}_{1/2} = 2|g|$, and $|r'_{1/2}| = 2|z|$, where $g \in \G$ and $z \in \Z_{|\G|}$. In this case, $|g|$ 
depends on the number of bytes required to represent the group elements in $\G$, and $|z|$ depends on the size of the group $|\G|$. Therefore, 
$CC(\Pi_l) = CC(\Pi) + q(3|g| + 2|z|) = CC(\Pi) + \log_2(l)(3|g| + 2|z|)$.


% Speed Stacking %
% \subsection{Homomorphisms}
A homomorphism between two algebraic objects $A$ and $B$ is a function $f: A \rightarrow B$ which preserves their algebraic structure. 
\begin{enumerate}
    \item If elements in $A$ satisfy some algebraic equation involving addition or multiplication, their images in $B$ also satisfy the same algebraic equation
    \item The details of the definitions of homomorphisms depend on algebraic structures of $A$ and $B$
\end{enumerate}

\begin{definition}[Group Homomorphisms]
Suppose we have arbitrary groups $A$ and $B$, with the group operators $\circ_A$ and $\circ_B$ respectively. A group homomorphism is a function $f: A \rightarrow B$ such that $f(a_1 \circ_A a_2) = f(a_1) \circ_B f(a_2)$ for all $a_1,a_2 \in A$
\end{definition}
% % section on compressed sigma protocols (compression mechanism)
% \subsection{Compressed $\Sigma$-protocols are stackable}
\begin{protocol}[label={prot:compressed-sigs}]{Base $\Sigma$-protocol $\Pi_0$ for the relation 
$$\mathcal R_{\text{compressed}} = \{(g \in \G^N, P \in \G, y \in G_T; \mathbf{x} \in \Z_q^N): P = \mathbf{g^x}, y = f(\mathbf{x})\}$$ introduced in \cite{attema}} 
   \vspace{-0.5cm}
   \begin{align*}
       &\text{Prover}& 
       &&
       &\text{Verifier}& 
       \\
       &r \samplefrom \Z_q^N,\ t = f(\mathbf{r}),\ T = \mathbf{g^r}&
       &\overset  {t, T} {\xrightarrow{\hspace{3cm}}}&
       && 
       \\
       &&
       &&
       &c \samplefrom \Z_q&
       \\
       &&
       &\overset c {\xleftarrow{\hspace{3cm}}}&
       &&
       \\
       &\mathbf{z} \leftarrow c\mathbf{x} + \mathbf{r}&
       &&
       &&
       \\
       &&
       &\overset {\mathbf{z}} {\xrightarrow{\hspace{3cm}}}&
       &f(\mathbf{z}) \verify cy + t, \ \mathbf{g^z} \verify TP^c&
   \end{align*}
   \label{prot:base-compressed}
\end{protocol}
% % section on speed stacking compiler 
% \input{background/speed-stacker.tex}

