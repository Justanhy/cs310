\section{Background}
\label{sec:background}

\subsection{Notation}
Throughout this paper, we use $\lambda$ to denote the computational security parameter and $\kappa$ to denote the statistical security parameter. Computational security refers to a cryptographic system's security against a computationally bounded adversary, while statistical security refers to security that is not dependent on the computational power of the adversary but instead on the security that negligible statistical probability provides.

We denote by $x \samplefrom \mathcal D$ the sampling of "$x$" from the distribution "$\mathcal D$", and $\verify$ means a boolean assertion of equality on the left-hand side and right-hand side of the symbol.

% We denote by $x \randselect \mathcal D$ the process of sampling "$x$" fro the distribution "$\mathcal D$" using pseudorandom coins derived from a pseudo-random generator (PRG) applied to the seed "$s$". 

\subsection{Disjunctive Zero-Knowledge}

\begin{definition}[NP Relations]
Let $R \subseteq \{0,1\}^* \times \{0,1\}^*$ be a binary relation. Then $w(x) = \{w \mid (x,w) \in R\}$ and $L_R = \{x \mid \exists w, (x,w) \in R\}$. If $(x,w) \in R$, we say that $w$ is a witness for $x$. $R$ is an NP-relation if it fulfils the following two properties:
\begin{enumerate}
    \item \textbf{Polynomially bounded.} We say that $R$ is \textit{polynomially bounded} if there exists a polynomial $p$ such that $|w| \le p(|x|), \forall (x,w) \in R$. 
    \item \textbf{Polynomial-time verification.} There exists a polynomial-time algorithm for deciding membership in $R$. Consequently, $L_R \in NP$. 
\end{enumerate}

Throughout this document, we will use $\mathcal R$ to refer to a binary NP-relation.
\end{definition}

\begin{definition}[Zero-Knowledge]\label{def:zeroknowledge}
A proof or argument system $(P,V)$ is zero-knowledge over $\mathcal R$ if there exists a \textit{probabilistic polynomial time} (PPT) simulator $\mathcal S$, such that for all $(x,w) \in R$, the distribution of the output $\mathcal S(1^\lambda, x)$ of the simulator is indistinguishable from the distribution over the conversations generated by the interaction of $P$ and $V$, from the perspective of $V$; we denote this with $\ViewPV$. Conversations between $P$ and $V$ are ordered triples of the form $(a,c,z)$, and are known as \textit{transcripts}.
\end{definition}

Intuitively, this means that $V$ should not learn anything from the transcripts  with $P$ that they cannot already learn on their own by running the simulator $\mathcal S$; they learn nothing new.

\begin{definition}[Disjunctive Zero-Knowledge]
Given a sequence of statements $(x_1,x_2,\ldots, x_l)$, a \textit{disjunctive zero-knowledge proof} is a protocol to prove in zero-knowledge that $x_1 \in \mathcal L_1 \lor x_2 \in \mathcal L_2 \lor \ldots \lor x_l \in \mathcal L_l$, for NP languages $\mathcal L_i$. We term clauses for which the prover has a witness for as \textit{active} clauses. 
\end{definition}

\begin{definition}[Honest-Verifier Zero-Knowledge (HVZK)]
A proof system is HVZK if it only requires that $\mathcal S$ is an efficient simulator 
for honest (non-malicious) probabilistic polynomial time verifier strategies $V$. If $V$ is malicious then the distribution 
of the output $\mathcal S(x)$ will no longer be indistinguishable from $\ViewPV$ for such proof systems. 
\end{definition}


\subsection{$\Sigma$-protocols}
\begin{definition}[$\Sigma$-Protocol \cite{StackingSigmas}]
Let $\mathcal R$ be an NP relation. A $\Sigma$-protocol $\Pi = (A, Z, \phi)$ for $\mathcal R$ is a 3-round protocol between a prover algorithm $P$ and a verifier algorithm $V$. The protocol consists of a tuple of probabilistic polynomial time algorithms $(A, Z, \phi)$ with the following interfaces:
\begin{itemize}
    \item $a \leftarrow A(x,w; r^p)$ : Given statement $x$, witness $w \in w(x)$, and prover randomness $r^p$ as input; output the first message $a$ that $P$ sends to $V$ in the first round. 
    \item $c \samplefrom \{0,1\}^\kappa$: $V$ samples a random challenge $c$ and sends it to $P$ in the second round. 
    \item $z \leftarrow Z(x,w,c; r^p)$: Given $x$, $w$, $c$, and $r^p$ as input; output the message $z$ that $P$ sends to $V$ in the third round.
    \item $b \leftarrow \phi(x,a,c,z)$: Given $x$, and the messages in the transcript, output a bit $b \in \{0,1\}$. This algorithm is executed by $V$, and $V$ accepts if $b = 1$.
\end{itemize}
A $\Sigma$-protocol has the following properties:
\begin{enumerate}
    \item \textbf{Completeness.} $\Pi$ is complete if for any $x$, $w \in w(x)$, and any prover randomness $r^p \samplefrom \{0,1\}^\lambda$, the verifier accepts with probability 1. 
    \begin{gather*}
        Pr\left[\phi(x,a,c,z) = 1 \st a \leftarrow A(x,w;r^p); c\samplefrom \{0,1\}^\kappa; z \rightarrow Z(x,w,c;r^p)\right] = 1
    \end{gather*}
    \item \textbf{Special Soundness.} $\Pi$ is said to have special soundness if  there exists a PPT extractor $\mathcal E$, such that given any two transcripts $(a,c,z)$ and $(a,c',z')$ for statement $x$, where $c \ne c'$ and $\phi(x,a,c,z) = \phi(x,a,c',z') = 1$, an element of $w(x)$ can be computed by $\mathcal E$.
    \item \textbf{Special Honest-Verifier Zero-Knowledge (SHVZK).} $\Pi$ is SHVZK if there exists a PPT simulator $\mathcal S$, such that for any $x$, $w$, $(x,w) \in \mathcal R$, the distribution over the output $\mathcal S(1^\lambda, x, c^*)$ is indistinguishable from the distribution over transcripts produced by the interaction between $V$ and $P$ when the challenge is $c^*$.
    \begin{multline*}
        \{(a, z) \mid c^* \samplefrom \{0,1\}^\kappa; (a,z) \leftarrow \mathcal S(1^\lambda,x,c^*)\} 
        \approx_{c^*} \\
        \{(a,z) \mid r^p \samplefrom \{0,1\}^\lambda; a \leftarrow A(x,w;r^p); c^* \samplefrom \{0,1\}^\kappa; z \leftarrow Z(x,w,c^*;r^p)\}
    \end{multline*}
\end{enumerate}
\end{definition}

\begin{definition}[Witness Indistinguishable (WI)]\label{def:wi}
A $\Sigma$-protocol is witness indistinguishable over $\mathcal R$ if for any $V'$, any large enough input $x$, any $w_1,w_2 \in w(x)$, and for any fixed challenge $c^*$, the distribution over transcripts in the form $(a_1, c, z_1)$ and $(a_2,c,z_2)$ are indistinguishable, where $a_i \leftarrow A(x,w_i;r^p)$ and $z_i \leftarrow Z(x,w_i, c^*; r^p)$ for $i \in \{1,2\}$. This means that the prover reveals no information about which are the active clauses. 
\end{definition}

\begin{definition}[Informal definition of Witness Hiding (WH)]\label{def:wh}
For any $x$ that is generated with a certain probability distribution by a generator $\mathcal G$ which outputs pairs $(x,w) \in \mathcal R$, a $\Sigma$-protocol is witness hiding over $\mathcal G$, if it does not help even a cheating verifier to compute a witness for $x$ with non-negligible probability. Refer to \cite{10.1145/100216.100272} for details. WH is a weaker property than general zero-knowledge, as it only asserts that the verifier cannot learn about the witness (not asserting anything about other information). That said, it can replace zero-knowledge in many protocol constructions, as it is in most $\Sigma$-protocols.

\end{definition}



% CDS %
\subsection{Secret Sharing Scheme}
A \textit{secret sharing scheme}, is a method of distributing a secret $s$ to $n$ participants in a way that no one participant has intelligible information about the secret. This is achieved by splitting up $s$ into \textit{shares}, distributing one share to each participant in a way that \textit{only} a subset of participants can reconstruct $s$. Subsets that can reconstruct the secret are called \textit{qualified sets}. The set of all qualified sets is the secret sharing scheme's \textit{access structure}.

In this work, we are concerned with \textit{perfect} secret sharing schemes. Perfect secret sharing schemes are ones where the participants in \textit{non-qualified} sets cannot obtain any information whatsoever about the secret. Additionally, for CDS94, we require our secret sharing scheme to have a few additional properties. 

\begin{definition}[Secret Sharing Schemes for CDS94]\label{def:secret-sharing}
Let $\Pi$ be a $\Sigma$-protocol for the relationship $\mathcal R = \{(x,w)\}$. We define a \textit{secret sharing scheme} for CDS94 as $S(k)$, where $k$ is the length of $x$ in bits. $S(k)$ splits a secret $s$ into $n$ shares such that $n$ is polynomial in $k$ ($n = poly(k)$). Let $D(s)$ refer to the probability distribution of all the shares that are produced when the secret $s$ is distributed. If we consider a subset of participants $A$, then $D_A(s)$ refers to the distribution of shares that only includes participants in $A$. Since the scheme is perfect, the probability distribution $D_A(s)$ is not affected by any other subset $B$ for any non-qualified set $A$. Therefore, we can simply write $D_A$ instead of $D_A(s)$ when $A$ is non-qualified. We now define the properties we require:
    \begin{enumerate}
        \item The length of shares produced in $S(k)$ is related to $k$ through a polynomial function.
        \item The secret can be distributed and reconstructed in a time that is polynomial in $k$.
        \item With a complete set of $n$ shares and the secret $s$, it is possible to check in a time that is polynomial in $k$ whether all qualified sets of shares determine $s$ as the secret.
        \item It is always possible to complete a set of shares for participants in a non-qualified set $A$, distributed according to $D_A$, to a full set of shares that are distributed according to $D(s)$ and consistent with the secret $s$. This completion process can be done in a time that is polynomial in $k$.
        \item The probability distribution $D_A$ for any non-qualified set $A$ is such that shares for the participants in $A$ are independently and uniformly chosen.
    \end{enumerate}
\end{definition}

An $S(k)$ which fulfils properties 1-4, is known as \textit{semi-smooth}. With a semi-smooth scheme, we can compile a SHVZK $\Sigma$-protocol using Theorem 9 of CDS94. A \textit{smooth} scheme, is where $S(k)$ fulfils all 5 properties. With this we can compile a HVZK $\Sigma$-protocol using Theorem 8 of CDS94.

\subsubsection{Shamir's Secret Sharing}\label{sec:sss}
In our implementation of the CDS94 compiler, we choose to make use of Shamir's secret sharing scheme \cite{DBLP:journals/cacm/Shamir79}. 

\begin{definition}[Shamir's Secret Sharing (SSS) Scheme] \label{def:sss}
 A smooth secret sharing scheme and a \textit{threshold sharing scheme}, SSS produces qualified sets of size $d$. Any $d$ out of $n$ participants can reconstruct the secret; with $d-1$ shares and less, no information about the secret can be obtained. 

 To distribute a secret:
\begin{enumerate}
    \item We first choose a random polynomial of degree $d - 1$. The constant term of the polynomial is the secret $s$ itself.
    \item We then associate each participant with an $x \in \Z_q$ where $q$ is a large prime. For each participant, we calculate $y_i$ by evaluating the polynomial at $x_i$ for the $i$-th participant. The $i$-th share (which corresponds to the $i$-th participant) is then computed by concatenating $x_i$ with $y_i$ ($share(i) = x_i\circ y_i$).
\end{enumerate}

To reconstruct a secret:
\begin{enumerate}
    \item Any $d$ or more participants can combine their shares. Using Lagrange interpolation, a polynomial of degree $d - 1$ can be computed which passes through the $d$ points that correspond to the shares. This means that the constant term can be derived and the secret is reconstructed. 
    \item If less than $d$ participants combine their shares, they will not have enough information to reconstruct the secret. This is because the polynomial that is derived from these shares is completely random. 
\end{enumerate}
\end{definition}

\begin{definition}[Qualified Set Completion for SSS]\label{def:sss-completion}
Given the secret $s$, an unqualified set of shares $U$, and an array of indexes for the active clauses $A$, we define the algorithm $Complete(s, U, A) \rightarrow Q$, where $Q$ is a set of shares with $x$ coordinates/values corresponding to the indexes in $A$. 

\begin{enumerate}
    \item Firstly, ensure that threshold is set to $N - d + 1$, where $d$ is the number of active clauses; in other words the threshold should be $k + 1$ where $k$ is the number of inactive clauses and $|U| = k$. 
    \item Construct a lagrange polynomial of degree $k$, with the secret $s$ and shares in the set $U$. 
    \item With this polynomial, interpolate at $x = i$ for $i \in A$ to determine the $y_i$ for each $i \in A$.
    \item Return $Q \leftarrow \{share(i) =  x_i\circ y_i\ |\ i \in A\}$.
    % \item Take these as $share(c_i)$, taking the first relevant number of bits if somehow the challenge is smaller than the shares, and extrapolating with random bits if the challenge is meant to be larger.
\end{enumerate}
    
\end{definition}

Notably, we do not use SSS within CDS94 to reconstruct a secret; we use it together with a known secret to complete a qualified set using the procedure outlined in Definition \ref{def:sss-completion}.

\subsection{CDS94 Compiler}
In this paper, Cramer {\em{et al}} \cite{CDS94} presents 2 primary ways to compile $\Sigma$-protocols depending 
on the underlying choice of $\Sigma$-protocol and the secret sharing scheme. Our implementation makes use of Theorem 
8 of the paper because we choose to use Schnorr's protocol (Section \ref{sec:schnorr}) and Shamir's secret sharing 
(Section \ref{sec:sss}). More details regarding our implementation will be discussed in a further section. Now, we recall 
theorem 8 of \cite{CDS94} -- note that we alter the notation slightly from the original paper to be more consistent 
with the rest of this report.

% In our implementation, we will use Schnorr's discrete log protocol over Ristretto25519 and Shamir's secret sharing scheme to demonstrate the compilation of $\Sigma$-protocol into a $\Sigma$-protocol for the disjunction of $n$ statements. 
% We will attempt to make the implementation as general as possible to open up for the future possibility to take any $\Sigma$-protocol that suits our requirements and transform it disjunctive zero-knowledge $\Sigma$-protocol.

% \subsection{The Witness Indistinguishable (WI) compilation}

% In their paper, Cramer {\em{et al}} \cite{CDS94} presents 2 primary ways to construct a WI protocol from a $\Sigma$-protocol $\mathcal P$ (Theorem 8 and 9). 

% \begin{itemize}
%     \item Theorem 8 requires a smooth secret sharing scheme, and a HVZK $\Sigma$-protocol, while
%     \item Theorem 9 requires special honest-verifier ZK (SHVZK) with at least a semi-smooth secret sharing scheme.
% \end{itemize}

% Since, SSS is a smooth threshold secret sharing scheme (required for 8), and Schnorr's protocol is SHVZK (required for 9), we can choose either construction. 
% \textit{We will use \textbf{Theorem 8} in this project.}

\textbf{Theorem 8 \cite{CDS94}}. Given $\Pi = (A, Z, \phi)$, $\mathcal R_\Gamma$ and $\mathcal S(k)$ where

\begin{itemize}
    \item $\Pi$ is a 3-round public coin ($\Sigma$-protocol) HVZK proof of knowledge for relation $\mathcal R$.
    \item $\{\mathcal S(k)\}$ is a family of smooth secret sharing schemes.
    \item $\mathcal R_\Gamma$ is a relation where $((x_1,\ldots,x_m),(w_1,\ldots,w_m)) \in \mathcal R_\Gamma$ if and only if ($\iff$)
    \begin{itemize}
        \item all $x_i$'s are of the same length $k$, and 
        \item the set of indices $i$ for which $(x_i,w_i) \in \mathcal R$ corresponds to a qualified set for $S(k)$
    \end{itemize}
\end{itemize}

Then, there exists a $\Sigma$-protocol, $\Pi' = (A', Z', \phi')$, that is witness indistinguishable 
(Definition \ref{def:wi}) for the relation $\mathcal R_\Gamma$. The description of this protocol is outlined in 
Protocol \ref{prot:cds-compiler}. Interested readers can refer to the original paper for the proof of this theorem \cite{CDS94}. 

\begin{protocol}[label={prot:cds-compiler}]{CDS94 Compiler. A compiler for composing $n$ instances of a 
    $\Sigma$-protocol $\Pi$ into a single $\Sigma$-protocol $\Pi'$ that proves the \textbf{disjunction} 
    of these $n$ instances.}
    Let $A$ be the set of indices $i$ of the \textit{active clauses}. \\
    \textbf{Statement:} $x = x_1,\ldots, x_n$ \\
    \textbf{Witness:} $w = \{w_i\st i \in A\ \land (x_i, w_i) \in R\}$
    \begin{itemize}
        \item \textbf{First Round:} the Prover, $P$, computes $A'(x,w; r^P) \rightarrow a$ accordingly:
        \begin{itemize}
            \item For each $i \in \bar A$, run the simulator for $\mathcal P$ for the statement $x_i$ to produce the transcripts $(m_1^i, c_i, m_2^i)$.
            \item For each $i \in A$, compute $m_1^i \leftarrow A(x_i, w_i; r^P)$.
            \item Send $a \leftarrow (m_1^1, \ldots, m_1^n)$ to $V$.
        \end{itemize}
        \item \textbf{Second Round:} $V$ sends $c_0 \leftarrow \{0,1\}^\lambda$ to $P$. 
        \item \textbf{Third Round:} Prover computes $Z'(x,w,c;r^P) \rightarrow z$:
        \begin{itemize}
            \item Compute $Q \leftarrow Complete(c_0, U, A)$, where $U = \{c_i\st i \in \bar A\}$.
            \item Set $c_i = share(i)$, for each $share(i) \in Q$
            \item For $i \in A$, compute $m_2^i \leftarrow Z(x_i, w_i, c_i; r^P)$. 
            \item For $i \in \bar A$, $m_2^i$ has already been computed in the first round using the simulator.
            \item Send $z \leftarrow ((c_1, m_2^1), \ldots, (c_n,m_2^n))$ to $V$.
        \end{itemize}
        \item \textbf{Verification:} Verifier computes $\phi'(x,a,c_0,z) \rightarrow b$ as follows:
        \begin{itemize}
            \item Extract $c_i$ and $m_2^i$ from $z \leftarrow ((c_1, m_2^1), \ldots, (c_n,m_2^n))$
            \item Check that all conversations $(m_1^i, c_i, m_2^i)$ would lead to acceptance by the verifier in $\mathcal P$.
            \item Check that each $share(c_i)$ is consistent with the secret $c_0$ using the reconstruction algorithm in Definition \ref{def:sss}.
            \item If any of the checks fail return $b \leftarrow 0$; else return $b \leftarrow 1$.
        \end{itemize}
    \end{itemize}
\end{protocol}

Using this compiler, we can construct a $\Sigma$-protocol that has a communication complexity linear in the number of clauses.

% Stacking Sigmas %
% Partially Binding Vector Commitment from Discrete Log
% !TeX root = ../dissertation.tex
\subsection{Partially-Binding Vector Commitments}
In Section 5 of \cite{StackingSigmas}, Goel {\em{et al}} introduce the concept of partially-binding vector commitment schemes. 
These commitment schemes allow a Prover to make a commitment to a vector of length $l$ with $t$ binding positions; 
the remaining positions (indexes) in the vector are equivocable. 
Here, we recall the definition of these commitment schemes and refer interested readers to Figure 2 of \cite{StackingSigmas} for the construction of the general $t$-out-of-$l$ 
scheme. 

\begin{definition}[t-out-of-l Binding Vector Commitment \cite{StackingSigmas}]\label{def:comm_scheme}
  Given a message space $\mathcal M$, the security paramater $\lambda$, the length of the vector $l$, and the number of binding positions $t$: 
  the tuple of PPT algorithms $(\mathsf{Setup}, \mathsf{Gen}, \mathsf{EquivCom}, \mathsf{Equiv}, \mathsf{BindCom})$ defines a $t$-out-of-$l$ 
  partially-binding vector commitment scheme. These algorithms are defined as follows:
  \begin{itemize}
  \item $\mathsf{pp} \leftarrow \mathsf{Setup}(1^\lambda)$: Given the security parameter $\lambda$,
  the $\mathsf{Setup}$ algorithm outputs public parameters $\mathsf{pp}$ for the commitment scheme.

  \item $(\mathsf{ck},\mathsf{ek}) \leftarrow \mathsf{Gen}(\mathsf{pp},B)$: Given public parameters $\mathsf{pp}$ 
  and a set $B$, where $B \subseteq [l] \land |B| = t$. The $\mathsf{Gen}$ algorithm returns a commitment key $\mathsf{ck}$ and equivocation key $\mathsf{ek}$.

  \item $(\mathsf{com},\mathsf{aux}) \leftarrow \mathsf{EquivCom}(\mathsf{pp},\mathsf{ek},v;r)$: Given public parameter 
  $\mathsf{pp}$, equivocation key $\mathsf{ek}$, vector $v$ with length $l$, and randomness $r$ as input, the $\mathsf{EquivCom}$ algorithm returns a 
  partially-binding commitment $\mathsf{com}$ and auxiliary equivocation information $\mathsf{aux}$. This means that the vector $v$ can be equivocated in 
  equivocable locations and $\mathsf{com}$ will be the same.  

  \item $r \leftarrow \mathsf{Equiv}(\mathsf{pp},\mathsf{ek},v,v',\mathsf{aux})$: Given public parameters $\mathsf{pp}$, 
  equivocation key $\mathsf{ek}$, original message vector $v$ and updated message vector $v'$ 
  where $\forall i \in B: v_i = v'_i$, and auxiliary equivocation information $\mathsf{aux}$. $\mathsf{Equiv}$ returns equivocation 
  randomness $r$.

  \item $\mathsf{com} \leftarrow \mathsf{BindCom}(\mathsf{pp}, \mathsf{ck}, v; r)$: Given public parameters 
  $\mathsf{pp}$, commitment key $\mathsf{ck}$, vector $v$, and randomness $r$ as input, $\mathsf{BindCom}$ returns a commitment 
  $\mathsf{com}$. This algorithm plays a similar role to that of $\mathsf{Open}$ in a typical commitment scheme.
  \end{itemize}

  Partially-binding vector commitment schemes have the following properties
  
  \begin{enumerate}
    \item \textbf{(Perfect) Hiding.} Suppose there are two vectors $\mathbf{v}^{(1)}, \mathbf{v}^{(2)} \in \mathcal M^l$ and two corresponding sets of 
    binding positions $B^{(1)}, B^{(2)} \in {(l) \choose t}$. A perfectly hiding commitment scheme is one where the commitment key $ck$ and commitment $com$ 
    produced by any two sets of binding position and original vectors $(\mathbf{B}^{(1)}, \mathbf{v}^{(1)})$ and $(\mathbf{B}^{(2)}, \mathbf{v}^{(2)})$ are 
    indistinguishable from each other when they are equivocated to the same vector $\mathbf{v}'$ (provided that $\mathbf{v}'$ is a valid equivocation for 
    both vectors). 

    \item \textbf{(Computational) Partial Binding.} A malicious user (that generates $\mathsf{ck}$ itself) is not able to cheat the system by equivocating 
    on more than $l - t$ positions.

    \item \textbf{Partial Equivocation.} It is always possible to equivocate to any vector $\mathbf{v}'$ as long as $\forall i \in B: v_i = v'_i$. 
  \end{enumerate}

  There is an efficiency requirement which requires the size of the commitment to be independent from the size of the messages 
  in the vector. This can be achieved by using a collision resistant hash function $H$. 
  For more details on each property, we refer the reader to Definition 4 of \cite{StackingSigmas}. 
\end{definition}


\subsection{Half-Bindings \& Q-Bindings}\label{sec:qbinding}
In this work, we make use of the generic 1-out-of-$2^q$ construction provided in Section 5.3 of \cite{StackingSigmas} 
for our implementation of the Stacking Sigmas compiler. We use this 
particular construction as it allows us to yield commitments that are logarthmic in size (in bytes) to the original 
vector of messages (\textbf{Theorem 2 \& Corollary 1} in \cite{StackingSigmas}).
A core ingredient in this construction is a 1-out-of-2 partially-binding vector commitment scheme, which is used recursively 
to form a \textit{binary} "tree of commitments". 
For brevity, we refer to the generic construction as "q-bindings" and the 1-out-of-2 construction as "half-bindings".
The leaves of this tree are the original messages in our vector; intermediate nodes are the resulting commitments to the 
nodes' children using half-bindings. 
Essentially, intermediate commitments are regarded as messages as well, and are commited to recursively until there is only 
1 root node -- the final commitment. 

While the source material provides the general $t$-out-of-$l$ construction for partially-binding vector commitments, 
it does not explicitly provide that for half-bindings. 
In Figure \ref{fig:half-binding}, we provide the construction for half-bindings derived from the general $t$-out-of-$l$ 
construction. 
The proof of correctness for this construction is trivial as it can be easily seen to be a specific case of the general 
construction when $t = 1$ and $l = 2$. 

\begin{figure}[h]
  \caption{Construction of a 1-out-of-2 partially-binding vector commitment scheme.}
  \label{fig:half-binding}
\end{figure}
\begin{breakablefig}
    \begin{minipage}{0.45\linewidth}
      \vspace{-4em}
      \underline{$pp \leftarrow$ Setup$(1^\lambda)$:}
      \begin{enumerate}
        \item $\G \leftarrow $ GenGroup$(1^\lambda)$; $g_0, h \samplefrom \G$
        \item $\textbf{return } pp \leftarrow (\G, g_0, h)$
      \end{enumerate}  
    \end{minipage}
    \begin{minipage}{0.5\linewidth}
      \underline{$(com, aux) \leftarrow$ EquivCom$(pp, ek, m_1, m_2)$:}
      \begin{enumerate}
        \item Extract $ck$ from $ek$
        \item $r \samplefrom \Z_{|\G|}^2$
        \item com $\leftarrow$ BindCom$(pp,ck,m_1,m_2,r)$
        \item \textbf{return} $(\text{com}, r)$
      \end{enumerate}  
    \end{minipage}

    \underline{$(ck, ek) \leftarrow$ Gen$(pp, B)$:}
    \begin{enumerate}
      \item Let $E = \{1,2\} \setminus B$ be the set of equivocal indexes. $|B| = 1$.
      \item Generate trapdoor $td$ for $i \in E: td \samplefrom \Z_{|\G|}, g_i \leftarrow h \cdot td$
      \item Interpolate the first element: $g_1 = \begin{cases}
       g_2 - g_0 & \text{if $2 \in E$} \\
       g_1 & \text{if $1 \in E$}
      \end{cases}$
      \item $ck \leftarrow g_1$
      \item $ek \leftarrow (B, td, ck)$
      \item \textbf{return} $(ck, ek)$
    \end{enumerate}

    \underline{$com \leftarrow$ BindCom$(pp,ck, m_1,m_2, r)$:}
    \begin{enumerate}
      \item Interpolate $g_2$: $g_2 = g_1 + g_0$
      \item $(r_1, r_2) \leftarrow r$
      \item Commit individually $\textbf{for } j \in \{1,2\}: com_j \leftarrow h\cdot r_j + g_j \cdot m_j \in \G$
      \item \textbf{return} $(com_1,com_2)$
    \end{enumerate}

    \underline{$r \leftarrow$ Equiv$(pp, ek, m_1, m_2, m_1', m_2', aux)$:}
    \begin{enumerate}
      \item Extract $B$ and $td$ from $ek$; Let $E = \{1,2\} \setminus B$.
      \item Parse $aux = (r_1, r_2) \in \Z_{|\G|}^2$
      \item Interpolate $g_2 = g_1 + g_0$
      \item for $i \in B: r'_i \leftarrow r_i$
      \item for $i \in E: r'_i \leftarrow r_i - td \cdot (m_i' - m_i)$
      \item \textbf{return} $r' \leftarrow (r_1', r_2')$
    \end{enumerate}
\end{breakablefig}

  \paragraph*{Communication Complexity.} Observing the construction of half-bindings, we can see that the size (in bytes) of the outputs of each method 
  is directly related to the choice of the group $\G$ and the number of bytes required to represent a group element $g \in \G$ and scalars $r \in \Z_{|\G|}$. 
  
  % For ease of reference, we also include the q-binding scheme (1-out-of-$2^q$ parital-binding vector commitment
  % scheme) in Figure \ref{fig:q-binding}.
  
  % \begin{figure}[h]
  %   \caption{Construction of a 1-out-of-$2^q$ partially-binding vector commitment scheme.}
  %   \label{fig:q-binding}
  % \end{figure}
  % \begin{breakablefig}
  %     \begin{minipage}{0.45\linewidth}
  %       \vspace{-2em}
  %       \underline{$pp \leftarrow$ Setup$(1^\lambda)$:}
  %       \begin{enumerate}
  %         \item $pp_A \leftarrow $ PBComm$_A$.Setup$(1^\lambda)$
  %         \item $pp_B \leftarrow $ PBComm$_B$.Setup$(1^\lambda)$
  %         \item \textbf{return} $(pp_A, pp_B)$
  %       \end{enumerate}  
  %     \end{minipage}
  %     \begin{minipage}{0.5\linewidth}
  %       \underline{$(ck, ek) \leftarrow $ Gen$(pp = (pp_A, pp_B), B = \{i\})$:}
  %       \begin{enumerate}
  %         \item Compute $i_A = i \bmod \ell_A, i_B = \lfloor i / \ell_A \rfloor$
  %       \end{enumerate}  
  %     \end{minipage}
  
  %     \underline{$(ck, ek) \leftarrow$ Gen$(pp, B)$:}
  %     \begin{enumerate}
  %       \item $i_A = i \bmod \ell_A$, $i_B = \lfloor i/\ell_A \rfloor$ {\footnotesize // Compute indices}
  %       \item $\mathbf{v}^{(A)} \leftarrow \mathbf{0}$, $v(A) \leftarrow v_i \cdot \delta_{i_A}$ {\footnotesize // Form a length 
  %       $l_A$ vector $v_A$, which is zero everywhere except at $i_A$ where it is $v_i$;}
  %       \item $(com_A, aux_A) \leftarrow \text{PBCommA.EquivCom}(pp_A, ek_A, v(A))$;
  %       \item Form a length $l_B$ vector $v_B$, which is zero everywhere except at $i_B$ where it is $com_A$;
  %       \item $v(B) \leftarrow 0$, $v(B) \leftarrow com_A \cdot \delta_{i_B}$;
  %       \item $(com_B, aux_B) \leftarrow \text{PBCommB.EquivCom}(pp_B, ek_B, v(B))$ \texttt{Return the root/outer commitment}
  %       \item \textbf{return} $(com_B, (aux_A, aux_B))$;
  %     \end{enumerate}
  
  %     \underline{$com \leftarrow$ BindCom$(pp,ck, m_1,m_2, r)$:}
  %     \begin{enumerate}
  %       \item Interpolate $g_2$: $g_2 = g_1 + g_0$
  %       \item $(r_1, r_2) \leftarrow r$
  %       \item Commit individually $\textbf{for } j \in \{1,2\}: com_j \leftarrow h\cdot r_j + g_j \cdot m_j \in \G$
  %       \item \textbf{return} $(com_1,com_2)$
  %     \end{enumerate}
  
  %     \underline{$r \leftarrow$ Equiv$(pp, ek, m_1, m_2, m_1', m_2', aux)$:}
  %     \begin{enumerate}
  %       \item Extract $B$ and $td$ from $ek$; Let $E = \{1,2\} \setminus B$.
  %       \item Parse $aux = (r_1, r_2) \in \Z_{|\G|}^2$
  %       \item Interpolate $g_2 = g_1 + g_0$
  %       \item for $i \in B: r'_i \leftarrow r_i$
  %       \item for $i \in E: r'_i \leftarrow r_i - td \cdot (m_i' - m_i)$
  %       \item \textbf{return} $r' \leftarrow (r_1', r_2')$
  %     \end{enumerate}
  % \end{breakablefig}


% Section on stacking sigmas - talk about how stacking works

% Speed Stacking %
\section{Homomorphisms}
A homomorphism between two algebraic objects $A$ and $B$ is a function $f: A \rightarrow B$ which preserves their algebraic structure. 
\begin{enumerate}
    \item If elements in $A$ satisfy some algebraic equation involving addition or multiplication, their images in $B$ also satisfy the same algebraic equation
    \item The details of the definitions of homomorphisms depend on algebraic structures of $A$ and $B$
\end{enumerate}

\begin{definition}[Group Homomorphisms]
Suppose we have arbitrary groups $A$ and $B$, with the group operators $\circ_A$ and $\circ_B$ respectively. A group homomorphism is a function $f: A \rightarrow B$ such that $f(a_1 \circ_A a_2) = f(a_1) \circ_B f(a_2)$ for all $a_1,a_2 \in A$
\end{definition}
% section on compressed sigma protocols (compression mechanism)
% section on speed stacking compiler 

