% !TeX root = dissertation.tex
\documentclass[a4paper,fleqn,10pt]{report} %report

%%%%%%%%%%%%%%%%%%%%

% !TeX root = ../dissertation/dissertation.tex
\usepackage[
    textwidth=450pt, 
    textheight=700pt, 
    % margins={1in,1in,1in,1in}
    ]{geometry}
\usepackage[utf8]{inputenc}
\usepackage[UKenglish]{babel}
\usepackage[UKenglish]{isodate}
\usepackage{amsmath}
\usepackage{array}
\usepackage{amsfonts}
\usepackage{amssymb}
\usepackage{amsthm}
\usepackage{graphicx}
\usepackage{chngpage}
\usepackage{calc}
\PassOptionsToPackage{hyphens}{url}
\usepackage{hyperref}
\usepackage[nameinlink]{cleveref}
\usepackage{fancyhdr}
\usepackage{titletoc}
\usepackage[explicit]{titlesec}
\usepackage[dvipsnames]{xcolor}
\usepackage[sc]{mathpazo}
\linespread{1.05}
\usepackage[T1]{fontenc}
\usepackage{minted}
\usepackage{csquotes}
\usepackage{booktabs}
\usepackage[most]{tcolorbox}
% \usepackage{ulem}
\usepackage{listings}
\usepackage[style=alphabetic, sorting=anyvt]{biblatex}
\addbibresource{../common/ref.bib}

\lstset{
    framerule=1pt,
    frame=tb,
    emphstyle={\small\ttfamily\bfseries\color{Orange}},
    numbers=left,
    numberstyle= \tiny\color{black},
    basicstyle = \small\ttfamily,
    keywordstyle    = \bfseries\color{BrickRed},
    identifierstyle = \bfseries\color{black},
    stringstyle     = \bfseries\color{ForestGreen},
    commentstyle    = \bfseries\color{Violet},
    breaklines      =   true,
    columns         =   fixed,
    basewidth       =   .5em,
    backgroundcolor=\color{Gray!5},
    tabsize=2,
    showspaces=false,
    showstringspaces=false,
}
\newtcolorbox[auto counter]{problem}[2][]{%
    enhanced,
    % breakable,
    colback=white,
    colbacktitle=white,
    coltitle=black,
    boxrule=.6pt,
    titlerule=.2pt,
    toptitle=3pt,
    % lefttitle=1pt,
    % righttitle=1pt,
    bottomtitle=3pt,
    title={#2}, %Problem~\thetcbcounter
    #1
}
\newtcolorbox[auto counter]{protocol}[2][]{%
    enhanced,
    breakable,
    colback=white,
    colbacktitle=white,
    coltitle=black,
    boxrule=.6pt,
    titlerule=.2pt,
    toptitle=3pt,
    % lefttitle=1pt,
    % righttitle=1pt,
    bottomtitle=3pt,
    title={\textbf{Protocol~\thetcbcounter}. #2}, %
    #1
}
\newtcolorbox[]{breakablefig}[1][]{%
    enhanced,
    breakable,
    colback=white,
    colbacktitle=white,
    coltitle=black,
    boxrule=.6pt,
    #1
}


\graphicspath{{./assets/}}

\hypersetup{
	colorlinks=true,
	linkcolor=Brown,
	urlcolor=blue,
	citecolor=Blue,
}

\setlength{\parindent}{0mm}
\setlength{\parskip}{\medskipamount}
\renewcommand\baselinestretch{1.2}

\cleanlookdateon

\makeatletter
\newcommand{\@assignment}[0]{Assignment}
\newcommand{\assignment}[1]{\renewcommand{\@assignment}{#1}}
\newcommand{\@supervisor}[0]{}
\newcommand{\supervisor}[1]{\renewcommand{\@supervisor}{#1}}
\newcommand{\@yearofstudy}[0]{}
\newcommand{\yearofstudy}[1]{\renewcommand{\@yearofstudy}{#1}}
\makeatletter

\newtoggle{IsDissertation}

\newcommand{\Z}[0]{\mathbb{Z}}
\newcommand{\F}[0]{\mathbb{F}}
\newcommand{\G}[0]{\mathbb{G}}

\theoremstyle{definition}
\newtheorem{definition}{Definition}

\newcommand{\samplefrom}[0]{\xleftarrow{\$}}
\newcommand{\setwith}[1]{\{#1\}}
\newcommand{\randselect}[0]{\samplefrom_s}
\newcommand{\st}{\ \vert\ }
\newcommand{\View}{\text{View}}
\newcommand{\ViewPV}{\View_V(P(x,w),V(x))}
\newcommand{\verify}{\overset {?} {=}}
% Tables
\newcommand{\rowheight}{0.7em}
\newcolumntype{P}[1]{>{\arraybackslash}p{#1}}
\newcolumntype{M}[1]{>{\arraybackslash}m{#1}}
\newcolumntype{C}[1]{>{\centering\arraybackslash}p{#1}}
\newcolumntype{R}[1]{>{\raggedright\arraybackslash}p{#1}}
\newcolumntype{A}[1]{>{\centering}M{#1}}
%%%%%%%%%%%%%%%%%%%%%%%%%%%%%%%%%%%%%%%%%%%%%%%%%%%%%%%%%%%%%%%%%%%%%%%%%%%%%%%
%% Project-specific configuration
%%%%%%%%%%%%%%%%%%%%%%%%%%%%%%%%%%%%%%%%%%%%%%%%%%%%%%%%%%%%%%%%%%%%%%%%%%%%%%%

\author{Justin Tan}
\title{Disjunctive Zero Knowledge}
\supervisor{Nicholas Spooner}
\yearofstudy{3\textsuperscript{rd}}

%%%%%%%%%%%%%%%%%%%%%%%%%%%%%%%%%%%%%%%%%%%%%%%%%%%%%%%%%%%%%%%%%%%%%%%%%%%%%%%


\toggletrue{IsDissertation}

%%%%%%%%%%%%%%%%%%%%

\pagestyle{plain}
\renewcommand{\headrulewidth}{0.0pt} % Set it so that header does not have divider line

\makeatletter
\fancypagestyle{plain}{
    \fancyhf{}
    % 	Settings for twoside document
    % 	\fancyhead[LE]{\thepage}
    % 	\fancyhead[RE]{\textit{\@author}}
    % 	\fancyhead[RO]{\thepage}
    % 	\fancyhead[LO]{\textit{\@title}}
    \fancyhead[R]{\thepage}
    \fancyhead[L]{\textit{\@title}}
    \setlength{\headheight}{0.5in}
}
\makeatother

%%%%%%%%%%%%%%%%%%%%

\begin{document}
\normalem

\input{../common/titlepage.tex}

\pagestyle{plain}

\newpage
\begin{abstract}
    % Your abstract goes here. This should be about 2-3 paragraphs summarising the motivation for 
    % your project and the main outcomes (software, results, etc.) of your project. 
    Zero-knowledge proofs are protocols that allow a prover to prove the validity of a 
    statement to a verifier without revealing any information about the statement. There has 
    been a long line of research into zero-knowledge proofs -- 
    in particular, zero-knowledge proofs for disjunctive statements have been a core target. 
    Disjunctive statements are made up of a set of clauses joined by logical OR operators, 
    and the goal of the prover is to prove that at least one of the clauses is true. 
    
    In this work we focus on one approach for constructing disjunctive zero-knowledge proofs: 
    general \emph{compilers} for zero-knowledge proofs. This approach has been explored by in early work by 
    Cramer and Damg{\aa}rd
    in \textquote{Proofs of Partial Knowledge and Simplified Design of Witness Hiding Protocols}, 
    and more recently by Goel {\em et al.} in \textquote{Stacking Sigmas: A Framework to Compose 
    $\Sigma$-Protocols for Disjunctions}. We implement the compilers 
    proposed in these two papers, benchmark their performance, and compare the results. 
    There has been a lack of implementations of these compilers, and this project serves to fill 
    in this gap, as well as to provide a better understanding of their performances, and 
    lay the groundwork for future work to benchmark against past work. 

    Our results validate that Goel {\em et al.}'s compiler, Stacking Sigmas, outperforms Cramer and 
    Damg{\aa}rd's compiler, CDS94, in terms of communication complexity as the number of clauses increase. 
    We also show that the choice of secret sharing scheme influences the 
    computational complexity of the CDS94 protocol greatly. 
    % Additionally, we show that by comparing the raw size of the proof, 
    % Stacking Sigmas provides significant savings over CDS94. 
\end{abstract}

\newpage
\tableofcontents

\newpage

\chapter{Introduction}
% !TeX root = dissertation.tex
A zero-knowledge proof \cite{GMR85} is a protocol with two parties, the prover and  verifier, and an NP statement that is publicly known to both parties. The goal of the prover is to prove to the verifier that this statement is true without revealing any extra information. In other words, the verifier does not learn anything apart from the validity of the statement. Early research in this area showed that all languages in NP have 
zero-knowledge proof systems \cite{DBLP:conf/focs/GoldreichMW86}. This served as a catalyst for years of research into developing 
more efficient zero-knowledge proofs for various use cases. Today, efficient zero-knowledge proofs are being used 
in practice, with several use cases such as e-voting and secure decentralized systems \cite{evoting, zcash}.

It is often desirable to have a zero-knowledge proof for a \textit{disjunctive statement}.
A disjunctive statement is an NP 
statement with a set of clauses that are connected with logical ORs:

$$
clause_1 \lor clause_2 \lor \cdots \lor clause_n
$$

Such zero-knowledge proofs fall under the category of "disjunctive zero-knowledge": the goal is for the Prover 
to show that they have knowledge of at least 1 of the clauses while revealing no additional information. 
In other words, the Prover shows that at least 1 of the clauses is true and does not reveal 
information such as the location of these clauses or the "answer" to these clauses. 
We can think of each clause as a statement that has an "answer" to it -- formally this "answer" 
is known as a \textit{witness} for the statement. Clauses that the Prover has 
knowledge of are called the \textit{active clauses}. 
Disjunctive statements have very useful properties that occur commonly in practice, and adding zero-knowledge can be very beneficial to  
systems that desire both privacy and verifiability. For example, it can be used to prove an 
individual's membership to a particular group while revealing nothing about the individual's identity. 
They can also be used to show that a bug exists in a verifier's code base \cite{StackedGF}. Consequently, there has been broad research interest in determining how to construct disjunctive zero-knowledge proofs, and how to do so efficiently: saving on communication complexity (the size, in bytes, of the messages between prover and verifier) and computational complexity.  

One approach is to alter the underlying zero-knowledge protocols manually so that they are compatible with disjunctive statements. This approach has been studied by recent work, and it has shown that disjunctive zero-knowledge proofs can be attained with a communication complexity that is sub-linear in the number of clauses \cite{StackedGF,attema}. Unfortunately, this approach does not generalise well and only works for the individual protocols
they target. An alternative approach to this is to develop generic compilers for disjunctive zero-knowledge \cite{CDS94,StackingSigmas}. These 
compilers can target a large class of zero-knowledge proofs known as $\Sigma$-protocols (also known as 
3-round public coin proofs of knowledge).

Research targeting \textit{disjunctive zero-knowledge compilers} began with Cramer, Damg{\aa}rd, and Schoenmakers \cite{CDS94}. 
They proposed a generic compiler to compose multiple instances of $\Sigma$-protocols into a disjunctive zero-knowledge proof. According to their findings, the proof has a communication complexity that grows linearly with the number of clauses.
In a more recent development, Goel {\em et al.} \cite{StackingSigmas} go a step further by introducing a generic compiler that focuses on a large subset of $\Sigma$-protocols and achieves a communication complexity that is sub-linear in the number of clauses.

\section{Our Contributions}

Despite a considerable amount of research, there are few noteworthy instances of these compilers being applied in practical scenarios\footnote{Hall-Andersen \cite{MHAStackSig} provides a benchmark of applying the compiler from
\cite{StackingSigmas} to Schnorr's discrete log protocol \cite{Schnorr}.}. 
In this work, we seek to fill in this gap and have implemented
the CDS94 compiler \cite{CDS94} and the Stacking Sigmas compiler from \cite{StackingSigmas}. 
We also benchmark these protocols to measure and compare the difference in their performances. 
From our analysis of the collected data, we provide insights into the trade-offs between these 
protocols and suggest when they might be the most useful. 

As a project extension, we are also working on an implementation of the compiler 
introduced in "Speed-Stacking: Fast Sublinear Zero-Knowledge Proofs for Disjunctions" 
\cite{SpeedStacking}. This compiler builds on the work in \cite{StackingSigmas} and 
explores how sublinear-sized zero-knowledge proofs can be compiled into a 
sublinear-sized disjunctive zero-knowledge proof, with a \textit{sublinear} 
running time. Specifically, we are working on applying this compiler to Compressed 
$\Sigma$-protocols \cite{attema}. This is still a work in progress.

Our aim is for our work to serve as a basis for future research to compare newly developed compilers with the existing ones, which could result in further advancements with a wider impact on cryptographic systems that rely on this particular use case. 

\section{Related work}
\label{sec:related_work}
In their contribution to \cite{StackingSigmas}, Hall-Andersen introduces an implementation of the Stacking Sigmas compiler \cite{MHAStackSig}. Their approach involves using the Stacking Sigmas compiler in combination with Schnorr over Ristretto25519 \cite{ristretto_web} to create efficient ring signatures. In our work, we include a comparison of our implementation of the compilers (both \cite{CDS94} and \cite{StackingSigmas}) to Hall-Andersen's implementation. 
The key difference between our implementations is that we use a newer commitment scheme (present in the updated and latest version of 
\cite{StackingSigmas}). Additionally, we also seek to improve the usability of the compiler by providing a more user-friendly API. 



% Stacked Garbling implementation
% elucidation of the problem and objectives of the project

\chapter{Background}\label{sec:background}
% !TeX root = dissertation.tex
Before discussing our implementation and results, we introduce the necessary background concepts for this project. 
We first formalise the notation we use in this report and highlight broad concepts that are relevant to the entire project. 
These concepts include zero-knowledge, disjunctive zero-knowledge, and $\Sigma$-protocols. 
After which, we split the remaining section into subsections: 
each for one of the compilers we implement \cite{CDS94, StackingSigmas}%, SpeedStacking}.

\section{Notation \& Terminology}\label{sec:notation}
\begin{table}[h]
  \centering
  \label{tab:notation}
  \caption{Notation used in this report}
  \begin{tabular}{|A{0.1\linewidth}|M{0.8\linewidth}|}
    \hline
    Symbol & Details \\\hline
    $\lambda$ & Computational security parameter. This refers to the ability of a cryptographic system to remain secure against an adversary who is computationally bounded. \\\hline
    $\kappa$ & Statistical security parameter. This pertains to the security provided by negligible statistical probability. \\\hline
    $\verify$ & Boolean assertion. \\\hline
    $\|$ & Bit concatenation: $0000 \| 1111 = 00001111$ \\\hline
    $\samplefrom$ & Sampling from a distribution. $x \samplefrom \mathcal D$ means that we sample $x$ from 
    the distribution "$\mathcal D$". \\
    \hline
    $[l]$ & The range of integers from 1 to $l$ \\
    \hline
  \end{tabular}
\end{table}

Throughout this paper, when "randomness" is mentioned in the definition of certain protocols, it usually refers to random bits generated by random number generators (RNG).

\subsection{Disjunctive Zero-Knowledge}

\begin{definition}[NP Relations]
Let $R \subseteq \{0,1\}^* \times \{0,1\}^*$ be a binary relation. Then $w(x) = \{w \mid (x,w) \in R\}$ and $L_R = \{x \mid \exists w, (x,w) \in R\}$. If $(x,w) \in R$, we say that $w$ is a witness for $x$. $R$ is an NP-relation if it fulfils the following two properties:
\begin{enumerate}
    \item \textbf{Polynomially bounded.} We say that $R$ is \textit{polynomially bounded} if there exists a polynomial $p$ such that $|w| \le p(|x|), \forall (x,w) \in R$. 
    \item \textbf{Polynomial-time verification.} There exists a polynomial-time algorithm for deciding membership in $R$. Consequently, $L_R \in NP$. 
\end{enumerate}

Throughout this document, we will use $\mathcal R$ to refer to a binary NP-relation.
\end{definition}

\begin{definition}[Zero-Knowledge]\label{def:zeroknowledge}
A proof or argument system $(P,V)$ is zero-knowledge over $\mathcal R$ if there exists a \textit{probabilistic polynomial time} (PPT) simulator $\mathcal S$, such that for all $(x,w) \in R$, the distribution of the output $\mathcal S(1^\lambda, x)$ of the simulator is indistinguishable from the distribution over the conversations generated by the interaction of $P$ and $V$, from the perspective of $V$; we denote this with $\ViewPV$. Conversations between $P$ and $V$ are ordered triples of the form $(a,c,z)$, and are known as \textit{transcripts}.
\end{definition}

Intuitively, this means that $V$ should not learn anything from the transcripts  with $P$ that they cannot already learn on their own by running the simulator $\mathcal S$; they learn nothing new.

\begin{definition}[Disjunctive Zero-Knowledge]
Given a sequence of statements $(x_1,x_2,\ldots, x_l)$, a \textit{disjunctive zero-knowledge proof} is a protocol to prove in zero-knowledge that $x_1 \in \mathcal L_1 \lor x_2 \in \mathcal L_2 \lor \ldots \lor x_l \in \mathcal L_l$, for NP languages $\mathcal L_i$. We term clauses for which the prover has a witness for as \textit{active} clauses. 
\end{definition}

\begin{definition}[Honest-Verifier Zero-Knowledge (HVZK)]
A proof system is HVZK if it only requires that $\mathcal S$ is an efficient simulator 
for honest (non-malicious) probabilistic polynomial time verifier strategies $V$. If $V$ is malicious then the distribution 
of the output $\mathcal S(x)$ will no longer be indistinguishable from $\ViewPV$ for such proof systems. 
\end{definition}


We model $\Sigma$-protocols with an interface.
Firstly, we design a set of generic types associated with the interface which will fit into 
the definition of the methods in the interface. These types are:
\begin{itemize}
  \item \texttt{Statement}: Public information about the protocol.
  \item \texttt{Witness}: Private information about the protocol (only known to Prover).
  \item \texttt{MessageA}: The first message of the protocol, sent from prover to verifier.
  \item \texttt{Challenge}: The challenge sent by the verifier to the prover.
  \item \texttt{MessageZ}: The third message of the protocol, sent from prover to verifier. 
  \item \texttt{State}: For most $\Sigma$-protocols, there is a particular state associated 
  to the execution of the first message. This state is often used by the prover again in 
  the third message but must not be sent to the verifier.
\end{itemize}

In particular, the design of the generic \texttt{State} type is worth discussing in more detail.
It is not explicitly mentioned in the formal definition of $\Sigma$-protocols, as it is 
assumed that the prover is able to track and store the private values that they obtain and 
require in the protocol. In the implementation, this has to be explicitly modelled and 
we do this using a functional programming approach, instead of an object-oriented
programming (OOP) approach. 
Firstly, this is because the functional approach is simpler, as we do not need to 
implement separate interfaces for the prover and verifier and subsequently include them as fields
within the \texttt{SigmaProtocol} interface. Secondly, the functional approach is more flexible,
as it does not restrict the user to a particular way of implementing the prover or verifier.
The functional approach simply provides two values in the first message (the state and the 
actual message), and the user of our interface can decide how to use these values.

Now, we present the methods that are associated with our $\Sigma$-protocol interface. 
These methods are: 
\begin{itemize}
  \item \texttt{first}: the first message of the protocol. This models algorithm 
  $a \leftarrow A(x, w; r^P)$ 
  in Definition \ref{def:sigma}.
  \item \texttt{second}: the second message of the protocol. $c \leftarrow \{0,1\}^\kappa$.
  \item \texttt{third}: the third message of the protocol. $z \leftarrow Z(x, w, c; r^P)$. 
  \item \texttt{verify}: verifies the transcript. $b \leftarrow \phi(x, a, c, z)$.
\end{itemize}

In Appendix \ref{code:SigmaProtocol} we provide a code snippet of these methods within 
the interface, which outlines which generic types are given as input and which are 
returned as output. Referring to the code snippet, readers will observe that our 
methods almost model the algorithms in Definition \ref{def:sigma} exactly in terms of 
input and output. The only difference is with the \texttt{State} type that we have already 
discussed.

\subsection{Schnorr's Protocol}
As a $\Sigma$-protocol, our first step in our implementation for Schnorr's protocol is 
to create concrete definitions for 
the methods and generic types of the \texttt{SigmaProtocol} interface. We use the following 
concrete types in the interface: 
\begin{itemize}
  \item \texttt{Statement}: \texttt{Schnorr} -- a type that simply contains the "public key"
  (which is the group element $H = x \cdot G$) as a field. We use the \texttt{RistrettoPoint} 
  type from \texttt{curve25519-dalek} to represent the public key, and it is essentially a 
  point on an Edward's curve. 
  \item \texttt{Witness}: \texttt{Scalar} -- also from the \texttt{curve22519-dalek} library. It 
  represents elements of the prime field in the Ristretto group. 
  \item \texttt{MessageA}: \texttt{CompressedRistretto} -- essentially the same as the 
  \texttt{RistrettoPoint} type, but differs in its representation. \texttt{CompressedRistretto}
  is a compressed representation of the point, which is more efficient to store and
  transmit.
  \item \texttt{Challenge}: \texttt{Scalar}
  \item \texttt{MessageZ}: \texttt{Scalar}
  \item \texttt{State}: \texttt{Scalar} 
\end{itemize}

The key takeaway is that we assign concrete types to the associated generic types provided in 
the interface for our implementation of Schnorr. The methods in the interface are then 
defined according to our definition of Schnorr in Definition \ref{def:schnorr}. 
\section{Schnorr's Identification Protocol}\label{sec:schnorr}
An \textit{identification scheme} involves two participants, a prover $P$ and a verifier $V$, whose goal is for $P$ to prove their identity to $V$. Specifically, $V$ should be persuaded that $P$ has access to the private key that corresponds to $P$'s public key. An example of an identification scheme is the commonly used password authentication protocol.

Schnorr's protocol, as described in \cite{Schnorr}, is a method of proving identity in which the prover, denoted by $P$, demonstrates knowledge of the discrete log $w$ of a group element $H$ in a finite abelian group $(\G, +)$ with $+$ as the binary operator\footnote{We choose to define $\G$ with the $+$ as our implementation uses elliptic curves which are finite abelian groups over addition. The equivalent definition with a group defined over the multiplicative operator is $h = g^w$}. Specifically, $H = w \cdot G$ for some generator $G \in \G$. The protocol relies on the discrete log assumption \cite{Diffie1976NewDI}, which states that computing $w$ given $H$ and $G$ is currently computationally infeasible, assuming that $\G$ is large enough. However, proving that $H = w \cdot G$ given $w$ and $G$ can be done efficiently. 

Our implementation of the protocol will use the Ristretto group \cite{ristretto_web}, which is constructed from a family of elliptic curves known as Edwards curves \cite{Edwards2007} and is of prime order. 

\begin{definition}[Schnorr's Protocol \cite{Schnorr}]\label{def:schnorr}
    Let $\G$ be an elliptic curve over the finite field $\F_q$ and let $E(\F_q)$ denote the group of points on $\G$. Suppose that the prover $P$ and verifier $V$ agree on $\G$ and $\F_q$, and let $H \in E(\F_q)$ be the public key that corresponds to the private key $w$, where $H = w \cdot G$. The prover convinces the verifier that they possess knowledge of the private key $w$ by following Protocol \ref{prot:schnorr}.
\end{definition}

\begin{protocol}[label={prot:schnorr}]{Schnorr's protocol. Public information: $\G = E(\F_q),\ H \in \G$. Statement to prove: $H = w \cdot G$} 
    \vspace{-0.5cm}
    \begin{align*}
        &\text{Prover}(w_p)& 
        &&
        &\text{Verifier}& 
        \\
        &r \samplefrom \F_q,\ \mathcal A = r \cdot G&
        &\overset  {\mathcal A} {\xrightarrow{\hspace{3cm}}}&
        && 
        \\
        &&
        &&
        &c \samplefrom \F_q&
        \\
        &&
        &\overset c {\xleftarrow{\hspace{3cm}}}&
        &&
        \\
        &z \leftarrow cw_p + r&
        &&
        && 
        \\
        &&
        &\overset {z} {\xrightarrow{\hspace{3cm}}}&
        &z \cdot G \verify \mathcal A + c \cdot H&
    \end{align*}
    \tcblower
    Communication between Prover and Verifier in Schnorr's protocol. Solving for the final equation shows  that $V$ accepts if and only if $w = w_P$.
    $$
    \begin{gathered}
    z \cdot G = \mathcal A + c \cdot H \iff
    r \cdot G + c \cdot w_P \cdot G  = r \cdot G + c \cdot w \cdot G \iff
    w  =  w_P
    \end{gathered}
    $$
\end{protocol}



\paragraph{Simulator $\mathcal S$ for Schnorr's protocol.} From our definition of a Zero Knowledge protocol in Definition 
\ref{def:zeroknowledge}, we shared that every zero-knowledge proof system has a PPT algorithm called a simulator. We can construct this simulator by running the protocol "in reverse" shown in Figure \ref{fig:schnorr-sim}.

\begin{figure}[h]
    \centering
    \begin{problem}[width=\linewidth/2]{Let $\mathcal S(H)$ be our simulator}
        $z \samplefrom \F_q$ 
        
        $c \samplefrom \F_q$
        
        $\mathcal A = z \cdot G - c \cdot H$
        \tcblower
        \textbf{output} $(\mathcal A,c,z)$
    \end{problem}
    \caption{A Simulator for Schnorr}
    \label{fig:schnorr-sim}
\end{figure}

This is a valid simulator as the resulting $\mathcal A$ is random because $z$ and $c$ are chosen randomly. This means that the output of our simulator is effectively random and will have the same distribution over the transcript in a real interaction.
Note that the simulator presented here achieves only honest-verifier zero-knowledge (Definition
\ref{def:hvzk}) as the only input to the simulator is the statement to prove $H$. We can 
easily modify this simulator to achieve special honest-verifier zero-knowledge (Definition 
\ref{def:sigma}) by adding a challenge $c$ as an input to the simulator.



% CDS %
\section{CDS94}
In this section, we introduce the components necessary for the \cite{CDS94} compiler. In addition to a $\Sigma$-protocol, which 
is relevant to every compiler in this project, the CDS94 compiler requires a \emph{secret sharing scheme} and the compiler 
itself. In the following 2 subsections, we introduce these components.
\subsection{Shamir's Secret Sharing Scheme}
A \textit{secret sharing scheme} is a method of distributing a secret $s$ to $n$ participants in a way that no one participant has intelligible information about the secret. This is achieved by splitting up $s$ into \textit{shares}, distributing one share to each participant in a way that a subset of participants can reconstruct $s$. Subsets that can reconstruct the secret are called \textit{qualified sets}. For \textit{perfect} secret sharing schemes, like Shamir's, participants in the complement \textit{non-qualified} sets cannot obtain any information about the secret.

Shamir's secret sharing scheme \cite{DBLP:journals/cacm/Shamir79} is also what is known as a \textit{threshold sharing scheme}. These are schemes that produce  qualified sets of size $d$. Any $d$ out of $n$ participants can reconstruct the secret; with $d-1$ shares and less, no  information about the secret can be obtained. 

\textbf{Share Reconstruction for Shamir's}
Explain in more detail in the future, but essentially:
\begin{enumerate}
    \item Given secret $s$, unqualified set of shares $U$, and array of indexes for active clauses $A$
    \item Firstly, ensure that threshold is set to $N - d + 1$, where $d$ is the number of active clauses; in other words the threshold should be $k + 1$ where $k$ is the number of inactive clauses and $|U| = k$. 
    \item Construct a lagrange polynomial of degree $k$, with the secret $s$ and set $U$. 
    \item With this polynomial, interpolate at $x = i$ for $i \in A$ to determine the share for each $i \in A$. 
    \item Take these as $share(c_i)$, taking the first relevant number of bits if somehow the challenge is smaller than the shares, and extrapolating with random bits if the challenge is meant to be larger.
\end{enumerate}
\subsubsection{CDS94 Compiler}
In this paper, Cramer {\em{et al}} \cite{CDS94} presents 2 primary ways to compile $\Sigma$-protocols depending 
on the underlying choice of $\Sigma$-protocol and the secret sharing scheme. Our implementation makes use of Theorem 
8 of the paper because we choose to use Schnorr's protocol (Section \ref{sec:schnorr}) and Shamir's secret sharing 
(Section \ref{sec:sss}). More details regarding our implementation will be discussed in a further section. Now, we recall 
theorem 8 of \cite{CDS94} -- note that we alter the notation slightly from the original paper to be more consistent 
with the rest of this report.

% In our implementation, we will use Schnorr's discrete log protocol over Ristretto25519 and Shamir's secret sharing scheme to demonstrate the compilation of $\Sigma$-protocol into a $\Sigma$-protocol for the disjunction of $n$ statements. 
% We will attempt to make the implementation as general as possible to open up for the future possibility to take any $\Sigma$-protocol that suits our requirements and transform it disjunctive zero-knowledge $\Sigma$-protocol.

% \subsubsection{The Witness Indistinguishable (WI) compilation}

% In their paper, Cramer {\em{et al}} \cite{CDS94} presents 2 primary ways to construct a WI protocol from a $\Sigma$-protocol $\mathcal P$ (Theorem 8 and 9). 

% \begin{itemize}
%     \item Theorem 8 requires a smooth secret sharing scheme, and a HVZK $\Sigma$-protocol, while
%     \item Theorem 9 requires special honest-verifier ZK (SHVZK) with at least a semi-smooth secret sharing scheme.
% \end{itemize}

% Since, SSS is a smooth threshold secret sharing scheme (required for 8), and Schnorr's protocol is SHVZK (required for 9), we can choose either construction. 
% \textit{We will use \textbf{Theorem 8} in this project.}

\textbf{Theorem 8 \cite{CDS94}}. Given $\Pi = (A, Z, \phi)$, $\mathcal R_\Gamma$ and $\mathcal S(k)$ where

\begin{itemize}
    \item $\Pi$ is a 3-round public coin ($\Sigma$-protocol) HVZK proof of knowledge for relation $\mathcal R$.
    \item $\{\mathcal S(k)\}$ is a family of smooth secret sharing schemes.
    \item $\mathcal R_\Gamma$ is a relation where $((x_1,\ldots,x_m),(w_1,\ldots,w_m)) \in \mathcal R_\Gamma$ if and only if ($\iff$)
    \begin{itemize}
        \item all $x_i$'s are of the same length $k$, and 
        \item the set of indices $i$ for which $(x_i,w_i) \in \mathcal R$ corresponds to a qualified set for $S(k)$
    \end{itemize}
\end{itemize}

Then, there exists a $\Sigma$-protocol, $\Pi' = (A', Z', \phi')$, that is witness indistinguishable 
(Definition \ref{def:wi}) for the relation $\mathcal R_\Gamma$. The description of this protocol is outlined in 
Protocol \ref{prot:cds-compiler}. Interested readers can refer to the original paper for the proof of this theorem \cite{CDS94}. 

\begin{protocol}[label={prot:cds-compiler}]{CDS94 Compiler. A compiler for composing $n$ instances of a 
    $\Sigma$-protocol $\Pi$ into a single $\Sigma$-protocol $\Pi'$ that proves the \textbf{disjunction} 
    of these $n$ instances.}
    Let $A$ be the set of indices $i$ of the \textit{active clauses}. \\
    \textbf{Statement:} $x = x_1,\ldots, x_n$ \\
    \textbf{Witness:} $w = \{w_i\st i \in A\ \land (x_i, w_i) \in R\}$
    \begin{itemize}
        \item \textbf{First Round:} the Prover, $P$, computes $A'(x,w; r^P) \rightarrow a$ accordingly:
        \begin{itemize}
            \item For each $i \in \bar A$, run the simulator for $\mathcal P$ for the statement $x_i$ to produce the transcripts $(m_1^i, c_i, m_2^i)$.
            \item For each $i \in A$, compute $m_1^i \leftarrow A(x_i, w_i; r^P)$.
            \item Send $a \leftarrow (m_1^1, \ldots, m_1^n)$ to $V$.
        \end{itemize}
        \item \textbf{Second Round:} $V$ sends $c_0 \leftarrow \{0,1\}^\lambda$ to $P$. 
        \item \textbf{Third Round:} Prover computes $Z'(x,w,c;r^P) \rightarrow z$:
        \begin{itemize}
            \item Compute $Q \leftarrow Complete(c_0, U, A)$, where $U = \{c_i\st i \in \bar A\}$.
            \item Set $c_i = share(i)$, for each $share(i) \in Q$
            \item For $i \in A$, compute $m_2^i \leftarrow Z(x_i, w_i, c_i; r^P)$. 
            \item For $i \in \bar A$, $m_2^i$ has already been computed in the first round using the simulator.
            \item Send $z \leftarrow ((c_1, m_2^1), \ldots, (c_n,m_2^n))$ to $V$.
        \end{itemize}
        \item \textbf{Verification:} Verifier computes $\phi'(x,a,c_0,z) \rightarrow b$ as follows:
        \begin{itemize}
            \item Extract $c_i$ and $m_2^i$ from $z \leftarrow ((c_1, m_2^1), \ldots, (c_n,m_2^n))$
            \item Check that all conversations $(m_1^i, c_i, m_2^i)$ would lead to acceptance by the verifier in $\mathcal P$.
            \item Check that each $share(c_i)$ is consistent with the secret $c_0$ using the reconstruction algorithm in Definition \ref{def:sss}.
            \item If any of the checks fail return $b \leftarrow 0$; else return $b \leftarrow 1$.
        \end{itemize}
    \end{itemize}
\end{protocol}

Using this compiler, we can construct a $\Sigma$-protocol that has a communication complexity linear in the number of clauses.

% Stacking Sigmas %
\section{Stacking Sigmas}
Moving on, we introduce the concept of a \emph{partially-binding vector commitment scheme} 
and present the Stacking Sigmas compiler proposed by \cite{StackingSigmas}. This compiler aims to improve on the 
communication complexity of the \cite{CDS94} compiler, by reducing the communication size to $O(\log n)$, where $n$ is the 
number of clauses in the disjunction. 
% Partially Binding Vector Commitment from Discrete Log
% !TeX root = ../dissertation.tex
\subsubsection{Partially-Binding Vector Commitments}
In Section 5 of \cite{StackingSigmas}, Goel {\em{et al}} introduce the concept of partially-binding vector commitment schemes. 
These commitment schemes allow a Prover to make a commitment to a vector of length $l$ with $t$ binding positions; 
the remaining positions (indexes) in the vector are equivocable. 
Here, we recall the definition of these commitment schemes and refer interested readers to Figure 2 of \cite{StackingSigmas} for the construction of the general $t$-out-of-$l$ 
scheme. 

\subsubsection{Half-Bindings \& Q-Bindings}
In this work, we make use of the generic 1-out-of-$2^q$ construction provided in Section 5.3 of \cite{StackingSigmas} for our implementation of the Stacking Sigmas compiler. We use this 
particular construction as it allows us to yield commitments that are logarthmic in size (in bytes) to the original vector of messages. 
A core ingredient in this construction is a 1-out-of-2 partially-binding vector commitment scheme, which is used recursively to form a \textit{binary} "tree of commitments". 
For brevity, we refer to the generic construction as "q-bindings" and the 1-out-of-2 construction as "half-bindings".
The leaves of this tree are the original messages in our vector; intermediate nodes are the resulting commitments to the nodes' children using half-bindings. 
Essentially, intermediate commitments are regarded as messages as well, and are commited to recursively until there is only 1 root node -- the final commitment. 

While the source material provides the general $t$-out-of-$l$ construction for partially-binding vector commitments, it does not explicitly provide that for half-bindings. 
In Figure \ref{fig:half-binding}, we provide the construction for half-bindings derived from the general $t$-out-of-$l$ construction. 
The proof of correctness for this construction is trivial as it can be easily seen to be a specific case of the general construction when $t = 1$ and $l = 2$. 

\begin{figure}
  \centering
  \begin{construction}[]
    \begin{minipage}{0.45\linewidth}
      \underline{$pp \leftarrow$ Setup$(1^\lambda)$:}
      \begin{enumerate}
        \item $\G \leftarrow $ GenGroup$(1^\lambda)$; $g_0, h \samplefrom \G$
        \item $\textbf{return } pp \leftarrow (\G, g_0, h)$
      \end{enumerate}  
    \end{minipage}
    \begin{minipage}{0.5\linewidth}
      \underline{$(com, aux) \leftarrow$ EquivCom$(pp, ek, m_1, m_2)$:}
      \begin{enumerate}
        \item Extract $ck$ from $ek$
        \item $r \samplefrom \Z_{|\G|}^2$
        \item com $\leftarrow$ BindCom$(pp,ck,m_1,m_2,r)$
        \item \textbf{return} $(\text{com}, r)$
      \end{enumerate}  
    \end{minipage}

    \underline{$(ck, ek) \leftarrow$ Gen$(pp, B)$:}
    \begin{enumerate}
      \item Let $E = \{1,2\} \setminus B$ be the set of equivocal indexes. $|B| = 1$.
      \item Generate trapdoor $td$ for $i \in E: td \samplefrom \Z_{|\G|}, g_i \leftarrow h \cdot r_i$
      \item Interpolate the first element: $g_1 \begin{cases}
       g_2 - g_0 & \text{if $2 \in E$} \\
       g_1 & \text{if $1 \in E$}
      \end{cases}$
      \item $ck \leftarrow g_1$
      \item $ek \leftarrow (B, td, ck)$
      \item \textbf{return} $(ck, ek)$
    \end{enumerate}

    \underline{$com \leftarrow$ BindCom$(pp,ck, m_1,m_2, r)$:}
    \begin{enumerate}
      \item Interpolate $g_2$: $g_2 = g_1 + g_0$
      \item $(r_1, r_2) \leftarrow r$
      \item Commit individually $\textbf{for} j \in \{1,2\}: com_j \leftarrow h\cdot r_j + g_j \cdot m_j \in \G$
      \item \textbf{return} $(com_1,com_2)$
    \end{enumerate}

    \underline{$r \leftarrow$ Equiv$(pp, ek, m_1, m_2, m_1', m_2', aux)$:}
    \begin{enumerate}
      \item Extract $B$ and $td$ from $ek$; Let $E = \{1,2\} \setminus B$.
      \item Parse $aux = (r_1, r_2) \in \Z_{|\G|}^2$
      \item Interpolate $g_2 = g_1 + g_0$
      \item for $i \in B: r'_i \leftarrow r_i$
      \item for $i \in E: r'_i \leftarrow r_i - td \cdot (m_i' - m_i)$
      \item \textbf{return} $r' \leftarrow (r_1', r_2')$
    \end{enumerate}
  \end{construction}
  \caption{Construction of a 1-out-of-2 partially-binding vector commitment scheme.}
  \label{fig:half-binding}
\end{figure}


% Section on stacking sigmas - talk about how stacking works
\subsubsection{Stacking Sigmas Compiler}
In Section 6 of \cite{StackingSigmas}, Goel \emph{et al} introduce the two main properties of stackable $\Sigma$-protocols: \textit{extended honest verifier 
zero-knowledge}, and \textit{recyclable third round messages}. Moreover, they go on to show that many $\Sigma$-protocols satisfy these properties, 
proving that all $\Sigma$-protocols can be made EHVZK (Observation 1), and many natural $\Sigma$-protocols have recyclable third round messages. 

In this section, we state the definition of these two properties as shown in \cite{StackingSigmas}, but do not dive into the details of how a large class of 
$\Sigma$-protocols have these two properties. We encourage readers to refer to \cite{StackingSigmas} for the proofs of these properties and the observations 
related to them.

\begin{definition}[Extended Honest Verifier Zero-Knowledge -- EHVZK]
  The $\Sigma$-protocol $\Pi = (A, Z, \phi)$ is EHVZK if there exists a \textit{deterministic} polynomial time "extended simulator" algorithm 
  $\mathcal S^{\text{\tiny EHVZK}}(1^\lambda, x, c, z)$ such that the following two distributions are indistinguishable for any $(x,w) \in \mathcal R$ 
  and $c \in \{0,1\}^\kappa$:
  \begin{multline*}
    \left\{
      (a,c,z) \st r^P \samplefrom \{0,1\}^\lambda; a \leftarrow A(x, w; r^P); z \leftarrow Z(x,w,c; r^P)
    \right\} \\
    \approx
    \left\{
      (a,c,z) \st z \samplefrom \mathcal D_{x,c}^{(z)}; a \leftarrow \mathcal S^{\text{\tiny EHVZK}}(1^\lambda, x, c, z)
    \right\}
  \end{multline*}
  Intuitively, this means that the simulation starts with a specific third round message $z$, and determining a unique first round message $a$ with respect to 
  $z$ and a fixed challenge $c$.
\end{definition}

\begin{definition}[Recyclable 3rd Round Messages]
  A $\Sigma$-protocol $\Pi = (A, Z, \phi)$ for $\mathcal R$ has recyclable third round messages if for every challenge $c \in \{0,1\}^\kappa$, there is an 
  efficiently sampleable distribution $\mathcal D_c^{(z)}$:
  $$
    \mathcal D_c^{(z)} \approx \left\{z \st r^P \samplefrom \{0,1\}^\lambda; a \leftarrow A(x,w;r^P); z \leftarrow Z(x,w,c;r^P) \right\}\quad 
    \forall (x,w) \in \mathcal R 
  $$
\end{definition}
  Essentially, this means that the distribution of $z$ is not dependent on the statement $x$ and hence does not "leak information" about the statement. 
  Consequently, this means that $z$ can be reused for any statement $x$, even non-active clauses, and still hide the active clauses. 

  Now, we present the Stacking Sigmas Compiler for stacking disjunctions of the same protocol (Section 7 of \cite{StackingSigmas}). The compiler 
  presented in Protocol \ref{prot:stacksig-compiler} is identical to that in Figure 5 of \cite{StackingSigmas}, and is produced by Theorem 5 of \cite{StackingSigmas}.

  \paragraph{Theorem 5 of \cite{StackingSigmas}.} Given a \textit{stackable} $\Sigma$-protocol $\Pi = (A, Z, \phi)$ and a 
  1-out-of-$l$ binding vector commitment scheme, we can produce a compiled $\Sigma$-protocol $\Pi' = (A', Z', \phi')$ that is also \textit{stackable}. If 
  $\Pi$ is for the relation $\mathcal R: \mathcal X \times \mathcal W \rightarrow \{0,1\}$, then $\Pi'$ is for the relation $\mathcal R': \mathcal X^l \times 
  ([l] \times \mathcal W) \rightarrow \{0,1\}$, where $\mathcal R'((x_1,\ldots, x_n), (a,w)) := \mathcal R(x_a, w)$.
  The 
  description of $\Pi'$ is shown in Protocol $\ref{prot:stacksig-compiler}$, and the proof of this theorem can be found in \cite{StackingSigmas}.

\begin{protocol}[label={prot:stacksig-compiler}]{Stacking Sigmas Compiler. A compiler for composing $n$ instances of a 
  $\Sigma$-protocol $\Pi$ into a single $\Sigma$-protocol $\Pi'$ that proves the \textbf{disjunction} 
  of these $n$ instances.}
  \textbf{Statement:} $x = x_1,\ldots, x_n$ \\
  \textbf{Witness:} $w = (\alpha, w_\alpha)$
  \begin{itemize}
      \item \textbf{First Round:} the Prover, $P$, computes $A'(x,w; r^P) \rightarrow a$ accordingly:
      \begin{itemize}
          \item Parse $r^P = (r^P_\alpha \| r)$
          \item Compute $a_\alpha \leftarrow A(x_\alpha, w_\alpha; r^P_\alpha)$
          \item Set $\mathbf{v} = (v_1,\ldots, v_l)$, where $v_\alpha = a_\alpha$ and $\forall i \in [l] \setminus \alpha,\ v_i = 0$.
          \item Compute $(\textsf{ck}, \textsf{ek}) \leftarrow \textsf{Gen}(\textsf{pp}, B = \{\alpha\})$.
          \item Compute $(\textsf{com}, \textsf{aux}) \leftarrow \textsf{EquivCom}(\textsf{pp}, \textsf{ek}, \mathbf{v}; r)$.
          \item Send $a = (\textsf{ck, com})$ to the verifier. 
      \end{itemize}
      \item \textbf{Second Round:} $V$ sends $c \leftarrow \{0,1\}^\lambda$ to $P$. 
      \item \textbf{Third Round:} Prover computes $Z'(x,w,c;r^P) \rightarrow z$:
        \begin{itemize}
          \item Parse $r^P = (r^P_\alpha \| r)$
          \item Compute $z^* \leftarrow Z(x_\alpha, w_\alpha, c; r^P_\alpha)$
          \item For $i \in [l] \setminus \alpha$, compute $a_i \leftarrow \mathcal S^{\text{\tiny EHVZK}}(x_i, c, z^*)$. 
          \item Set $\mathbf{v}' = (a_1, \ldots, a_l)$
          \item Compute $r' \leftarrow \textsf{Equiv}(\textsf{pp, ek,} \mathbf{v}, \mathbf{v}'; \textsf{aux})$ (where $\textsf{aux}$ can be 
          regenerated with $\textsf{r}$).
          \item Send $z = (\textsf{ck}, z^*, r')$ to the verifier.
        \end{itemize}
      \item \textbf{Verification:} Verifier computes $\phi'(x,a,c_0,z) \rightarrow b$ as follows:
      \begin{itemize}
        \item Parse $a = (\textsf{ck, com})$ and $z = (\textsf{ck}, z^*, r')$.
        \item Set $a_i \leftarrow \mathcal S^{\text{\tiny EHVZK}}(x_i, c, z^*)$
        \item Set $\mathbf{v}' = (a_1, \ldots, a_l)$
        \item Compute and return: 
      \end{itemize}
      \[
          b = (\textsf{ck} \verify \textsf{ck}') \land \left(\textsf{com} \verify \textsf{BindCom}(\textsf{pp, ck}, \mathbf{v}'; r')\right) \land 
          \left(\bigwedge_{i \in [l]} \phi(x_i, a_i, c,z^*)\right)
        \]
  \end{itemize}
\end{protocol}

\paragraph{Communication Complexity.} By using the Q-Binding construction (Figure 4 of \cite{StackingSigmas}) together with Half-Bindings 
(Figure \ref{fig:half-binding}) we obtain a $1$-out-of-$2^q$ 
binding vector commitment scheme with communication complexity $q \cdot CC(\textsf{HalfBinding})$ 
(\textbf{Theorem 2 \& Corollary 1} in \cite{StackingSigmas}). According to the proofs in Theorem 5 \cite{StackingSigmas}, the communication complexity of 
$\Pi'$ is 
$CC(\Pi_l) = CC(\Pi) + |\textsf{ck}| + |\textsf{com}| + |\textsf{r'}|$, where the last three components are directly related to the choice of the 
partially-binding vector commitment. 

Since we are using a Q-Binding construction, we get the following communication complexity: 
$CC(\Pi_l) = CC(\Pi) + q(|\textsf{ck}_{1/2}| + |\textsf{com}_{1/2}| + |\textsf{r'}_{1/2}|)$. From Figure \ref{fig:half-binding}, we can see that 
$|\textsf{ck}_{1/2}| = |g|$, $|\textsf{com}_{1/2} = 2|g|$, and $|r'_{1/2}| = 2|z|$, where $g \in \G$ and $z \in \Z_{|\G|}$. In this case, $|g|$ 
depends on the number of bytes required to represent the group elements in $\G$, and $|z|$ depends on the size of the group $|\G|$. Therefore, 
$CC(\Pi_l) = CC(\Pi) + q(3|g| + 2|z|) = CC(\Pi) + \log_2(l)(3|g| + 2|z|)$.


% Speed Stacking %
% \subsection{Homomorphisms}
A homomorphism between two algebraic objects $A$ and $B$ is a function $f: A \rightarrow B$ which preserves their algebraic structure. 
\begin{enumerate}
    \item If elements in $A$ satisfy some algebraic equation involving addition or multiplication, their images in $B$ also satisfy the same algebraic equation
    \item The details of the definitions of homomorphisms depend on algebraic structures of $A$ and $B$
\end{enumerate}

\begin{definition}[Group Homomorphisms]
Suppose we have arbitrary groups $A$ and $B$, with the group operators $\circ_A$ and $\circ_B$ respectively. A group homomorphism is a function $f: A \rightarrow B$ such that $f(a_1 \circ_A a_2) = f(a_1) \circ_B f(a_2)$ for all $a_1,a_2 \in A$
\end{definition}
% % section on compressed sigma protocols (compression mechanism)
% \subsection{Compressed $\Sigma$-protocols are stackable}
\begin{protocol}[label={prot:compressed-sigs}]{Base $\Sigma$-protocol $\Pi_0$ for the relation 
$$\mathcal R_{\text{compressed}} = \{(g \in \G^N, P \in \G, y \in G_T; \mathbf{x} \in \Z_q^N): P = \mathbf{g^x}, y = f(\mathbf{x})\}$$ introduced in \cite{attema}} 
   \vspace{-0.5cm}
   \begin{align*}
       &\text{Prover}& 
       &&
       &\text{Verifier}& 
       \\
       &r \samplefrom \Z_q^N,\ t = f(\mathbf{r}),\ T = \mathbf{g^r}&
       &\overset  {t, T} {\xrightarrow{\hspace{3cm}}}&
       && 
       \\
       &&
       &&
       &c \samplefrom \Z_q&
       \\
       &&
       &\overset c {\xleftarrow{\hspace{3cm}}}&
       &&
       \\
       &\mathbf{z} \leftarrow c\mathbf{x} + \mathbf{r}&
       &&
       &&
       \\
       &&
       &\overset {\mathbf{z}} {\xrightarrow{\hspace{3cm}}}&
       &f(\mathbf{z}) \verify cy + t, \ \mathbf{g^z} \verify TP^c&
   \end{align*}
   \label{prot:base-compressed}
\end{protocol}
% % section on speed stacking compiler 
% \input{background/speed-stacker.tex}


% in-depth investigation of the literature

\chapter{Project Management}\label{ch:project-management}
%%%%%%%%%%%%%%%%%%%%%%%%%%%%%%%%%%%%%%%%%%%%%%%%%%%%%%%%%%%%%%%%%%%%%%%%%%%%%%%
%% Project-specific configuration
%%%%%%%%%%%%%%%%%%%%%%%%%%%%%%%%%%%%%%%%%%%%%%%%%%%%%%%%%%%%%%%%%%%%%%%%%%%%%%%

\author{Justin Tan}
\title{Disjunctive Zero Knowledge}
\supervisor{Nicholas Spooner}
\yearofstudy{3\textsuperscript{rd}}

%%%%%%%%%%%%%%%%%%%%%%%%%%%%%%%%%%%%%%%%%%%%%%%%%%%%%%%%%%%%%%%%%%%%%%%%%%%%%%%

% clear description of the stages of the life cycle undertaken
% description of the use of appropriate tools to support development
\chapter{Design \& Implementation}\label{sec:design}
In this chapter, we outline and discuss the design of the project, and our 
implementation of salient components. 
We first begin with a brief overview of our core requirements, and justify certain design choices.  
Next, we discuss of how we approach testing -- a crucial process in the 
verification of our software components. This is followed by a section discussing the design 
choices we make for our benchmarks (Section \ref{design:bench}). After which, 
we elaborate on the general approach we take in implementing the protocols described in 
the literature. Finally, we will elaborate on the design of important components 
and their implementation.

Here, we include the components associated with each compiler for easy reference:
% General design philosphy behind each component: when we use tratis and when we use structs

\begin{enumerate}
  \item CDS94 Compiler
  \begin{itemize}
    \item Schnorr's Protocol     
    \item Shamir's Secret Sharing 
    \item CDS94 Compiler 
  \end{itemize}
  \item Stacking Sigmas Compiler
  \begin{itemize}
    \item Schnorr's Protocol 
    \item Partially-Binding Vector Commitment scheme: Q-Binding of half-bindings 
    \item Self-Stacking Compiler
  \end{itemize}
\end{enumerate}

\section{Overview}\label{design:overview}
In this section, we 
elaborate on the core requirements that influence the design choices of 
every component, discuss our choice of programming language, and justify 
our choice of libraries 
that we use throughout various components in the project. 
\subsection{Core Requirements}
In the design of each component, we consider the following 
three core requirements: \emph{correctness}, \emph{generality}, and 
\emph{usability}. 

\paragraph{Correctness.} Unsurprisingly, correctness has the highest priority, 
as we want to ensure that we are accurately comparing 
the performance of the compilers. In Section \ref{design:testing}, 
we discuss how we test our compilers to ensure correctness. 

\paragraph{Generality.} This requirement is concerned with how easy it is for 
future developers to use our interfaces\footnote{Rust has a different 
terminology for interfaces called "traits". While traits and classic interfaces are not 
exactly the same, they share many commonalities. For the sake of simplicity, we will 
substitute terms in Rust associated with traits with those associated with classic 
interfaces. Where this is not possible, we will explicitly use Rust terminology and 
explain the difference.}
 and components for their own work. 
One of the motivations for this project is to lay the foundations for 
future researchers to conveniently compare their own implementations 
of existing or new compilers to ours. Additionally, this will also help 
developers to easily build on our work in the future. 
With this in mind, it is important that our code is easily extensible and 
has components (which may be shared) that are modular. Thus, we have 
organised our source code such that each component is an individual
library and can be selectively chosen. This is coupled with interfaces 
designed to be as general as possible, allowing
developers to choose which components they want to use, and 
which to implement themselves. 

\paragraph{Usability.} We strive to make our compilers easy to use, understand, 
and maintain using thorough documentation and convenient 
methods to easily use the components. In the following sections, we
discuss how we attempt to achieve this in each component. 

\subsection{Programming Language}
Our choice of programming language is Rust: a modern 
language that is designed with a focus on safety and reliability.
It promises memory safety, by using an ownership and borrowing
system unique to Rust, and it is a strongly typed language which 
helps to identify errors at compile-time. This helps to prevent errors such 
as integer overflows and null pointer references. Furthermore, 
Rust is known for its high performance and speed \cite{rust-book}, and 
has functional programming features such as pattern matching and 
closures. These features make Rust a good choice for our project,
particularly because many of the components we are implementing are 
defined mathematically which lend themselves well to functional 
programming. Meanwhile, robustness and speed are highly ideal properties
in any system, especially one that relies on cryptographic protocols.

Additionally, Hall-Andersen's \cite{MHAStackSig}
Stacking Sigmas compiler is also implemented in the language. This makes 
Rust a natural choice for our project, it allows us to easily compare 
our implementation to Hall-Andersen's. Rust has other useful benefits 
that pertain to more specific components of our project, such as 
testing; we highlight these in their respective sections.

\subsection{Libraries}\label{sec:libraries}
Throughout this project, a number of notable libraries are repeatedly 
used across different components. The few that we would like to highlight 
are:
\begin{enumerate}
  \item \texttt{curve25519-dalek} \cite{curve25519-dalek}: a Rust library that 
  provides group operations on the Ristretto group \cite{ristretto_web}. In many 
  of our protocols, a prime order group is required; we use this library as 
  the underlying prime order group in the concrete implementations of 
  our project. We choose the Ristretto group because it is a prime order 
  group constructed from a non-prime-order Edwards curve \cite{Edwards2007}
  (a family of elliptic curves), which is known for both security and speed. 
  At the same time, the Ristretto group does not suffer from the same issues as 
  other modern elliptic curve implementations that do not provide a prime-order group\footnote{
    Modern elliptic curve implementations usually provide a group of order $h \cdot q$, where 
    $h$ is a small cofactor (4 or 8), and $q$ is a large prime number. When using these 
    implementations for protocols that require a prime order group, the abstraction from 
    non-prime order group to prime order group is often handled by developers up the stack 
    (by users of the library). This means that they are not as familiar with the underyling 
      implementation, and may introduce vulnerabilities due to subtle design complications.
  }. 
  
  \item \texttt{rand\_chacha} \cite{rand-chacha}: a random number generator (RNG) that uses the 
  "ChaCha20" stream cipher \cite{bernstein2008chacha}. We use this to create RNGs 
  that are needed for our protocols. Notably, we chose this library 
  specifically because it also provides seedable RNGs, 
  which is useful for testing. 
  \item \texttt{group} \cite{group}: a library that provides general interfaces for 
  groups and fields. In the pursuit to make our components as general as 
  possible, we make use of the interfaces in this library to define the 
  kind of groups and fields that are accepted by our interfaces. When users 
  develop their own concrete implementation of our interfaces, they can 
  choose to use any group or field that implements the interfaces in this 
  library and are not restricted to the concrete implementations that we 
  provide. 
\end{enumerate} 


\section{Testing}\label{design:testing}
As mentioned in Project Management (Chapter \ref{ch:project-management}), we use Test-Driven Development to 
ensure correctness. By writing tests before features, we encourage focus on the 
requirements of our software and ensure that we design our components to target 
these requirements. Once these tests are written, we have an efficient workflow to 
consistently ensure our software is correct, even as we add new features. This improves
development speed and reduces the time needed to identify and fix bugs.

In addition, we perform extensive static analysis to complement testing and improve 
debugging. To support static analysis, we 
intentionally design our components and interfaces to reflect their 
mathematical definitions in the literature. A good example of this is our implementation of q-bindings and half-bindings
(Section \ref{design:qbinding}). Doing this allows us to easily verify that our implementations
adhere to their mathematical definitions as closely as possible, increasing the chances of 
identifying an issue if there is one.

\paragraph{Unit \& Integration Testing.}
We use the built-in testing framework in Rust to test our components. This framework allows us to write unit 
tests within the same source code file as our implementation, and conveniently run them within our 
IDE (VSCode) which consequently speeds up development. We also write integration tests with 
the same framework, except that they are often located in a separate file to emphasise that they are testing 
multiple components as these components have to be explicitly imported into the scope of the this testing module. 

\paragraph{Code Coverage.} To complement testing, we use code coverage reports to ensure that we are testing 
as much of our code as possible. Code coverage is a way to track and determine which sections of code (e.g. 
functions, branches, lines etc.) are executed. This is useful for us as it highlights sections of code that 
are neglected by our existing tests and ensures that we cover important test cases. We use the 
\texttt{cargo-llvm-cov} \cite{cargo-llvm-cov} tool to carry out code coverage analysis. 
This tool is a wrapper around the Rust compiler's in-built code coverage tool, 
providing a convenient way to generate code coverage reports as they integrate directly 
with our testing suite. 

\subsection{Extracting Requirements \& Writing Tests}
Throughout this project, the requirements of our software components are mostly derived 
from the literature with the exception of our benchmarks. Therefore, our testing 
strategy is centered around testing these requirements against our implementation. Of 
course, additional unit tests are written to test the correctness of specific functions
as well, but we will not discuss these in detail. 

Extracting these requirements is a manual process that requires a thorough 
understanding of the literature. Our approach begins with reading the literature and
writing down the requirements explicitly. This is a tedious process,
but it is necessary to ensure that we have a complete understanding of the protocols
and what is required of them. We then translate these requirements into tests. 
\begin{enumerate}
  \item Usually, the first step is to complete a function for setting up mock data 
  that is likely to be used in the tests. We design this function to take in a set of 
  parameters and returns a set of mock data. These can then be separately called in 
  each test, with parameters, to generate data specific for that test. 
  \item Next, we write the content of the tests themselves. Because we are desining our 
  components to model the mathematical definitions of the protocols, we can often easily 
  determine the exact methods to call. Often, a 
  challenge in TDD is not knowing what functions are available to test because they 
  are not yet implemented. However, this is not a problem in this project as we have 
  the definitions in the literature to refer to. There are two main kinds of tests 
  that we write:
  \begin{itemize}
    \item \textbf{Positive Tests:} These are tests that ensure the implementation
    works as expected. 
    \item \textbf{Negative Tests:} These are tests that ensure the implementation
    fails as expected.
  \end{itemize}
  \item Following this, we proceed to implement our components. This often starts with 
  the interfaces (if any), and then to any classes that are required. 
  \item Finally, we run the tests to ensure that they pass. If they do not, we 
  investigate and either fix the implementation or the tests.
\end{enumerate}

In Appendix \ref{code:cds-tests}, we provide an example of our tests with those 
for the CDS94 compiler. Often the first tests that are written target the main 
requirements that we glean from the literature. As development progresses, we 
continue to write tests or extend existing ones to ensure that we are testing 
comprehensively. 


\section{Benchmarks}\label{design:bench}
In our benchmarks, we are concerned with measuring the growth of certain 
metrics when the number of clauses in the disjunction increases. Suppose 
we have a disjunctive zero-knowledge proof $\Pi = (A, Z, \phi)$, then 
the metrics we are concerned with are as follows:
\begin{enumerate}
  \item \textbf{Communication Complexity:} 
  size of communication between the prover and verifier (in bytes). This is 
  the size of the messages $a \leftarrow A$, $z \leftarrow Z$, and $c \leftarrow \{0,1\}^\kappa$.
  \item \textbf{Prover Computation Time:}
  time taken for the prover to run the algorithms $A$ and $Z$ 
  \item \textbf{Verifier Computation Time:}
  time taken for the verifier to run the algorithm $\phi$.
\end{enumerate}

Crucially, we are also interested in learning the \textbf{total computational time} of 
the compiled proof, which we can easily computed by summing the prover and verifier 
running times. 

We target these three metrics as they are the main properties 
that are measurable and are used for evaluating the performance of the compiled 
proofs. Other properties such as security limitations (e.g. the requirement for a
trusted setup) are not measurable, and thus are not considered in our benchmarks.

The range we use for our clauses is $(2, 2^2, \ldots, 2^{13})$; this means that 
at each step, we double the number of clauses until we reach $2^{13} = 8192$ clauses. 
Theoretically, the maximum number of clauses our compilers can accept is much larger than 
this\footnote{The limit for CDS94 is $2^{252}$ (constrained by the size of the Ristretto group 
\cite{ristretto_web} that we use for Shamir's secret sharing), and for Stacking Sigmas the number 
is theoretically unbounded.}, however the proofs take too long to run after a certain number of clauses. 
After consulting the project supervisor, we decided that this is a suitable range for our benchmarks, 
as it allows us to measure a large range while ensuring benchmarks keep to a reasonable running time. 

\subsection{Benchmarking Tools}
Next, we discuss the choice of tools we use to construct and 
analyse our benchmarks.

\paragraph{Criterion.} We use the \texttt{criterion} library \cite{criterion} to construct our 
benchmark tests. Compared to the standard benchmarking library provided by the Rust language \cite{cargo-bench},
\texttt{criterion} is a statistics driven benchmarking library that provides a simple API for writing benchmarks, 
automatically providing the mean, standard deviation, and the confidence interval for each benchmark. 
It also includes useful defaults such as the number samples to run for each benchmark, a standard 
warmup time before collecting, and a HTML report generator. We provide examples of 
the HTML report generated automatically by \texttt{criterion} in Figure 
\ref{fig:criterion-report} below. 
Naturally, after some initial research we made the decision to use \texttt{criterion}
due to the abundance of useful features and applicability to our project. 

\paragraph{Helper Interfaces.} In our design, we also decided 
to include a \texttt{Message} interface that aids in the measurement of the 
communication size of each benchmark. This interface is inspired by one of 
the same name in Hall-Andersen's implementation \cite{MHAStackSig}. 

This interface requires that the type implement a few methods that are ultimately
used to measure the size of the messages sent between the prover and verifier. 
By implementing this interface, future developers can easily measure the size 
of these messages within benchmarks with a single call to the \texttt{size()} method. 
Furthermore, we improved on the interface by requiring that it "inherit" another interface, 
\texttt{Default}, which asserts that a method is written for constructing
a default instance of the type that implements \texttt{Message}. This provides 
further utility when writing tests and benchmarks, and helps to improve the 
overall usability of the compiler.

\paragraph{Data Analysis.} To analyse our data, we collect the mean 
of each benchmark provided by \texttt{criterion} \cite{criterion} and save it in a 
CSV (comma-separated values) file. We then employ the use of Python data analysis libraries 
\texttt{pandas} \cite{reback2020pandas,mckinney-proc-scipy-2010} 
and \texttt{plotly} \cite{plotly} to parse and plot the data 
respectively. We chose these two libraries as they are widely used in the data science, 
and have the necessary functions we need to parse data and produce high quality plots.

\begin{figure}
  \centering
  \includegraphics[width=0.9\linewidth]{../assets/html-report-example.png}
  \includegraphics[width=0.9\linewidth]{../assets/performance-change-examplel.png}
  \caption{Example of the HTML report generated by \texttt{criterion}.
  The first image shows an example of the additional statistics that are 
  conveniently provided in the report. The second image shows an example of 
  the performance comparison between two different benchmark runs. If the 
  change in performance falls within a certain threshold, the change is marked as 
  significant.
  }
  \label{fig:criterion-report}
\end{figure}

\subsection{Implementation of Benchmarks}
\label{sec:benchmarks-implementation}
Our goal is to measure the metrics shared earlier and to visualise the data on 
a plot. The main challenge in this task is to ensure that the benchmarks are 
accurate and reliable -- only necessary computation should be measured, and 
appropriate calculations should be used to measure communication size. Hence, 
tests are not suitable for evaluating the validity of our benchmarks, and 
will require a more manual process of accessing the data and ensuring that
the benchmarks are appropriate. In the following subsections, we discuss how 
we implement our benchmarks to achieve these goals. 

\subsubsection{Computation Time} Recall the definition of a $\Sigma$-protocol 
$\Pi = (A, Z, \phi)$. To ensure that we are measuring an accurate estimate 
of the computation time of the prover and verifier, we design and implement the 
benchmarks such that they isolate the execution of the prover algorithms ($A$ and $Z$), 
and the verifier algorithms ($\phi$ and $c \leftarrow \{0,1\}^\kappa$). The key idea 
is that we are implementing a benchmark, hence there is no need to run the 
$\Sigma$-protocol in 
order. We simply precompute the challenge $c$ from the second round of the protocol 
without measuring the time taken to compute it. Next, we execute algorithms $A$ and $Z$ 
in the prover's benchmark, collecting 
data on the computation time. Within this benchmark, 
there may be additional executions that we are not directly interested in but are 
necessary. 
Here we provide an example from the benchmarks of CDS94:


\begin{lstlisting}[language=rust]
  // Precompute challenge
  let challenge = Scalar::random(&mut verifier_rng);

  // Intialize variables to store results from first and third round 
  let mut message_a: Vec<CompressedRistretto> =
      Vec::new();
  let mut message_z: Vec<(usize, Scalar, Scalar)> =
      Vec::new();

  // Benchmarking environment for Prover benchmark
  group.bench_with_input(
      BenchmarkId::new("prover_bench", &proverparams),
      &mut proverparams,
      |b, s| {
          b.iter(|| {
              // Code within these braces are being benchmarked and measured
              // Should have negligible cost 
              let prover_rng = &mut s
                  .prover_rng
                  .clone();
              // First round of protocol
              let (transcripts, commitments) =
                  SelfCompiler94::first(
                      &s.statement,
                      &s.witness,
                      prover_rng,
                  );
              // Third round of protocol
              let proof = SelfCompiler94::third(
                  &s.statement,
                  transcripts,
                  &s.witness,
                  &s.challenge,
                  prover_rng,
              );
              // Should have negligible cost
              message_a = commitments;
              message_z = proof;
          })
      },
  );

  /// ...

  let verifier_rng = &mut ChaCha20Rng::from_entropy();

  // Benchmarking environment for Verifier benchmark
  group.bench_with_input(
      BenchmarkId::new("verifier_bench", &v_params),
      &v_params,
      |b, s| {
          b.iter(|| {
              // Re-execution of second round
              SelfCompiler94::<Schnorr>::second(
                  verifier_rng,
              );
              // Verification algorithm
              SelfCompiler94::verify(
                  &s.statement,
                  &s.message_a,
                  &s.challenge,
                  &s.message_z,
              )
          })
      },
  );
\end{lstlisting}

On line 17 and 36 of the code listing, we indicate (with comments) the blocks 
of code that are necessary and should have negligible cost. These are needed 
to ensure the protocol executes without error and that the output from the 
first and third round can be collected for the verifier's benchmark. Within the 
verifier's benchmark, we re-execute the second round of the protocol (line 53) to ensure
that the verifier's computation time for this round is also measured even though 
the resulting challenge is not used in the verification algorithm.

\subsubsection{Communication Size} 
As mentioned earlier, we use the \texttt{Message} interface to conveniently 
compute the communication size of each benchmark. Once the interface is 
implemented for each required type, measuring the size of the messages 
can be done with a single call to the \texttt{size()} method. 

\begin{lstlisting}[language=rust]
  let mut communication_sizes: Vec<usize> =
        Vec::with_capacity(Q - 1);

  /// ...

  communication_sizes.push(
      message_a.size()
          + message_z.size()
          + challenge.size(),
  );
\end{lstlisting}

To calculate the total communication size, we simply sum the sizes of the 
messages from each round of the protocol. We append this to a vector of 
communication sizes for each number of benchmarks, and use this to 
plot the growth of communication size against the number of clauses.  

\section{General Design Approach for Compiler Components}\label{design:approach}
Our general approach to designing our compiler-related software components is inspired by the 
well known "SOLID" design principles \cite{martin2000design} of object-oriented programming. 
Crucially, we ensure that our general design approach contributes directly to our core requirements 
of generality and usability. We split our compiler-related software components into two broad categories: 
general interfaces 
and concrete implementations. The general interfaces define the structure of protocols, while the 
concrete implementations are the actual implementations of these interfaces (with possible extensions).

Firstly, we want to ensure that that we design our concrete implementations such that 
they \textit{accept general types that implement these interfaces} instead of the concrete 
types directly.
For example, our Stacking Sigmas compiler is designed to 
accept a generic type that requires an implementation of the \texttt{SigmaProtocol} interface and the 
\texttt{EHVzk} interface, instead of accepting the concrete type, which in our case is Schnorr's protocol. 
This allows developers to use the Stacking Sigmas compiler with any $\Sigma$-protocol implementation, as 
long as they implement the relevant interfaces. This also supports our choice to use a 
microservices architecture and organising our code into separate modules for each component. 
By doing this, modules which define our interfaces should not be importing concrete types; modules 
defining concrete types should only import the interfaces they need to implement and explicitly 
import concrete types that are absolutely necessary. 

We also want to make sure that our interfaces have \textit{a single responsibility} and do not 
encompass too many functionalities. For example, we intentionally segregate the two interfaces 
\texttt{SigmaProtocol} and \texttt{EHVzk}, even though we require both of them to be implemented 
for the Stacking Sigmas compiler. This is so that we do not assume how future users will use 
our interfaces. For the CDS94 compiler, there is the \texttt{HVzk} interface. We know that the 
HVZK property can be thought of as a superset of the EHVZK property because a $\Sigma$-protocol that is 
EHVZK is HVZK, but not necessarily the other way around. With this in mind, we
should not need to implement the simulator as defined in the context of the \texttt{HVzk} interface if 
it is already defined in terms of \texttt{EHVzk}. 
Likewise, a future user may choose to develop a 
\texttt{SigmaProtocol} that has a different requirement for the zero-knowledge property. By 
segregating the interfaces into atomic functionalities, we provide users with more flexibility 
in how they use our interfaces.

Lastly, as mentioned in the testing section \ref{design:testing}, to support static analysis and 
improve the readability of our code, we \textit{design our interfaces to reflect the 
mathematical definitions} of the protocols as much as possible. This facilitates easy 
comparison between our implementations and the mathematical definitions, and helps to 
verify that our implementation is sound. 

% \section{$\Sigma$-protocols}
% We model $\Sigma$-protocols with an interface.
Firstly, we design a set of generic types associated with the interface which will fit into 
the definition of the methods in the interface. These types are:
\begin{itemize}
  \item \texttt{Statement}: Public information about the protocol.
  \item \texttt{Witness}: Private information about the protocol (only known to Prover).
  \item \texttt{MessageA}: The first message of the protocol, sent from prover to verifier.
  \item \texttt{Challenge}: The challenge sent by the verifier to the prover.
  \item \texttt{MessageZ}: The third message of the protocol, sent from prover to verifier. 
  \item \texttt{State}: For most $\Sigma$-protocols, there is a particular state associated 
  to the execution of the first message. This state is often used by the prover again in 
  the third message but must not be sent to the verifier.
\end{itemize}

In particular, the design of the generic \texttt{State} type is worth discussing in more detail.
It is not explicitly mentioned in the formal definition of $\Sigma$-protocols, as it is 
assumed that the prover is able to track and store the private values that they obtain and 
require in the protocol. In the implementation, this has to be explicitly modelled and 
we do this using a functional programming approach, instead of an object-oriented
programming (OOP) approach. 
Firstly, this is because the functional approach is simpler, as we do not need to 
implement separate interfaces for the prover and verifier and subsequently include them as fields
within the \texttt{SigmaProtocol} interface. Secondly, the functional approach is more flexible,
as it does not restrict the user to a particular way of implementing the prover or verifier.
The functional approach simply provides two values in the first message (the state and the 
actual message), and the user of our interface can decide how to use these values.

Now, we present the methods that are associated with our $\Sigma$-protocol interface. 
These methods are: 
\begin{itemize}
  \item \texttt{first}: the first message of the protocol. This models algorithm 
  $a \leftarrow A(x, w; r^P)$ 
  in Definition \ref{def:sigma}.
  \item \texttt{second}: the second message of the protocol. $c \leftarrow \{0,1\}^\kappa$.
  \item \texttt{third}: the third message of the protocol. $z \leftarrow Z(x, w, c; r^P)$. 
  \item \texttt{verify}: verifies the transcript. $b \leftarrow \phi(x, a, c, z)$.
\end{itemize}

In Appendix \ref{code:SigmaProtocol} we provide a code snippet of these methods within 
the interface, which outlines which generic types are given as input and which are 
returned as output. Referring to the code snippet, readers will observe that our 
methods almost model the algorithms in Definition \ref{def:sigma} exactly in terms of 
input and output. The only difference is with the \texttt{State} type that we have already 
discussed.

\subsection{Schnorr's Protocol}
As a $\Sigma$-protocol, our first step in our implementation for Schnorr's protocol is 
to create concrete definitions for 
the methods and generic types of the \texttt{SigmaProtocol} interface. We use the following 
concrete types in the interface: 
\begin{itemize}
  \item \texttt{Statement}: \texttt{Schnorr} -- a type that simply contains the "public key"
  (which is the group element $H = x \cdot G$) as a field. We use the \texttt{RistrettoPoint} 
  type from \texttt{curve25519-dalek} to represent the public key, and it is essentially a 
  point on an Edward's curve. 
  \item \texttt{Witness}: \texttt{Scalar} -- also from the \texttt{curve22519-dalek} library. It 
  represents elements of the prime field in the Ristretto group. 
  \item \texttt{MessageA}: \texttt{CompressedRistretto} -- essentially the same as the 
  \texttt{RistrettoPoint} type, but differs in its representation. \texttt{CompressedRistretto}
  is a compressed representation of the point, which is more efficient to store and
  transmit.
  \item \texttt{Challenge}: \texttt{Scalar}
  \item \texttt{MessageZ}: \texttt{Scalar}
  \item \texttt{State}: \texttt{Scalar} 
\end{itemize}

The key takeaway is that we assign concrete types to the associated generic types provided in 
the interface for our implementation of Schnorr. The methods in the interface are then 
defined according to our definition of Schnorr in Definition \ref{def:schnorr}. 

% \section{Schnorr's Protocol}\label{design:schnorr}

\section{Design \& Implementation of Key Components}
In this section, we will highlight the design and implementation 
of important components of our project in their respective subsections.

\subsection{Sigma Protocols}\label{design:sigma}
We model $\Sigma$-protocols with an interface.
Firstly, we design a set of generic types associated with the interface which will fit into 
the definition of the methods in the interface. These types are:
\begin{itemize}
  \item \texttt{Statement}: Public information about the protocol.
  \item \texttt{Witness}: Private information about the protocol (only known to Prover).
  \item \texttt{MessageA}: The first message of the protocol, sent from prover to verifier.
  \item \texttt{Challenge}: The challenge sent by the verifier to the prover.
  \item \texttt{MessageZ}: The third message of the protocol, sent from prover to verifier. 
  \item \texttt{State}: For most $\Sigma$-protocols, there is a particular state associated 
  to the execution of the first message. This state is often used by the prover again in 
  the third message but must not be sent to the verifier.
\end{itemize}

In particular, the design of the generic \texttt{State} type is worth discussing in more detail.
It is not explicitly mentioned in the formal definition of $\Sigma$-protocols, as it is 
assumed that the prover is able to track and store the private values that they obtain and 
require in the protocol. In the implementation, this has to be explicitly modelled and 
we do this using a functional programming approach, instead of an object-oriented
programming (OOP) approach. 
Firstly, this is because the functional approach is simpler, as we do not need to 
implement separate interfaces for the prover and verifier and subsequently include them as fields
within the \texttt{SigmaProtocol} interface. Secondly, the functional approach is more flexible,
as it does not restrict the user to a particular way of implementing the prover or verifier.
The functional approach simply provides two values in the first message (the state and the 
actual message), and the user of our interface can decide how to use these values.

Now, we present the methods that are associated with our $\Sigma$-protocol interface. 
These methods are: 
\begin{itemize}
  \item \texttt{first}: the first message of the protocol. This models algorithm 
  $a \leftarrow A(x, w; r^P)$ 
  in Definition \ref{def:sigma}.
  \item \texttt{second}: the second message of the protocol. $c \leftarrow \{0,1\}^\kappa$.
  \item \texttt{third}: the third message of the protocol. $z \leftarrow Z(x, w, c; r^P)$. 
  \item \texttt{verify}: verifies the transcript. $b \leftarrow \phi(x, a, c, z)$.
\end{itemize}

In Appendix \ref{code:SigmaProtocol} we provide a code snippet of these methods within 
the interface, which outlines which generic types are given as input and which are 
returned as output. Referring to the code snippet, readers will observe that our 
methods almost model the algorithms in Definition \ref{def:sigma} exactly in terms of 
input and output. The only difference is with the \texttt{State} type that we have already 
discussed.

\subsection{Schnorr's Protocol}
As a $\Sigma$-protocol, our first step in our implementation for Schnorr's protocol is 
to create concrete definitions for 
the methods and generic types of the \texttt{SigmaProtocol} interface. We use the following 
concrete types in the interface: 
\begin{itemize}
  \item \texttt{Statement}: \texttt{Schnorr} -- a type that simply contains the "public key"
  (which is the group element $H = x \cdot G$) as a field. We use the \texttt{RistrettoPoint} 
  type from \texttt{curve25519-dalek} to represent the public key, and it is essentially a 
  point on an Edward's curve. 
  \item \texttt{Witness}: \texttt{Scalar} -- also from the \texttt{curve22519-dalek} library. It 
  represents elements of the prime field in the Ristretto group. 
  \item \texttt{MessageA}: \texttt{CompressedRistretto} -- essentially the same as the 
  \texttt{RistrettoPoint} type, but differs in its representation. \texttt{CompressedRistretto}
  is a compressed representation of the point, which is more efficient to store and
  transmit.
  \item \texttt{Challenge}: \texttt{Scalar}
  \item \texttt{MessageZ}: \texttt{Scalar}
  \item \texttt{State}: \texttt{Scalar} 
\end{itemize}

The key takeaway is that we assign concrete types to the associated generic types provided in 
the interface for our implementation of Schnorr. The methods in the interface are then 
defined according to our definition of Schnorr in Definition \ref{def:schnorr}. 

\subsection{Shamir's Secret Sharing}\label{design:sss}
For our implementation of Shamir's Secret Sharing, we use Lagrange's interpolating 
polynomial to interpolate the $x$ and $y$ coordinates that correspond to shares. We 
chose to use Lagrange's interpolation formula because it is simple to implement, 
albeit not the most efficient. This trade-off is acceptable as the main goal of the project 
is to measure the performance change with respect to the number of clauses, and 
not the pure performance of the compiler. A more efficient implementation is therefore 
not a priority, and can always be implemented in the future if needed. 

Initially, we used an existing library \texttt{vsss-rs} \cite{vsss-rs} as our implementation of 
Shamir's secret sharing scheme. However, we encountered a few issues with this library. Firstly, 
the library does not provide the functionality to complete qualified shares given the secret 
and an unqualified set of shares. This is not surprising as this is not the main use case for 
a secret-sharing scheme like Shamir's. However, this is a necessary feature for our compiler
as we use it to ensure that the prover is unable to arbitrarily select challenges to cheat the 
verifier. We initially tackled this by simply adding the missing functionality to the library, 
in the form of new functions. However, we later found out about the second issue with the library:
its implementation of Lagrange's algorithm is inefficient.

In their implementation, they call the \texttt{invert} 
method in the inner loop of the algorithm resulting in $n^2$ calls to the function, 
where $n$ is the number of shares. The \texttt{invert} method is required to perform 
division on our group elements, but it is a costly operation. 
This can be avoided, however, by computing the numerator and denominator 
separately and performing the \texttt{invert} function only $n$ times -- outside of the inner loop. 
The following snippet of code shows the comparison of the two implementations. 

\begin{lstlisting}[language=rust]
  // Inefficient implementation
  for i in 0..n { // outer-loop
    for j in 0..n { // inner-loop
      if i != j
        basis *= (x - x[j]) * (x[i] - x[j]).invert()
      result += basis * y[i]
    }
  }

  // More efficient implementation
  for i in 0..n {
    for j in 0..n {
      if i != j {
        numerator *= x - x[j]
        denominator *= x[i] - x[j]
      }
    }
    result += numerator * denominator.invert() * y[i]
  }
\end{lstlisting}

Due to this issue, we created our own implementation of Shamir's secret-sharing scheme. 
We represent a Lagrange polynomial with two fields: 
$x$-coordinates and $y$-coordinates which are both vectors of the same length. 
For a polynomial with degree $d$, the vectors will have length $d$ as well. 

An alternative we considered is to represent the polynomial with a vector of coefficients, 
where the $i$-th element of the vector is the coefficient of the $i$-th degree term. This representation 
is useful in the method for distributing shares given the secret, as we can use Horner's method 
to evaluate the polynomial at a given point efficiently \cite{horner}. However, CDS94 does not 
rely on the secret sharing scheme's functionality for distributing shares and instead uses
the \textsf{Complete} algorithm (Definition \ref{def:sss-completion}) to complete a qualified set.
For this algorithm, we need to use polynomial interpolation as we are given the secret and an unqualified
set of shares (and equivalently points on the polynomial). 
Therefore, we choose to represent the polynomial in this way because it is more natural for use in 
Lagrange's interpolation formula (Definition \ref{def:lagrange}). 

\subsection{Half-Binding \& Q-Binding}\label{design:qbinding}
Before implementing half-binding and q-bindings, we implement the \texttt{PartialBindingCommScheme} interface 
to model the definition of partially-binding vector commitments provided in Definition \ref{def:comm_scheme}. 
This interface has methods that correspond to each function in the definition: \texttt{setup}, \texttt{gen}, 
\texttt{bind} (equivalent to \textsf{BindCom}), \texttt{equiv}, and \texttt{equivcom}. 
The interface also has a set of generic types to complete the definition of these methods (similar to how 
they function in the \texttt{SigmaProtocol} interface). These generic types are:
\begin{itemize}
  \item \texttt{PublicParams}: public parameters for the commitment scheme. 
  \item \texttt{BindingIndex}: a type that represents the binding index(es) of the commitment. 
  \item \texttt{CommitKey}: a type that represents the commitment key.
  \item \texttt{EquivKey}: a type that represents the equivocation key. 
  \item \texttt{Commitment}: a type that represents the commitment.
  \item \texttt{Randomness}: a type that represents the auxiliary values that are needed for equivocation. 
  \item \texttt{Msg}: a type that represents the vector of messages. 
\end{itemize}

With this interface, we implement \texttt{HalfBinding} and \texttt{QBinding} according to their 
construction in Figure \ref{fig:half-binding} of this report, and Figure 4 of \cite{StackingSigmas} 
respectively. The definition of q-binding is recursive in nature, where the base case is the 
half-binding scheme. In an attempt to improve the usability of our q-binding implementation, we introduce 
some overhead in how we represent the tree of half-bindings. Intuitively, this could simply be a 
recursive data structure, in which the every node has two children which continue down until the leaves. 
However, due to Rust's strict type system and our limited experience with the language, we were unable 
to implement this without heavily sacrificing the usability of the interface. 

Instead, we represent the tree as a set of vectors of data types, where each index in the vector represents a 
layer of the tree. We have multiple vectors for each concrete type for the half-binding implementation of the 
\texttt{PartialBindingCommScheme} interface. While this introduces some overhead as the vector must be 
traversed if we want to copy the data, it allows us to implement the interface in a way that has good 
usability and is still easy to understand. We hypothesize that this is the main reason our implementation of 
Stacking Sigmas is slower than Hall-Andersen's implementation \cite{MHAStackSig}. This will be discussed in 
Chapter \ref{sec:evaluation} in more detail.

\subsection{\texttt{HVzk} \& \texttt{EHVzk} Interface}
These two interface encompasses the honest-verifier and extended honest-verifier zero-knowledge 
property resepectively, which means that the only required method for these two interfaces 
is a simulator method (Definition \ref{def:sigma}) for the concrete $\Sigma$-protocol that 
implements this interface. As mentioned in the section on the general approach to design, 
we split this interface from the \texttt{SigmaProtocol} interface to adhere to the 
"single responsibility" principle in "SOLID" design \cite{martin2000design}. We provide 
code-snippets of these two interface in Appendix \ref{code:hvzk}.

\subsection{CDS94 Compiler}\label{design:cds94}
We implement the CDS94 compiler for disjunctions over the same $\Sigma$-protocol. This means we are concerned 
with many instances of the same $\Sigma$-protocol, even though it is possible to use CDS94 to compile 
disjunctions over different $\Sigma$-protocols. We do this to simplify our implementation and to focus on the 
most basic implementation of the CDS94 compiler. The disjunctions that this compiler supports are $d$-out-of-$n$
disjunctions, where $d$ is the minimum number of active clauses, and $n$ is the total number of clauses.
At a high level, the design of the CDS94 compiler is not complex, and simply follows the construction provided 
in Protocol \ref{prot:cds-compiler}. That said, we make certain design decisions that are worth highlighting.

Firstly, we cannot only implement the \texttt{SigmaProtocol} interface for the underlying $\Sigma$-protocol
that we want to use with the CDS94 compiler.
This is because the CDS94 compiler requires further restrictions on the generic 
types that is used during a concrete implementation. Notably, Shamir's secret sharing scheme requires that 
the shares passed to it have $x$ and $y$ coordinates that are field elements. This means we need to 
provide a way to map the generic \texttt{Challenge} type (from \texttt{SigmaProtocol} interface) to a 
type that implements the \texttt{PrimeField} interface from the \texttt{group} library (Section \ref{sec:libraries}).
For Schnorr's protocol, this is not a theoretical issue because the challenges that are used are field elements
and can be used directly with Shamir's secret sharing. However, in practice, the \texttt{Scalar} type from the
\texttt{curve25519-dalek} library does not implement the required interface. 

An alternative is to directly require that the \texttt{Challenge}
type parameter in the \texttt{SigmaProtocol} interface to implement the \texttt{PrimeField} interface. However,
this is arguably more restrictive than necessary. This is because, we may have existing implementations of a 
$\Sigma$-protocol that does not use a challenge that is a field element, but we still want to use it with the
CDS94 compiler. In this case, we can simply define a mapping function that maps the challenge to a field element
and use it with Shamir's secret sharing. 
Hence, to capture this relationship, we introduce the \texttt{Shareable} interface.

\paragraph{\texttt{Shareable}.} A simple interface to ensure type-level compatability between 
the generic types used in the CDS94 compiler and the generic types used in Shamir's secret sharing scheme.
This interface is defined as follows: 
\begin{lstlisting}[language=rust]
  pub trait Shareable: Clone + Default {
      type F: PrimeField;
      // Map the type to a prime field element
      fn share(&self) -> Self::F;
      // Derive the instance of the type from a field element
      fn derive(elem: Self::F) -> Self;
      // Convert field element into usize
      fn to_usize(elem: Self::F) -> usize;
  } 
\end{lstlisting}
\begin{itemize}
  \item The concrete type that implements this interface can be any type that also implements the \texttt{Clone} and
  \texttt{Default} traits. 
  \item The \texttt{F} type parameter is a type that implements the \texttt{PrimeField}
  interface from the \texttt{group} library. 
  \item The \texttt{share} method is a function that maps the generic
  type to the \texttt{F} type parameter.
\end{itemize}

This allows developers to implement the \texttt{Shareable} interface for any type that they want to use in the
CDS94 compiler, as long as it is possible to define a mapping function from the chosen type to one that 
implements the \texttt{PrimeField} interface. 
For example, in the case of Schnorr's protocol, the \texttt{Challenge} type is a \texttt{Scalar}
from the \texttt{curve25519-dalek} library. This type does not implement the \texttt{PrimeField} interface, but
we can implement the \texttt{Shareable} interface for it as follows:
\begin{lstlisting}[language=rust]
  impl Shareable for Scalar {
      type F = WrappedScalar;
      // ...
  }
\end{lstlisting}

where \texttt{WrappedScalar} is a wrapper type that implements the \texttt{PrimeField} interface for the 
\texttt{Scalar} type\footnote{We omit the implementation of the methods as they are specific to the 
\texttt{WrappedScalar} type and are not relevant to the discussion here.}.


\paragraph{\texttt{Composable}.} With the \texttt{Shareable} interface, we can now implement the \texttt{Composable}
interface for the base $\Sigma$-protocol to be compiled by CDS94. This interface is defined as follows:

\begin{lstlisting}[language=rust]
  pub trait Composable: SigmaProtocol<Challenge: Shareable> + HVzk {}
\end{lstlisting}

This is a simple interface that enforces the requirement for the base protocol to implement 
the \texttt{SigmaProtocol} interface, with a \texttt{Challenge} type implementing the \texttt{Shareable} 
interface, and to also implement the \texttt{HVzk} interface.

\paragraph{\texttt{CDS94}.} With the \texttt{Shareable} and \texttt{Composable} interfaces, we can now implement
the \texttt{CDS94} compiler. We require that the base $\Sigma$-protocol to be compiled implements the
\texttt{Composable} interface. Let this generic type corresponding to the base $\Sigma$-protocol be 
denoted by \texttt{S}. With \texttt{S}, we can define the generic types associated with the \texttt{SigmaProtocol} interface 
for CDS94:

\begin{itemize}
  \item \texttt{Statement}: \texttt{Statement94<S>} -- a type that contains public information 
  for the disjunctive proof: the number of clauses, the threshold, and the statement for each 
  clause of type \texttt{S}.
  \item \texttt{Witness}: \texttt{Witness94<S>} -- a type containing the private information for the 
  disjunctive proof: the witnesses for each clause, and the indexes of the active clauses 
  \item \texttt{MessageA}: \texttt{Vec<S::MessageA>} -- a vector of messages corresponding to the 
  first message in the $\Sigma$-protocol \texttt{S} for each clause.
  \item \texttt{Challenge}: \texttt{S::Challenge} -- the challenge type for the base $\Sigma$-protocol \texttt{S}.
  \item \texttt{MessageZ}: \texttt{Vec<(usize, S::Challenge, S::MessageZ)>} -- A vector of tuples containing 
  the index of the clause, the challenge corresponding to the clause, and the third message \texttt{z} 
  of that clause.
  \item \texttt{State}: \texttt{State94<S>} -- the relevant information generated from the first round
  of the disjunctive protocol that should be private to the Prover. This 
  contains the state of each instance of the base $\Sigma$-protocol (in each active clause), 
  the simulated challenge and 
  third messages $c_i$ and $z_i$ respectively for each inactive clause. 
\end{itemize}

With these generic types, we implement the methods according to the construction in Protocol \ref{prot:cds-compiler}.
Interested readers can compare how our implementation models the construction by referring to the source code 
in Appendix \ref{code:cds}.

\subsection{Stacking Sigmas Compiler}\label{design:stacking}
Similarly to CDS94, we implement the self-stacking compiler for Stacking Sigmas where 
we are concerned with disjunctions of the same $\Sigma$-protocol. An important distinction between 
the two compilers is that the Stacking Sigmas compiler is a 1-out-of-$n$ compiler, where $n$ is the
number of $\Sigma$-protocols in the disjunction. Goel \emph{et al} does include a brief outline of 
how a $d$-out-of-$n$ compiler may be constructed in Section 9 of \cite{StackingSigmas}, but we deem 
this out of scope for the project. 

In our implementation, we compile Schnorr's protocol, using the 
compiler protocol together with the q-binding scheme (Definition \ref{sec:qbinding}).
Overall, the methods associated with the \texttt{SigmaProtocol} library that we implement for the 
\texttt{SelfStacker} class follow the construction in Protocol \ref{prot:stacksig-compiler} closely. 
More interestingly, the generic types and interfaces involved are worth discussing in detail.

\texttt{\texttt{Message}.} This interface is useful in the Stacking Sigmas compiler 
for hashing the messages provided to the partial-binding vector commitments. This is 
an important efficiency requirement outlined in Definition \ref{def:comm_scheme}. 
With the \texttt{Message} interface, we require messages to the partial-binding vector
commitments to be writable to a buffer, which can then be hashed. Additionally, this 
provides a convenient interface for our benchmarks to measure the size of our messages. 

\begin{lstlisting}[language=rust]
  pub trait Message: Debug + Default + Clone {
      fn write<W: Write>(&self, writer: &mut W)
      where
          Self: Sized;

      fn size(&self) -> usize {
          let mut v: Vec<u8> = Vec::new();
          self.write(&mut v);
          v.len()
      }
  }  
\end{lstlisting}

Observing this code snippet, we can see that once we are able to use the \texttt{write} 
method to write bits of the message to a buffer, we can trivially obtain the size of 
the message by writing it to a vector and then obtaining the length of the vector.

As mentioned earlier, this interface was implemented by Hall-Andersen in 
\cite{MHAStackSig}, and we have extended it slightly. In particular, we further require types 
that implement the \texttt{Message} interface to also implement the \texttt{Default} interface. 
This is a useful addition 
because we can use this to create a default instance of the message type when we are 
using the q-binding scheme. Recall that this scheme requires the "0" message for the 
\textsf{EquivCom}, \textsf{Equiv}, and \textsf{BindCom} methods.

\paragraph{\texttt{Stackable}.} This interface is similar to the \texttt{Composable}
interface in the CDS94 compiler. This interface models a $\Sigma$-protocol that is stackable 
by the Stacking Sigmas compiler.

\begin{lstlisting}[language=rust]
  pub trait Stackable: SigmaProtocol<MessageA: Message, MessageZ: Message> + EHVzk {}
\end{lstlisting}

From its definition, we can see that the main requirements are the \texttt{SigmaProtocol} 
interface and the \texttt{EHVzk} interface, which corresponds correctly to the definition 
of a stackable $\Sigma$-protocol in Definition \ref{def:stackable}. 

\paragraph{\texttt{StackingSigmas}.} The concrete class of the Stacking 
Sigmas' "Self-Stacking" compiler is the \texttt{Self\-Stacker<S>} class, where \texttt{S} 
is a class that implements the \texttt{Stackable} interface. The generic types that 
are associated with this class are:
\begin{itemize}
  \item \texttt{Statement}: \texttt{StackedStatement<S>} -- a type that contains public information 
  for the disjunctive proof: the public parameters $pp$ for the q-binding scheme, 
  the height of the commitment tree, the number of clauses in total, and the 
  vector of messages which correspond to the statement 
  for each clause of type \texttt{S} .
  \item \texttt{Witness}: \texttt{StackedWitness<S::Witness>} -- a type containing the 
  private information for the disjunctive proof: the witnesses for the active clause (only 1), 
  and the index of the active clause.
  \item \texttt{MessageA}: \texttt{StackedA} -- the first message of the Stacking Sigmas 
  protocol (Definition \ref{prot:stacksig-compiler}). This is a 2-tuple, containing 
  the commitment key and the commitment. 
  \item \texttt{Challenge}: \texttt{S::Challenge} -- the challenge type for the base $\Sigma$-protocol \texttt{S}.
  \item \texttt{MessageZ}: \texttt{StackedZ<S>} -- the third message of the Stacking 
  Sigmas protocol containing: the commitment key, the third message ($z$) of the 
  underlying $\Sigma$-protocol, and the auxiliary value needed for the q-binding scheme.
  \item \texttt{State}: \texttt{State94<S>} -- the relevant information generated from the first round
  of the disjunctive protocol that should be private to the Prover. This 
  contains the state of the underlying $\Sigma$-protocol, the first message of the 
  underlying $\Sigma$-protocol, the initial messages given to the q-binding scheme 
  in the first round of the Stacking Sigmas protocol, the commitment key, the equivocation 
  key, and the auxiliary value for the q-binding scheme.
\end{itemize}

With these generic types, we implement the methods for the \texttt{SelfStacker} class
according to Protocol \ref{prot:stacksig-compiler}. Refer to Appendix \ref{code:stacksig} 
for the implementation.





% \section{CDS94 Compiler}\label{design:cds94}

% \section{Half-Binding \& Q-Binding}\label{design:qbinding}

% \section{Self-Stacking Compiler}\label{design:stacking}

% clear description and rationale of research methods used

% \chapter{Implementation}\label{sec:implementation}
% \section{Implementation}
\label{ch:implementation}

In this chapter, we describe the implementation of the design we described in \Cref{ch:design}. You should \textbf{not} describe every line of code in your implementation. Instead, you should focus on the interesting aspects of the implementation: that is, the most challenging parts that would not be obvious to an average Computer Scientist. Include diagrams, short code snippets, etc. for illustration. 

\begin{enumerate}
    \item Translating protocol description from math and words into code.
    \begin{enumerate}
        \item Difficult because it is not a 1-1 mapping
        \item Need to think about efficiency and speed
        \item Even after that, you may make mistakes in the implementation and debugging becomes difficult because you have to conduct static analysis to understand what has gone wrong. For example, in 1 case I didn't realise that I unknowingly passed the same vector into two functions that were supposed to receive two different vectors. Really requires understanding ....
    \end{enumerate}
    \item Debugging 
    \begin{enumerate}
        \item Even when implementing code according to the description of the protocol; there are times when it still doesn't work.
        \item A unique way of debugging where static analysis is the only reasonable way
    \end{enumerate}
\end{enumerate}

\chapter{Evaluation}\label{sec:evaluation}
% Describe the approaches you have used to evaluate that the solution you have 
% designed in \Cref{ch:design} and executed in \Cref{ch:implementation} actually 
% solves the problem identified in \Cref{ch:introduction}.

% While you can discuss unit testing etc. you have carried here a little bit, 
% that is the minimum. You should present data here and discuss that. This might 
% include \emph{e.g.} performance data you have obtained from benchmarks, survey 
% results, or application telemetry / analytics. Tables and graphs displaying this 
% data are good.
In this chapter, we evaluate our implementation with respect to testing and
the results of our benchmarks. 

\section{Testing}\label{eval:testing}
We implement tests for all major components of our system based on our approach 
described in Section \ref{design:testing}. These tests have been scrutinised to ensure 
they test the expected behaviour of each component, especially those 
that are based on protocols from the literature. Readers who are interested can 
run these tests using the \texttt{cargo test} command, to verify that all tests pass. 
Using code coverage analysis, we also ensure that we do not miss any important test 
cases and that we are testing as much of our code and as many edge cases as possible.
We use the following command to generate a code coverage report:

\begin{lstlisting}[language=bash]
  cargo llvm-cov --open --ignore-filename-regex libs/stacksig-compiler/src/rot256
\end{lstlisting}

\begin{figure}[t]
  \centering
  \includegraphics[width=\linewidth]{../assets/code-coverage.png}
  \caption{Code coverage of our project}
  \label{fig:coverage}
\end{figure}

We exclude the files within the \texttt{rot256} directory as those are related to Hall-Andersen's implementation \cite{MHAStackSig}. The code coverage report produced by this command can be seen in Figure \ref{fig:coverage}. 

From this report, we observe that we have only 58.89\% function 
coverage (we only test 58.89\% of the functions in our project), 80.10\% line 
coverage, and 72.45\% region coverage\footnote{Regions are blocks of code with 
respect to the compiler -- these can be multiple lines of code with no control flow 
or a single line of code. These regions for Rust are determined by LLVM. \cite{llvm-cov-explain}}. These numbers may appear to be low but upon closer inspection,
we observe the following:
\begin{enumerate}
  \item \textbf{Functions}: the majority of the functions that are not tested are either
  \begin{itemize}
    \item Getter and setter functions for classes (structs in Rust). These functions 
    simply return or set a value within the class, and are not tested as they are very simple. 
    \item or derived traits \cite{rust-book-derived-traits} in Rust. These are functions that are automatically generated according to "macros", which allow users to derive the implementation of specific interfaces for their classes automatically. These functions are not tested as they are automatically generated by the compiler. 
  \end{itemize}
  \item \textbf{Lines}: A portion of these lines include those for the Speed Stacking 
  compiler that we are still working on \cite{SpeedStacking}. These could not be 
  easily removed from the code coverage report, and affect the line coverage statistics.
  Additionally, many of these lines are automatically generated by the compiler from 
  derived traits. 
  \item \textbf{Regions}: While most regions are affected by the same reasons as functions and lines, it should be noted that there are some regions that should be 
  tested more thoroughly but are not. These regions are related to error-producing 
  branches of code, which should be tested to ensure that our implementation handles 
  errors correctly. The bulk of these regions are located in our implementation of 
  Shamir's secret sharing scheme. However, due to the low priority, and time constraints 
  of this project, we have not been able to test these cases thoroughly.
\end{enumerate}

Aside from these areas, all the main requirements of our compilers and respective 
software components are tested thoroughly. While it is ideal to achieve higher 
than 80\% code coverage in all categories, we believe that code coverage is useful in 
identifying potential areas of improvement for testing, but should not be used as the
sole metric for evaluating the quality of our tests. In fact, the code coverage report
was useful in highlighting that we previously did not implement a proper 
negative test for the CDS94 compiler as the block of code that returns false in 
the verification algorithm was not covered by any tests. This was subsequently fixed, 
improving the code coverage report marginally, but covering an important case 
for our compiler.

In summary, we assess that our quality and thoroughness of testing are sufficient 
for the scope of this project. However, we acknowledge that there is still room for
improvement in our testing, especially in the areas of error handling. 

\section{Benchmark Results}\label{eval:benchmarks}
In this section, we present the results of our benchmarks and discuss their 
implications. We have four benchmarks: two for the CDS94 compiler \cite{CDS94}, 
one for the Stacking Sigmas compiler \cite{StackingSigmas}, and one from Hall-Andersen's implementation \cite{MHAStackSig}. One of the two benchmarks for the CDS94 compiler compares 
the growth of the key metrics against the number of active clauses (instead of the number of clauses). We will first present and discuss the results regarding the 
growth in communication size across all benchmarks. After this, we will discuss
the results for the measurements of the growth in the computation time. In table \ref{tab:comm-size} and 
\ref{tab:comp-time}, we present the overall results of our benchmarks. These benchmarks were executed on an M1 MacBook Air (16GB) -- we provide the hardware specifications of our machine in Section \ref{sec:hardware-spec}. Note that the first row of Stacking Sigmas has no data as a result of a flaw in our current implementation of the q-binding scheme as mentioned in \ref{design:qbinding}.
\begin{table}[H]
  \centering\caption{Communication Size Results (in bytes)} 
  \label{tab:comm-size}
  \begin{tabular}{rcc}
    \toprule
    \textbf{Clauses} & \textbf{CDS94} & \textbf{Stacking Sigmas} \\
    \midrule
    2 & 224 & - \\
    4 & 416 & 256 \\
    8 & 800 & 320 \\
    16 & 1568 & 384 \\
    32 & 3104 & 448 \\
    64 & 6176 & 512 \\
    128 & 12320 & 576 \\
    256 & 24608 & 640 \\
    512 & 49184 & 704 \\
    1024 & 98336 & 768 \\
    2048 & 196640 & 832 \\
    4096 & 393248 & 896 \\
    8192 & 786368 & 960 \\
    \bottomrule
  \end{tabular}
\end{table}

\begin{table}[H]
\centering
\caption{Computation Time Results (in milliseconds)}
\label{tab:comp-time}
\begin{tabular}{rccccc}
\toprule
\textbf{Clauses} & \textbf{StackSig Prover} & \textbf{CDS Prover} & \textbf{StackSig Verifier} & \textbf{CDS Verifier} & \textbf{Rot256} \\
\midrule
2 & 0.3 & 0.081124 & 0.3 & 0.10955 & 3.4760 \\
4 & 0.64893 & 0.19991 & 0.64233 & 0.22346 & 7 \\
8 & 1.8222 & 0.45870 & 1.3788 & 0.47586 & 10.9 \\
16 & 4.2626 & 0.98115 & 2.8526 & 0.97444 & 15.021 \\
32 & 9.3257 & 2.1472 & 5.9700 & 2.0985 & 20.069 \\
64 & 19.616 & 5.0377 & 11.705 & 4.9214 & 37 \\
128 & 40.584 & 13.105 & 24.474 & 12.696 & 37.131 \\
256 & 82.243 & 37.381 & 47.087 & 36.691 & 54.213 \\
512 & 165.70 & 120.33 & 93.798 & 118.92 & 85 \\
1024 & 334.63 & 422.50 & 190.63 & 419.73 & 145.33 \\
2048 & 671.82 & 1572.9 & 375.45 & 1583.5 & 257.25 \\
4096 & 1359.1 & 6110.2 & 742.17 & 6081.9 & 482.62 \\
8192 & 2685.4 & 24141 & 1477.9 & 23884 & 932.34 \\
\bottomrule
\end{tabular}
\end{table}

\paragraph{Running benchmarks.} To run our benchmarks, we use the \texttt{cargo bench} command. To run the benchmarks for a specific compiler, we use the \texttt{{-}{-}bench} flag. Below we provide an example of how to run each benchmark available:
\begin{lstlisting}[language=bash]
  # run benchmark for CDS94 (as clauses increase)
  cargo bench --bench cds_benchmark 
  # run benchmark for CDS94 (as active clauses increase)
  cargo bench --bench cds_benchmark2 
  # run benchmark for Stacking Sigmas (as clauses increase)
  cargo bench --bench stacksig_benchmark 
  # run benchmark for Hall-Andersen's implementation (as clauses increase)
  cargo bench --bench rot256_benchmark
  # run all benchmarks available: requires a long time
  cargo bench
\end{lstlisting}

\subsection{Communication Size}\label{eval:comm}
We present two figures, Figure \ref{fig:cds_comm} and Figure \ref{fig:stacksig_comm},
for the CDS94 compiler and Stacking Sigmas compiler respectively. These figures 
show the growth in communication size for each compiler as the number of clauses
increases. Note that the $x$-axis of these figures is logarithmic. Therefore, a 
linear and logarithmic growth in communication size is represented by a quadratic and linear growth in the logarithmic scale respectively. 
\begin{figure}[h]
  \centering
  \includegraphics[width=\linewidth]{../assets/plots/cds_commsize.png}
  \caption{The growth in communication size for the CDS94 compiler. }
  \label{fig:cds_comm}  
\end{figure}

\begin{figure}[h]
  \centering
  \includegraphics[width=\linewidth]{../assets/plots/ss_commsize.png}
  \caption{The growth in communication size for the Stacking Sigmas compiler. }
  \label{fig:stacksig_comm}
\end{figure}

In both cases, we point out that the communication size growth is consistent with 
the theoretical proofs of the respective compilers. This further validates our
implementation of the compilers. The expected growth in communication size for
the CDS94 compiler is linear, which appears as a quadratic growth with a 
logarithmic scale. Meanwhile, the equivalent for the Stacking Sigmas compiler is
a logarithmic growth, which appears as a linear growth in the logarithmic scale. 

With an increasing number of active clauses, but a constant number of clauses, 
the communication size for the CDS94 compiler is not expected to change. This is
because the number of active clauses does not affect the number of elements in the 
vector of messages sent by the prover. This is supported by the results in Figure
\ref{fig:cds_comm2}. This figure shows that the communication size for 512 
clauses is constant at 53.28KB, regardless of the number of active clauses.

\begin{figure}[h]
  \centering
  \includegraphics[width=\linewidth]{../assets/plots/cds2_threshold.png}
  \caption{The growth in communication size for the CDS94 compiler as the number of active clauses increases (total clauses = 512).}
  \label{fig:cds_comm2}
\end{figure}

We do not provide a similar figure for Hall-Andersen's implementation as the 
results are identical to Figure \ref{fig:stacksig_comm}. In conclusion, the 
observed growth in communication size is consistent with the theoretical
proofs of the respective compilers, further supporting the correctness of our
implementations, and also validating these proofs.

\subsection{Computation Time}\label{eval:time}
Firstly, we present the comparison between the prover and verifier running 
time for each compiler. 

\paragraph{CDS94 Prover vs Verifier.} In Figure \ref{fig:cds_vs}, we compare the 
prover and 
verifier running time in our implementation of \cite{CDS94}. From the figure, 
we observe that both running times are quadratic as 
the number of clauses increases. This is mainly because polynomial interpolation
using Lagrange is the bottleneck with a quadratic running time. This 
directly affects both the prover and verifier running times, as both the 
third-round protocol and the verification algorithm use this algorithm. 

\begin{figure}[H]
  \centering
  \includegraphics[width=\linewidth]{../assets/plots/cds_vs.png}
  \caption{A comparison of the prover and verifier running time of CDS94. }
  \label{fig:cds_vs}
\end{figure}

\begin{figure}[H]
  \centering
  \includegraphics[width=\linewidth]{../assets/plots/ss_vs.png}
  \caption{A comparison of the prover and verifier running time of Stacking Sigmas.}
  \label{fig:stacksig_vs}
\end{figure}

\paragraph{Stacking Sigmas Prover vs Verifier.} In Figure \ref{fig:stacksig_vs}, we compare the prover and verifier running time
for the Stacking Sigmas compiler. The graph shows a linear growth in computation time 
for both the prover and verifier, which is expected. The computation time of the prover has 
a larger constant factor than that of the verifier. This is likely the case 
because the prover algorithms call the \textsf{Equiv} and \textsf{EquivCom} 
methods from the q-binding scheme, which are more computationally expensive than
the \textsf{Bind} method used in the verifier algorithm.

\paragraph{CDS94 vs Stacking Sigmas.} Next, we compare the prover algorithms of CDS94 and Stacking Sigmas. In Figure \ref{fig:provers_vs}, we see that up until 512 clauses, the CDS94 prover is more 
performant. However, due to the quadratic running time of the CDS94 prover, the 
Stacking Sigmas prover starts to outperform CDS94 after 512 clauses. 
Similarly in Figure \ref{fig:verifiers_vs}, where we compare the verifier algorithms 
across the two compilers, CDS94 is more performant when there are 256 or fewer clauses. 
 After which, Stacking Sigmas overtakes and takes less time to run. Note that both the 
$x$-axis and the $y$-axis are on the logarithmic scale. 

\begin{figure}[h]
  \centering
  \includegraphics[width=0.9\linewidth]{../assets/plots/vs_provers.png}
  \caption{Comparison of CDS94 Prover and Stacking Sigmas Prover Algorithms.}
  \label{fig:provers_vs}
\end{figure}

\begin{figure}[H]
  \centering
  \includegraphics[width=\linewidth]{../assets/plots/vs_verifiers.png}
  \caption{Comparison of the Verifier algorithms of CDS94 and Stacking Sigmas.}
  \label{fig:verifiers_vs}
\end{figure}

These results indicate that if computation speed is the priority,
 it may be more appropriate to use Stacking Sigmas 
when the number of clauses is large (i.e. 512 or more), compared to an implementation of 
CDS94 that uses Lagrange's interpolation \cite{lagrange} with Shamir's secret sharing 
\cite{DBLP:journals/cacm/Shamir79}. That said, this only applies to 1-out-of-n 
disjunctive zero-knowledge, as k-out-of-n disjunctions are not supported by this 
implementation of Stacking Sigmas.

\paragraph{Total Running Time.} Extending this comparison, we now compare the total running time (prover and verifier)
of the three implementations. In Figure \ref{fig:3log}, we see that the total running time of the CDS94 compiler, the Stacking Sigmas compiler, and Mathias Hall-Andersen's 
implementation. We observe that the total running time of the CDS94 compiler and both 
implementations of Stacking Sigmas is similar to what we observe in 
Figures \ref{fig:provers_vs} and \ref{fig:verifiers_vs}. Comparing the two 
Stacking Sigmas implementations, we notice that the running time of Hall-Andersen's 
implementation is better starting from 128 clauses. As mentioned in Section 
\ref{design:qbinding}, this difference in performance is likely due to our 
decision to trade off performance for usability.


\begin{figure}[H]
  \centering
  \includegraphics[width=0.9\linewidth]{../assets/plots/3log.png}
  \caption{Comparison of the total running time of the 3 compilers: \cite{CDS94}, 
  Stacking Sigmas \cite{StackingSigmas}, and
  Mathias Hall-Andersen's implementation \cite{MHAStackSig} of Stacking Sigmas.}
  \label{fig:3log}
\end{figure}

 

Our current implementation of the q-binding scheme leads to many calls to the 
\texttt{clone} function which duplicates data. The reason for this is mainly due 
to our lack of experience with Rust and can be improved in future work. Furthermore, 
in Hall-Andersen's implementation, a protocol for every power of two number of 
clauses requires the instantiation of a new type. This means that 
a disjunction of 2 clauses, 4 clauses, 8 clauses, and so on, must use a different 
type. This is not ideal as it requires the developer to write a lot of boilerplate
code, and also results in code duplication. For this reason,
we believe that the trade-off is sound as it allows for a more usable interface for 
developers and can be further improved. Moreover, this performance difference does 
not affect the correctness of the implementation or the goal of our benchmarks. 


\paragraph{CDS94 with an increase in Active Clauses.} Finally, we plot the running time of the CDS94 prover and verifier as the number of
\textbf{active clauses} increase (every power of 2 from 1 to 512 active clauses). 
For this benchmark, we fix the total number of clauses to 512.

\begin{figure}[H]
  \centering
  \includegraphics[width=\linewidth]{../assets/plots/cds_threshold.png}
  \caption{Comparison of prover and verifier algorithms as \textit{active clauses} increase. Note that the $x$-axis is on the logarithmic scale.}
  \label{fig:cds_threshold}
\end{figure}

In Figure \ref{fig:cds_threshold}, we see that the
prover running time peaks at 128 clauses, and drops abruptly at 512 clauses. Meanwhile, 
the verifier running time is constant as expected because the number of active clauses does not affect the verifier algorithm -- only the total number of clauses does.
The behaviour of the prover's running time can be directly attributed to 
our \textsf{Complete} algorithm (Definition \ref{def:sss-completion})
which we show in Listing \ref{lst:sss-completion}.

When completing the shares for the active clauses, we are required to interpolate 
at the point corresponding to the index of the active clause (this is the \texttt{x}
variable in the code listing). This interpolation 
uses the secret and the maximally unqualified set of shares to obtain the 
polynomial. As mentioned earlier, Lagrange's polynomial interpolation is an 
$O(n^2)$ algorithm, where $n$ is the number of points used to interpolate. 
Furthermore, the interpolation algorithm is called $k$ times where $k$ is the number of active 
clauses. This means that $n = \texttt{total clauses} - k + 1$. 

\begin{lstlisting}[language=rust, caption={Share Completion Algorithm},label={lst:sss-completion}]
  let remaining_shares = remaining_xs
      .iter()
      .map(|x| {
          let y = poly.interpolate(*x);
          Share { x: *x, y }
      })
      .collect();
\end{lstlisting}

Hence, the total running time of the \textsf{Complete} algorithm is $O(k \cdot n^2)$ (or
$g(k) = k \cdot (513 - k)^2$ in our case). By plotting function $g$ as a graph, we observe a 
direct correlation with what we observe in Figure \ref{fig:cds_threshold}. This is 
easy to see if we show this function as a table:

\begin{table}[H]
  \centering\caption{Active clauses \& $g(k)$}
  \vspace{0.5em}
  \begin{tabular}{lll}
    \toprule
    \textbf{Active Clauses} & \textbf{Threshold} & \textbf{$g$} \\
    \midrule
    1  & 512  & $262144$  \\
    2  & 511  & $522242$  \\
    4  & 509  & $1036324$  \\
    8  & 505  & $2040200$  \\
    16 & 497  & $3952144$ \\
    32 & 481  & $7403552$ \\
    64 & 449  & $12902464$ \\
    128 & 385 & $18972800$\\
    256 & 257 & $16908544$ \\
    512 & 1   & $512$ \\
    \bottomrule
  \end{tabular}
\end{table}

These results shed light on the effects of using Lagrange's polynomial interpolation algorithm 
with Shamir's secret sharing scheme. The CDS94 compiler \cite{CDS94} with Shamir's Secret Sharing 
and Lagrange's interpolation is most effective when the number of active clauses matches the 
total number of clauses. However, this is arguably an uninteresting case as it is no different to 
a \textit{conjunction} of clauses requiring the verifier to check every clause. In practice, we 
expect the number of active clauses to be in a smaller range. This indicates that Lagrange's 
interpolation algorithm is not the most ideal algorithm for Shamir's secret sharing scheme, 
as the running time increases cubically with the number of active clauses until it peaks 
at two powers of 2 below the total number of clauses.

\subsection{Summary of Results}
To summarise, we have first verified the proofs regarding the communication size of each 
compiler as the results here are consistent with the results in the literature 
\cite{CDS94,StackingSigmas}. 

Next, we have revealed important insights into the performance 
of the CDS94 and Stacking Sigmas compiler. It is clear that our implementation of the 
CDS94 compiler is significantly faster than the Stacking Sigmas compiler when the number of 
clauses is small. That said, the Stacking Sigmas compiler has a communication size that 
grows logarithmically with the number of clauses, while the CDS94 compiler's communication
size grows linearly. Hence, a trade-off has to be made between the need for a smaller 
communication size and the need for a faster computation time. 

In general, we believe that 
the Stacking Sigmas compiler is more suitable in any use case where a 1-out-of-n disjunctive proof is suitable. This is primarily because of the large savings 
in communication size (kilobytes of data) and the small increase in computation time 
(a few milliseconds) of using it when the number of clauses is small. When the number of 
clauses increases, CDS94 performs worse and the performance degradation is far more significant 
because of the large number of clauses. For use cases where a k-out-of-n threshold proof is 
required, the CDS94 compiler has to be used. Further improvements could be made to 
our implementation of CDS94 to further reduce computation time and improve performance 
as the number of active clauses or the total number of clauses increase. This will be interesting 
to explore in future work. 

\section{Limitations}\label{sec:limitations}
Despite achieving our primary objectives for this project, we had initially planned for this project to be more comprehensive and complete. Within our progress report (Chapter \ref{ch:progress-report}), we had initially planned to conduct basic security testing on our software using techniques such as fuzzing and static analysis. This is outlined in Section \ref{sec:security_testing}. However, due to delays in software development mainly due to difficulties in fixing bugs related to correctness, this became infeasible within our time constraints. As a result, our efforts toward security come only in the form of threat modelling and simple static analysis. With more time, more robust and comprehensive tests should be performed to fine-tune our implementation. Additionally, our implementation of the q-binding scheme is currently far from ideal and will require further improvements. As mentioned in previous sections, the two main issues are with the performance overhead and the mistake in our implementation for the case of a disjunction of 2 clauses. The latter is to be prioritised as it directly relates to the correctness of our implementation. 
% description of outcome of research (results)
% critical appraisal of project indicating rationale for methods used, lessons learnt
% during the project, evaluation of the outcome (review plan, and deviations)

\chapter{Conclusions}\label{sec:conclusions}
Overall, we accomplished our goals of firstly implementing the CDS94 \cite{CDS94} and Stacking 
Sigmas \cite{StackingSigmas} compiler, and secondly benchmarking their performance. Through our 
work, we have shown that Stacking Sigmas outperforms CDS94 by providing significant savings in 
the communication size of the proof, especially as the number of clauses increase. 
Furthermore, while CDS94 is faster than Stacking Sigmas when the number of clauses are small,
we show that the difference in speed is negligible because the number of clauses are small. 
When the number of clauses increase, the difference in speed is more significant and CDS94 
performs worse than Stacking Sigmas in this case. Moreover, we also reveal that the bottleneck of the CDS94 compiler is the secret sharing scheme,
and highlight how the choice of this scheme and its implementation has a significant impact on the performance
of the compiler. This is all while we uphold our goal of keeping the implementation of the compiler
easily usable and extensible, in order to lay a good foundation for future work in this area. 

Additionally, there were many useful lessons that we learnt throughout the project. Firstly, we 
learnt the importance of proper project management and planning for a successful project. The 
risk management plan that we created at the start of the project helped us to identify potential
risks and to mitigate them. This allowed us to focus on the project and to avoid any distractions
that could have hindered our progress significantly. Secondly, we realised the importance 
and convenience of good testing practices. Because of our testing practices, we were able to
identify and fix bugs early, accelerating our progress. Lastly, we were able to learn more about 
the inner workings of zero-knowledge proofs and other cryptographic concepts. Work in this field 
is often theoretical, and it is interesting to see how these concepts can be implemented in practice.

\section{Future work}
\label{sec:futurework}
There are many directions that we can take this project in the future. 

\paragraph{Speed Stacking.} Firstly, our work on the 
Speed Stacking compiler \cite{SpeedStacking} is still ongoing. We have implemented the base 
$\Sigma$-protocol of our choice (Compressed $\Sigma$-protocols or Folding Arguments) and 
are currently working on the compiler. Possible extensions to this, would be to use the same 
compiler for interactive oracle proofs \cite{iops}. 

\paragraph{Stacking Sigmas.} Next, our implementation of Stacking Sigmas is the Self-Stacking 
implementation, where we use the same $\Sigma$-protocol for each clause. We can also implement
the Cross-Stacking compiler, which is compatible with different $\Sigma$-protocols for each clause.
This would allow us to compare the performance of the compiler when using different $\Sigma$-protocols, 
and possibly test how the choice of $\Sigma$-protocols affects the performance of the compiler. Another 
extension is to explore Goel {\em et al.}'s idea for a $k$-out-of-$l$ compiler (Section 9 of \cite{StackingSigmas}).
This idea relies on the parallel execution of $k$ 1-out-of-$l$ proofs together with a family of hash functions to 
ensure that the verifier can check that each execution is for a unique clause. 

\paragraph{CDS94.} Similar to Stacking Sigmas, we can also explore the idea of compiling different $\Sigma$-protocols
with CDS94. However, a more closely related idea is to explore the idea of using a different secret sharing scheme or 
a more efficient implementation of Shamir's secret sharing scheme. This would allow us to compare the performance of
the compiler when using different secret sharing schemes, and test how the choice of secret sharing scheme
affects the performance of the compiler. \\

All these suggestions, rely on the benchmarking model that we have implemented. A useful extension to this would be to 
implement a benchmarking framework that provides a convenient testing interface for different compilers. This framework 
may have features to produce mock data automatically and simply take instances of compilers and base protocols as input, 
and subsequently run the benchmarking process. It would also be useful to automatically generate CSV files from 
the raw data collected from these benchmarks, as they are currently manually extracted. This will help to enhance 
the benchmarking process and make it far more convenient to use. 

\section{Acknowledgements}
\label{sec:acknowledgements}
We would like to thank our supervisor, Nicholas Spooner, for his invaluable guidance and support throughout the project.


\printbibliography

\appendix
\newpage
\section{Specification}
\input{specification/spec_content}


\section{Hardware Specification}
\label{sec:hardware-spec}
\lstinputlisting[]{assets/hardware-specs.txt}


\section{Threat Model}
\label{sec:threat-model-appendix}
\section{Overview}

\begin{figure}[h]
    \centering
    \includegraphics[width=\linewidth]{../assets/threat-model.png}
    \caption{Threat Model Diagram}
    \label{fig:threat_model}
\end{figure}

There should be no known way for adversaries to use the protocol in a way that it is not designed to. For example, under normal circumstances, the verifier should not have rewind access to a prover, this means that we will need to model state within the protocol and ensure that it is used properly (state-machine model). This should be a target of penetration tests. 




\end{document}